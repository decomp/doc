\documentclass[12pt, a4paper]{article}

% Preamble

\usepackage[utf8]{inputenc}
\usepackage[english]{babel}
\usepackage[margin=1in]{geometry}
\usepackage[parfill]{parskip}
\usepackage[hidelinks]{hyperref}
\bibliographystyle{ieeetr}

\title{Compositional Decompilation} % TODO: This title will change.
\author{Robin Eklind}
\date{2014-10-04}

% Document

\begin{document}

\maketitle

\pagebreak

\tableofcontents

\pagebreak

% === [ Introduction ] =========================================================

\section{Introduction}

A compiler is a piece of software which translates human readable high-level
programming languages (e.g. C) to machine readable low-level languages (e.g.
Assembly). In the usual flow of compilation, code is lowered through a set of
transformations from a high-level to a low-level representation, and not the
other way around. This paper introduces the concept of decompilation (originally
referred to as reverse compilation \cite{rev_comp}) which moves in the opposite
direction by lifting code from a low-level to a high-level representation.

\pagebreak

\bibliography{references}

\end{document}
