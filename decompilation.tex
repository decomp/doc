% TODO: Use British and not Amarican English:
%    s/artifact/artefact/

\documentclass[12pt, a4paper]{article}

% Preamble

\usepackage[utf8]{inputenc}
\usepackage[english]{babel}
\usepackage[margin=1in]{geometry}
\usepackage[parfill]{parskip}
\usepackage[hidelinks]{hyperref}
\usepackage{graphicx}
\bibliographystyle{ieeetr}

\title{Compositional Decompilation} % TODO: This title may change.
\author{Robin Eklind}
\date{2014-10-04}

% Document

\begin{document}

\maketitle

\begin{abstract}
TODO
\end{abstract}

\pagebreak

\tableofcontents

\pagebreak

%%%%%%%%%%%%%%%%%%%%%%%%%%%%%%%%%%%%%%%%%%%%%%%%%%%%%%%%%%%%%%%%%%%%%%%%%%%%%%%%
% Introduction
%%%%%%%%%%%%%%%%%%%%%%%%%%%%%%%%%%%%%%%%%%%%%%%%%%%%%%%%%%%%%%%%%%%%%%%%%%%%%%%%

% === [ Introduction ] =========================================================

\section{Introduction}

% --- [ Background ] -----------------------------------------------------------

\subsection{Background}

A compiler is a piece of software which translates human readable high-level programming languages (e.g. C) to machine readable low-level languages (e.g. Assembly). In the usual flow of compilation, code is lowered through a set of transformations from a high-level to a low-level representation. This paper introduces the concept of decompilation (originally referred to as reverse compilation \cite{rev_comp}) which moves in the opposite direction by lifting code from a low-level to a high-level representation.

% TODO: Find a good place to write about the applications of decompilation;
%    automated vuln audit \cite{vuln_audit}, etc.

% --- [ Project Aim and Objectives ] -------------------------------------------

\subsection{Project Aim and Objectives}

The aim of this project is to facilitate decompilation workflows using composition of language-agnostic decompilation passes; specifically the reconstruction of high-level control structures and, as a future ambition, expressions.

In order to achieve this aim, the author will:
\begin{enumerate}
	\item Review traditional decompilation techniques, including control flow analysis and data flow analysis.
	\item Critically evaluate a set of Intermediate Representations (IR), which describes low-, medium- and high-level language semantics, to identify one or more suitable for the decompilation pipeline.
	\item Analyze the formal grammar (language specification) of the IR to verify that it is unambiguous. If the grammar is ambiguous or if no formal grammar exists, produce a formal grammar. This objective is critical for language-independence, as the IR works as a bridge between different programming languages.
	\item Determine if any existing library for the IR satisfies our requirements; and if not develop one. The requirements would include a suitable in-memory representation, and support for on-disk file storage and arbitrary manipulations (inject, delete, etc) of the IR.
	\item Design and develop components which identify the structural patterns of high-level control structures using control flow analysis of the IR.
	\item Develop tools which perform one or more decompilation passes on a given IR. The tools will be reusable by other programming language environments as their input and output is specified by a formally defined IR.
	\item As a future ambition, design and develop components which perform expression propagation using data flow analysis of the IR.
\end{enumerate}

% --- [ Deliverables ] ---------------------------------------------------------

\subsection{Deliverables}

The source code and the report of this project have been released into the public domain \cite{cc0} and are made available at:
\begin{itemize}
	\item \url{https://github.com/mewlang/llvm}
	\item \url{https://github.com/mewpaper/decompilation}
\end{itemize}

% TODO: Specify the deliverables of the project after the development phase has finished. Use the text below as a basis.
%
% The following documents have been produced:
% * Project report.
% * Formal grammar for the LLVM IR language; refer to objective 3.
%
% The following system artifacts have been developed:
% * Library for interacting with LLVM IR; refer to objective 4.
% * Library for reconstructing high- level control structures from LLVM IR; refer to objective 5.
% * Tool for performing one or more decompilation passes on a given IR; refer to objective 6.

% --- [ Disposition ] ----------------------------------------------------------

\subsection{Disposition}

%%%%%%%%%%%%%%%%%%%%%%%%%%%%%%%%%%%%%%%%%%%%%%%%%%%%%%%%%%%%%%%%%%%%%%%%%%%%%%%%
% Literature Review
%%%%%%%%%%%%%%%%%%%%%%%%%%%%%%%%%%%%%%%%%%%%%%%%%%%%%%%%%%%%%%%%%%%%%%%%%%%%%%%%

% === [ Literature Review ] ====================================================

\section{Literature Review}

% --- [ Terminology ] ----------------------------------------------------------

\subsection{Terminology}


% ~~~ [ Intermediate Representation ] ~~~~~~~~~~~~~~~~~~~~~~~~~~~~~~~~~~~~~~~~~~

\subsubsection{Intermediate Representation}

% ~~~ [ Static Single Assignment Form ] ~~~~~~~~~~~~~~~~~~~~~~~~~~~~~~~~~~~~~~~~

\subsubsection{Static Single Assignment Form}

\cite{ssa_decomp}

% ~~~ [ Control Flow Graph ] ~~~~~~~~~~~~~~~~~~~~~~~~~~~~~~~~~~~~~~~~~~~~~~~~~~~

% TODO: Control-Flow Graphs or Control Flow Graphs?

\subsubsection{Control Flow Graph}

% --- [ Decompilation Phases ] -------------------------------------------------

\subsection{Decompilation Phases}

A core principle utilized in decompilers is the separation of concern through the use of abstractions, and a lot of work involves translating into and breaking out of various abstraction layers. In general a decompiler is composed of distinct phases which parses, analyzes or transforms the input. These phases are conceptually grouped into three modules to separate concerns regarding source machine language and target programming language. The front-end module parses object files and translates their platform dependent assembly into a platform independent intermediate representation (IR). The middle-end module performs a set of decompilation passes to lift the IR, from a low-level to a high-level representation, by reconstruction high-level control structures and expressions. Finally the back-end module translates the high-level IR to a specific target programming language. Figure \ref{overview} gives an overview of the decompilation modules and visualizes their relationship.

\begin{figure}[htbp]
	\includegraphics[width=\textwidth]{inc/overview.png}
	\caption{The front-end module accepts several object file formats (PE, ELF, …) as input and translates their platform dependent assembly (x86, ARM, …) to a low-level IR. The middle-end module then lifts the low-level IR to a high-level IR through a set of decompilation passes. Finally the backend-module translates the high-level IR into one of several target programming languages (C, Go, Python, …).}
	\label{overview}
\end{figure}

The remainder of this section describes the distinct decompilation phases, most of which have been outlined by Cristina Cifuentes in her influential paper ``Reverse Compilation Techniques'' \cite{rev_comp}.

% ~~~ [ Object File Analysis ] ~~~~~~~~~~~~~~~~~~~~~~~~~~~~~~~~~~~~~~~~~~~~~~~~~

\subsubsection{Object File Analysis}

% ~~~ [ Syntax Analysis ] ~~~~~~~~~~~~~~~~~~~~~~~~~~~~~~~~~~~~~~~~~~~~~~~~~~~~~~

\subsubsection{Syntax Analysis}

% ~~~ [ Semantic Analysis ] ~~~~~~~~~~~~~~~~~~~~~~~~~~~~~~~~~~~~~~~~~~~~~~~~~~~~

\subsubsection{Semantic Analysis}

% ~~~ [ Intermediate Code Generation ] ~~~~~~~~~~~~~~~~~~~~~~~~~~~~~~~~~~~~~~~~~

\subsubsection{Intermediate Code Generation}

% ~~~ [ Control Flow Graph Generation ] ~~~~~~~~~~~~~~~~~~~~~~~~~~~~~~~~~~~~~~~~

\subsubsection{Control Flow Graph Generation}

% ~~~ [ Data Flow Analysis ] ~~~~~~~~~~~~~~~~~~~~~~~~~~~~~~~~~~~~~~~~~~~~~~~~~~~

\subsubsection{Data Flow Analysis}

\cite{type_decomp}

% ~~~ [ Control Flow Analysis ] ~~~~~~~~~~~~~~~~~~~~~~~~~~~~~~~~~~~~~~~~~~~~~~~~

\subsubsection{Control Flow Analysis}

% === [ Related Work ] =========================================================

\section{Related Work}

% --- [ Academic Prototypes ] --------------------------------------------------

\subsection{Academic Prototypes}

% ~~~ [ The dcc Decompiler ] ~~~~~~~~~~~~~~~~~~~~~~~~~~~~~~~~~~~~~~~~~~~~~~~~~~~

\subsubsection{The \texttt{dcc} Decompiler}

\cite{rev_comp}

% ~~~ [ The opencl Decompiler ] ~~~~~~~~~~~~~~~~~~~~~~~~~~~~~~~~~~~~~~~~~~~~~~~~

\subsubsection{The \texttt{opencl} Decompiler}

\cite{decomp_llvm}

% ~~~ [ The opencl Decompiler ] ~~~~~~~~~~~~~~~~~~~~~~~~~~~~~~~~~~~~~~~~~~~~~~~~

\subsubsection{The \texttt{C-Decompiler}}

\cite{readable_decomp}

\subsubsection{The \texttt{REcompile} Decompilation Framework}

\cite{recompile}

% --- [ Open Source Projects ] -------------------------------------------------

\subsection{Open Source Projects}

% ~~~ [ Boomerang ] ~~~~~~~~~~~~~~~~~~~~~~~~~~~~~~~~~~~~~~~~~~~~~~~~~~~~~~~~~~~~

\subsubsection{Boomerang}

\cite{boomerang}

% ~~~ [ radare ] ~~~~~~~~~~~~~~~~~~~~~~~~~~~~~~~~~~~~~~~~~~~~~~~~~~~~~~~~~~~~~~~

\subsubsection{radare}

\cite{radare}

% --- [ Commercial Products ] --------------------------------------------------

\subsection{Commercial Products}

% TODO: Add Mac-specific Decompiler.

% ~~~ [ Hex-Rays Decompiler ] ~~~~~~~~~~~~~~~~~~~~~~~~~~~~~~~~~~~~~~~~~~~~~~~~~~

\subsubsection{Hex-Rays Decompiler}

\cite{hexrays}

% --- [ Decompilation as a Services ] ------------------------------------------

\subsection{Decompilation as a Services}

% ~~~ [ The Retargetable Decompiler ] ~~~~~~~~~~~~~~~~~~~~~~~~~~~~~~~~~~~~~~~~~~

\subsubsection{The Retargetable Decompiler}

\cite{retargetable_decomp}

% ==============================================================================
% Design and Implementation
% ==============================================================================

% === [ Requirements ] =========================================================

\section{Requirements}

% === [ Methodology ] ==========================================================

\section{Methodology}

% --- [ Evolutionary Development and Throwaway Prototyping ] -------------------

\subsection{Evolutionary Development and Throwaway Prototyping}

% ~~~ [ Revision Control System ] ~~~~~~~~~~~~~~~~~~~~~~~~~~~~~~~~~~~~~~~~~~~~~~

\subsubsection{Revision Control System}

% --- [ Public Issue Tracking ] ------------------------------------------------

\subsection{Public Issue Tracking}

% --- [ Continuous Integration ] -----------------------------------------------

\subsection{Continuous Integration}

% --- [ Test Driven Design ] ---------------------------------------------------

\subsection{Test Driven Design}

% === [ Design ] ===============================================================

\section{Design}

% --- [ Choice of Programming Language ] ---------------------------------------

\subsection{Choice of Programming Language}

% TODO: Mention software composition.

% --- [ Choice of Intermediate Representation ] --------------------------------

% TODO: Find synonym for "Choice of".

\subsection{Choice of Intermediate Representation}

% --- [ Decompiler Pipeline ] --------------------------------------------------

\subsection{Decompiler Pipeline}

% --- [ System Architecture ] --------------------------------------------------

\subsection{System Architecture}

% TODO: Mention package division.

% === [ Development ] ==========================================================

\section{Development}

% --- [ Formal Grammar of LLVM IR ] --------------------------------------------

\subsection{Formal Grammar of LLVM IR}

% --- [ Documentation ] --------------------------------------------------------

\subsection{Documentation}

% --- [ Ideomatic Language Use ] -----------------------------------------------

\subsection{Ideomatic Language Use}

% --- [ TODO ] -----------------------------------------------------------------

\subsection{TODO}

% --- [ Testing ] --------------------------------------------------------------

\subsection{Testing}

% ~~~ [ Fuzzing the LLVM IR Parser ] ~~~~~~~~~~~~~~~~~~~~~~~~~~~~~~~~~~~~~~~~~~~

\subsubsection{Fuzzing the LLVM IR Parser}

%%%%%%%%%%%%%%%%%%%%%%%%%%%%%%%%%%%%%%%%%%%%%%%%%%%%%%%%%%%%%%%%%%%%%%%%%%%%%%%%
% Evaluation and Conclusion
%%%%%%%%%%%%%%%%%%%%%%%%%%%%%%%%%%%%%%%%%%%%%%%%%%%%%%%%%%%%%%%%%%%%%%%%%%%%%%%%

% === [ Evaluation ] ===========================================================

\section{Evaluation}

% --- [ Evaluation against Requirements ] --------------------------------------

\subsection{Evaluation against Requirements}

% --- [ TODO ] -----------------------------------------------------------------

\subsection{TODO}

% --- [ Third Party Adaptation ] -----------------------------------------------

\subsection{Third Party Adaptation}

% --- [ Profiling and Benchmarks ] ---------------------------------------------

\subsection{Profiling and Benchmarks}

% === [ Future Work ] ==========================================================

\section{Future Work}

% === [ Conclusion ] ===========================================================

\section{Conclusion}

% --- [ Introduction ] ---------------------------------------------------------

\subsection{Introduction}

% --- [ Personal Reflections ] -------------------------------------------------

\subsection{Personal Reflections}

% --- [ Personal Development ] -------------------------------------------------

\subsection{Personal Development}

% --- [ Final Thoughts ] -------------------------------------------------------

\subsection{Final Thoughts}

\pagebreak

% === [ References ] ===========================================================

\section{References}

\bibliography{references}

\end{document}
