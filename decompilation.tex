\documentclass[12pt, a4paper]{article}

% Preamble

\usepackage[utf8]{inputenc}
\usepackage[english]{babel}
\usepackage[margin=1in]{geometry}
\usepackage[parfill]{parskip}
\usepackage[hidelinks]{hyperref}
\bibliographystyle{ieeetr}

\title{Compositional Decompilation} % TODO: This title will change.
\author{Robin Eklind}
\date{2014-10-04}

% Document

\begin{document}

\maketitle

\begin{abstract}
TODO
\end{abstract}

\pagebreak

\tableofcontents

\pagebreak

%%%%%%%%%%%%%%%%%%%%%%%%%%%%%%%%%%%%%%%%%%%%%%%%%%%%%%%%%%%%%%%%%%%%%%%%%%%%%%%%
% Introduction
%%%%%%%%%%%%%%%%%%%%%%%%%%%%%%%%%%%%%%%%%%%%%%%%%%%%%%%%%%%%%%%%%%%%%%%%%%%%%%%%

% === [ Introduction ] =========================================================

\section{Introduction}

% --- [ Background ] -----------------------------------------------------------

\subsection{Background}

A compiler is a piece of software which translates human readable high-level
programming languages (e.g. C) to machine readable low-level languages (e.g.
Assembly). In the usual flow of compilation, code is lowered through a set of
transformations from a high-level to a low-level representation, and not the
other way around. This paper introduces the concept of decompilation (originally
referred to as reverse compilation \cite{rev_comp}) which moves in the opposite
direction by lifting code from a low-level to a high-level representation.

% --- [ Purpose ] --------------------------------------------------------------

\subsection{Purpose}

% --- [ Project Deliverables ] -------------------------------------------------

\subsection{Project Deliverables}

The source code and the report of this project have been released into the
public domain \cite{CC0}, and are made available at:
\begin{itemize}
	\item \url{https://github.com/mewlang/llvm}
	\item \url{https://github.com/mewpaper/decompilation}
\end{itemize}

% --- [ Disposition ] ----------------------------------------------------------

\subsection{Disposition}

%%%%%%%%%%%%%%%%%%%%%%%%%%%%%%%%%%%%%%%%%%%%%%%%%%%%%%%%%%%%%%%%%%%%%%%%%%%%%%%%
% Literature Review
%%%%%%%%%%%%%%%%%%%%%%%%%%%%%%%%%%%%%%%%%%%%%%%%%%%%%%%%%%%%%%%%%%%%%%%%%%%%%%%%

% === [ Literature Review ] ====================================================

\section{Literature Review}

% --- [ Terminology ] ----------------------------------------------------------

\subsection{Terminology}


% ~~~ [ Intermediate Representation ] ~~~~~~~~~~~~~~~~~~~~~~~~~~~~~~~~~~~~~~~~~~

\subsubsection{Intermediate Representation}

% ~~~ [ Static Single Assignment Form ] ~~~~~~~~~~~~~~~~~~~~~~~~~~~~~~~~~~~~~~~~

\subsubsection{Static Single Assignment Form}

% ~~~ [ Control Flow Graph ] ~~~~~~~~~~~~~~~~~~~~~~~~~~~~~~~~~~~~~~~~~~~~~~~~~~~

% TODO: Control-Flow Graphs or Control Flow Graphs?

\subsubsection{Control Flow Graph}

% ~~~ [ Abstract Syntax Tree ] ~~~~~~~~~~~~~~~~~~~~~~~~~~~~~~~~~~~~~~~~~~~~~~~~~

\subsubsection{Abstract Syntax Tree}

% --- [ Decompilation Phases ] -------------------------------------------------

\subsection{Decompilation Phases}

% TODO: Make sure the word hallmark can be used in this context.

The following decompilation phases have been outlined by Cristina Cifuentes in
her hallmark paper ``Reverse Compilation Techniques'' \cite{rev_comp}.

% ~~~ [ Syntax Analysis ] ~~~~~~~~~~~~~~~~~~~~~~~~~~~~~~~~~~~~~~~~~~~~~~~~~~~~~~

\subsubsection{Syntax Analysis}

% ~~~ [ Semantic Analysis ] ~~~~~~~~~~~~~~~~~~~~~~~~~~~~~~~~~~~~~~~~~~~~~~~~~~~~

\subsubsection{Semantic Analysis}

% ~~~ [ Intermediate Code Generation ] ~~~~~~~~~~~~~~~~~~~~~~~~~~~~~~~~~~~~~~~~~

\subsubsection{Intermediate Code Generation}

% ~~~ [ Control Flow Graph Generation ] ~~~~~~~~~~~~~~~~~~~~~~~~~~~~~~~~~~~~~~~~

\subsubsection{Control Flow Graph Generation}

% ~~~ [ Data Flow Analysis ] ~~~~~~~~~~~~~~~~~~~~~~~~~~~~~~~~~~~~~~~~~~~~~~~~~~~

\subsubsection{Data Flow Analysis}

% ~~~ [ Control Flow Analysis ] ~~~~~~~~~~~~~~~~~~~~~~~~~~~~~~~~~~~~~~~~~~~~~~~~

\subsubsection{Control Flow Analysis}

% === [ Related Work ] =========================================================

\section{Related Work}

% --- [ Academic Prototypes ] --------------------------------------------------

\subsection{Academic Prototypes}

% ~~~ [ The dcc Decompiler ] ~~~~~~~~~~~~~~~~~~~~~~~~~~~~~~~~~~~~~~~~~~~~~~~~~~~

\subsubsection{The \texttt{dcc} Decompiler}

\cite{rev_comp}

% ~~~ [ The opencl Decompiler ] ~~~~~~~~~~~~~~~~~~~~~~~~~~~~~~~~~~~~~~~~~~~~~~~~

\subsubsection{The \texttt{opencl} Decompiler}

\cite{decomp_llvm}

% --- [ Open Source Projects ] -------------------------------------------------

\subsection{Open Source Projects}

% ~~~ [ Boomerang ] ~~~~~~~~~~~~~~~~~~~~~~~~~~~~~~~~~~~~~~~~~~~~~~~~~~~~~~~~~~~~

\subsubsection{Boomerang}

\cite{boomerang}

% ~~~ [ radare ] ~~~~~~~~~~~~~~~~~~~~~~~~~~~~~~~~~~~~~~~~~~~~~~~~~~~~~~~~~~~~~~~

\subsubsection{radare}

\cite{radare}

% --- [ Commercial Products ] --------------------------------------------------

\subsection{Commercial Products}

% TODO: Add Mac-specific Decompiler.

% ~~~ [ The Interactive Disassembler ] ~~~~~~~~~~~~~~~~~~~~~~~~~~~~~~~~~~~~~~~~~

\subsubsection{The Interactive Disassembler}

% ~~~ [ Hex-Rays Decompiler ] ~~~~~~~~~~~~~~~~~~~~~~~~~~~~~~~~~~~~~~~~~~~~~~~~~~

\subsubsection{Hex-Rays Decompiler}

%%%%%%%%%%%%%%%%%%%%%%%%%%%%%%%%%%%%%%%%%%%%%%%%%%%%%%%%%%%%%%%%%%%%%%%%%%%%%%%%
% Design and Implementation
%%%%%%%%%%%%%%%%%%%%%%%%%%%%%%%%%%%%%%%%%%%%%%%%%%%%%%%%%%%%%%%%%%%%%%%%%%%%%%%%

% === [ Requirements ] =========================================================

\section{Requirements}

% === [ Methodology ] ==========================================================

\section{Methodology}

% --- [ Evolutionary Development and Throwaway Prototyping ] -------------------

\subsection{Evolutionary Development and Throwaway Prototyping}

% ~~~ [ Revision Control System ] ~~~~~~~~~~~~~~~~~~~~~~~~~~~~~~~~~~~~~~~~~~~~~~

\subsubsection{Revision Control System}

% --- [ Public Issue Tracking ] ------------------------------------------------

\subsection{Public Issue Tracking}

% --- [ Continuous Integration ] -----------------------------------------------

\subsection{Continuous Integration}

% --- [ Test Driven Design ] ---------------------------------------------------

\subsection{Test Driven Design}

% === [ Design ] ===============================================================

\section{Design}

% --- [ Choice of Programming Language ] ---------------------------------------

\subsection{Choice of Programming Language}

% TODO: Mention software composition.

% --- [ Choice of Intermediate Representation ] --------------------------------

% TODO: Find synonym for "Choice of".

\subsection{Choice of Intermediate Representation}

% --- [ Decompiler Pipeline ] --------------------------------------------------

\subsection{Decompiler Pipeline}

% --- [ System Architecture ] --------------------------------------------------

\subsection{System Architecture}

% TODO: Mention package division.

% === [ Development ] ==========================================================

\section{Development}

% --- [ TODO ] ------------------------------------------------------------

\subsection{TODO}

% --- [ Testing ] --------------------------------------------------------------

\subsection{Testing}

% ~~~ [ Fuzzing the LLVM IR Parser ] ~~~~~~~~~~~~~~~~~~~~~~~~~~~~~~~~~~~~~~~~~~~

\subsubsection{Fuzzing the LLVM IR Parser}

%%%%%%%%%%%%%%%%%%%%%%%%%%%%%%%%%%%%%%%%%%%%%%%%%%%%%%%%%%%%%%%%%%%%%%%%%%%%%%%%
% Evaluation and Conclusion
%%%%%%%%%%%%%%%%%%%%%%%%%%%%%%%%%%%%%%%%%%%%%%%%%%%%%%%%%%%%%%%%%%%%%%%%%%%%%%%%

% === [ Evaluation ] ===========================================================

\section{Evaluation}

% --- [ Evaluation against Requirements ] --------------------------------------

\subsection{Evaluation against Requirements}

% --- [ Third Party Adaptation ] -----------------------------------------------

\subsection{Third Party Adaptation}

% --- [ Profiling and Benchmarks ] ---------------------------------------------

\subsection{Profiling and Benchmarks}

% === [ Future Work ] ==========================================================

\section{Future Work}

% === [ Conclusion ] ===========================================================

\section{Conclusion}

% --- [ Introduction ] ---------------------------------------------------------

\subsection{Introduction}

% --- [ Personal Reflections ] -------------------------------------------------

\subsection{Personal Reflections}

% --- [ Personal Development ] -------------------------------------------------

\subsection{Personal Development}

% --- [ Final Thoughts ] -------------------------------------------------------

\subsection{Final Thoughts}

\pagebreak

% === [ References ] ===========================================================

\section{References}

\bibliography{references}

\end{document}
