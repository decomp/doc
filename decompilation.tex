% TODO: Use British and not American English:
%    * s/artifact/artefact/
%    * Too many commas. Read about best practices.
% TODO: Proof read for common mistakes:
%    * exist vs exists.

\documentclass[12pt, a4paper]{article}

% Preamble

\usepackage[utf8]{inputenc}
\usepackage[english]{babel}
\usepackage[margin=1in]{geometry}
\usepackage[parfill]{parskip}
\usepackage[hidelinks]{hyperref}
\usepackage{graphicx}
\bibliographystyle{ieeetr}

\title{Language-Agnostic Composition of Decompilation Components}
\author{Robin Eklind}
\date{2014-10-04}

% Document

\begin{document}

\maketitle

\begin{abstract}
TODO
\end{abstract}

\pagebreak

\tableofcontents

\pagebreak

%%%%%%%%%%%%%%%%%%%%%%%%%%%%%%%%%%%%%%%%%%%%%%%%%%%%%%%%%%%%%%%%%%%%%%%%%%%%%%%%
% Introduction
%%%%%%%%%%%%%%%%%%%%%%%%%%%%%%%%%%%%%%%%%%%%%%%%%%%%%%%%%%%%%%%%%%%%%%%%%%%%%%%%

% === [ Introduction ] =========================================================

\section{Introduction}

Lorem ipsum dolor sit amet, consectetur adipiscing elit, sed do eiusmod tempor incididunt ut labore et dolore magna aliqua.

A compiler is a piece of software which translates human readable high-level programming languages (e.g. C) to machine readable low-level languages (e.g. Assembly). In the usual flow of compilation, code is lowered through a set of transformations from a high-level to a low-level representation. This paper introduces the concept of decompilation (originally referred to as reverse compilation \cite{rev_comp}) which moves in the opposite direction by lifting code from a low-level to a high-level representation.

% TODO(u): Write about the purpose of this project, and why it may add something new to the table.

% TODO: Find a good place to write about the applications of decompilation;
%    automated vuln audit \cite{vuln_audit}, etc.

% --- [ Project Aim and Objectives ] -------------------------------------------

\subsection{Project Aim and Objectives}

The aim of this project is to facilitate decompilation workflows using composition of language-agnostic decompilation passes; specifically the reconstruction of high-level control structures and, as a future ambition, expressions.

In order to achieve this aim, the author will:
\begin{enumerate}
	\item Review traditional decompilation techniques, including control flow analysis and data flow analysis.
	\item Critically evaluate a set of Intermediate Representations (IR), which describes low-, medium- and high-level language semantics, to identify one or more suitable for the decompilation pipeline.
	\item Analyze the formal grammar (language specification) of the IR to verify that it is unambiguous. If the grammar is ambiguous or if no formal grammar exists, produce a formal grammar. This objective is critical for language-independence, as the IR works as a bridge between different programming languages.
	\item Determine if any existing library for the IR satisfies our requirements; and if not develop one. The requirements would include a suitable in-memory representation, and support for on-disk file storage and arbitrary manipulations (inject, delete, etc) of the IR.
	\item Design and develop components which identify the structural patterns of high-level control structures using control flow analysis of the IR.
	\item Develop tools which perform one or more decompilation passes on a given IR. The tools will be reusable by other programming language environments as their input and output is specified by a formally defined IR.
	\item As a future ambition, design and develop components which perform expression propagation using data flow analysis of the IR.
\end{enumerate}

% --- [ Deliverables ] ---------------------------------------------------------

\subsection{Deliverables}

The source code and the report of this project have been released into the public domain \cite{cc0} and are made available at:
\begin{itemize}
	\item \url{https://github.com/mewlang/llvm}
	\item \url{https://github.com/mewpaper/decompilation}
\end{itemize}

% TODO: Specify the deliverables of the project after the development phase has finished. Use the text below as a basis.
%
% The following documents have been produced:
% * Project report.
% * Formal grammar for the LLVM IR language; refer to objective 3.
%
% The following system artifacts have been developed:
% * Library for interacting with LLVM IR; refer to objective 4.
% * Library for reconstructing high- level control structures from LLVM IR; refer to objective 5.
% * Tool for performing one or more decompilation passes on a given IR; refer to objective 6.

% --- [ Disposition ] ----------------------------------------------------------

\subsection{Disposition}

Lorem ipsum dolor sit amet, consectetur adipiscing elit, sed do eiusmod tempor incididunt ut labore et dolore magna aliqua.

%%%%%%%%%%%%%%%%%%%%%%%%%%%%%%%%%%%%%%%%%%%%%%%%%%%%%%%%%%%%%%%%%%%%%%%%%%%%%%%%
% Literature Review
%%%%%%%%%%%%%%%%%%%%%%%%%%%%%%%%%%%%%%%%%%%%%%%%%%%%%%%%%%%%%%%%%%%%%%%%%%%%%%%%

% === [ Literature Review ] ====================================================

\section{Literature Review}

Lorem ipsum dolor sit amet, consectetur adipiscing elit, sed do eiusmod tempor incididunt ut labore et dolore magna aliqua.

% --- [ Terminology ] ----------------------------------------------------------

\subsection{Terminology}

Lorem ipsum dolor sit amet, consectetur adipiscing elit, sed do eiusmod tempor incididunt ut labore et dolore magna aliqua.

% ~~~ [ Intermediate Representation ] ~~~~~~~~~~~~~~~~~~~~~~~~~~~~~~~~~~~~~~~~~~

\subsubsection{Intermediate Representation}

Lorem ipsum dolor sit amet, consectetur adipiscing elit, sed do eiusmod tempor incididunt ut labore et dolore magna aliqua.

% ~~~ [ Static Single Assignment Form ] ~~~~~~~~~~~~~~~~~~~~~~~~~~~~~~~~~~~~~~~~

\subsubsection{Static Single Assignment Form}

Lorem ipsum dolor sit amet, consectetur adipiscing elit, sed do eiusmod tempor incididunt ut labore et dolore magna aliqua.

\cite{ssa_decomp}

% ~~~ [ Control Flow Graph ] ~~~~~~~~~~~~~~~~~~~~~~~~~~~~~~~~~~~~~~~~~~~~~~~~~~~

\subsubsection{Control Flow Graph}

Lorem ipsum dolor sit amet, consectetur adipiscing elit, sed do eiusmod tempor incididunt ut labore et dolore magna aliqua.

% --- [ Dissection of an Executable ] ------------------------------------------

% TODO: Change section header to Anatomy of an Executable?

\subsection{Dissection of an Executable}

Lorem ipsum dolor sit amet, consectetur adipiscing elit, sed do eiusmod tempor incididunt ut labore et dolore magna aliqua.

\cite{elf_dissection}

% --- [ Decompilation Phases ] -------------------------------------------------

\subsection{Decompilation Phases}

A core principle utilized in decompilers is the separation of concern through the use of abstractions, and a lot of work involves translating into and breaking out of various abstraction layers. In general a decompiler is composed of distinct phases which parses, analyzes or transforms the input. These phases are conceptually grouped into three modules to separate concerns regarding source machine language and target programming language. The front-end module parses executable files and translates their platform dependent assembly into a platform independent intermediate representation (IR). The middle-end module performs a set of decompilation passes to lift the IR, from a low-level to a high-level representation, by reconstruction high-level control structures and expressions. Finally the back-end module translates the high-level IR to a specific target programming language. Figure \ref{overview} gives an overview of the decompilation modules and visualizes their relationship.

\begin{figure}[htbp]
	\includegraphics[width=\textwidth]{inc/overview.png}
	\caption{The front-end module accepts several executable file formats (PE, ELF, …) as input and translates their platform dependent assembly (x86, ARM, …) to a low-level IR. The middle-end module then lifts the low-level IR to a high-level IR through a set of decompilation passes. Finally the backend-module translates the high-level IR into one of several target programming languages (C, Go, Python, …).}
	\label{overview}
\end{figure}

The remainder of this section describes the distinct decompilation phases, most of which have been outlined by Cristina Cifuentes in her influential paper ``Reverse Compilation Techniques'' \cite{rev_comp}.

% ~~~ [ Executable File Analysis ] ~~~~~~~~~~~~~~~~~~~~~~~~~~~~~~~~~~~~~~~~~~~~~

% TODO: Rename to Binary Analysis?

\subsubsection{Executable File Analysis}

The representation of executables, shared libraries and object code is standardized by a variety of file formats which provides encapsulation of assembly instructions and data. Two such formats are the Portable Executable (PE) file format and the Executable and Linkable Format (ELF), which are used by Windows and Linux respectively. Both of these formats partition executable code and data into sections and assign appropriate access permissions to each section, as summarized by table \ref{sections}. In general no single section has both write and execute permissions as this could compromise the security of the system.

\begin{table}[htbp]
	\begin{center}
		\begin{tabular}{|l|l|l|}
			\hline
			Section name & Usage description & Access permissions \\
			\hline
			\texttt{.text} & Assembly instructions & \texttt{r-x} \\
			\texttt{.rodata} & Read-only data & \texttt{r--} \\
			\texttt{.data} & Data & \texttt{rw-} \\
			\texttt{.bss} & Uninitialized data & \texttt{rw-} \\
			\hline
		\end{tabular}
	\end{center}
	\caption{A summary of the most commonly used sections in ELF files. The \texttt{.text} section contains executable code while the \texttt{.rodata}, \texttt{.data} and \texttt{.bss} sections contains data in various forms.}
	\label{sections}
\end{table}

TODO: continue here.

% ~~~ [ Syntax Analysis ] ~~~~~~~~~~~~~~~~~~~~~~~~~~~~~~~~~~~~~~~~~~~~~~~~~~~~~~

\subsubsection{Syntax Analysis}

Lorem ipsum dolor sit amet, consectetur adipiscing elit, sed do eiusmod tempor incididunt ut labore et dolore magna aliqua.

% ~~~ [ Semantic Analysis ] ~~~~~~~~~~~~~~~~~~~~~~~~~~~~~~~~~~~~~~~~~~~~~~~~~~~~

\subsubsection{Semantic Analysis}

Lorem ipsum dolor sit amet, consectetur adipiscing elit, sed do eiusmod tempor incididunt ut labore et dolore magna aliqua.

% ~~~ [ Intermediate Code Generation ] ~~~~~~~~~~~~~~~~~~~~~~~~~~~~~~~~~~~~~~~~~

\subsubsection{Intermediate Code Generation}

Lorem ipsum dolor sit amet, consectetur adipiscing elit, sed do eiusmod tempor incididunt ut labore et dolore magna aliqua.

% ~~~ [ Control Flow Graph Generation ] ~~~~~~~~~~~~~~~~~~~~~~~~~~~~~~~~~~~~~~~~

\subsubsection{Control Flow Graph Generation}

Lorem ipsum dolor sit amet, consectetur adipiscing elit, sed do eiusmod tempor incididunt ut labore et dolore magna aliqua.

% ~~~ [ Data Flow Analysis ] ~~~~~~~~~~~~~~~~~~~~~~~~~~~~~~~~~~~~~~~~~~~~~~~~~~~

\subsubsection{Data Flow Analysis}

Lorem ipsum dolor sit amet, consectetur adipiscing elit, sed do eiusmod tempor incididunt ut labore et dolore magna aliqua.

\cite{type_decomp}

% ~~~ [ Control Flow Analysis ] ~~~~~~~~~~~~~~~~~~~~~~~~~~~~~~~~~~~~~~~~~~~~~~~~

\subsubsection{Control Flow Analysis}

Lorem ipsum dolor sit amet, consectetur adipiscing elit, sed do eiusmod tempor incididunt ut labore et dolore magna aliqua.

% === [ Related Work ] =========================================================

\section{Related Work}

Lorem ipsum dolor sit amet, consectetur adipiscing elit, sed do eiusmod tempor incididunt ut labore et dolore magna aliqua.

% --- [ Academic Prototypes ] --------------------------------------------------

\subsection{Academic Prototypes}

Lorem ipsum dolor sit amet, consectetur adipiscing elit, sed do eiusmod tempor incididunt ut labore et dolore magna aliqua.

% ~~~ [ The dcc Decompiler ] ~~~~~~~~~~~~~~~~~~~~~~~~~~~~~~~~~~~~~~~~~~~~~~~~~~~

\subsubsection{The \texttt{dcc} Decompiler}

Lorem ipsum dolor sit amet, consectetur adipiscing elit, sed do eiusmod tempor incididunt ut labore et dolore magna aliqua.

\cite{rev_comp}

% ~~~ [ The opencl Decompiler ] ~~~~~~~~~~~~~~~~~~~~~~~~~~~~~~~~~~~~~~~~~~~~~~~~

\subsubsection{The \texttt{opencl} Decompiler}

Lorem ipsum dolor sit amet, consectetur adipiscing elit, sed do eiusmod tempor incididunt ut labore et dolore magna aliqua.

\cite{decomp_llvm}

% ~~~ [ C-Decompiler ] ~~~~~~~~~~~~~~~~~~~~~~~~~~~~~~~~~~~~~~~~~~~~~~~~~~~~~~~~~

\subsubsection{\texttt{C-Decompiler}}

The \texttt{C-Decompiler} translates machine code into C source code. It focuses primarily on improving the readability of the generated C source code, and does so by extending the traditional decompilation techniques outlined by Cristina Cifuentes in three ways. Firstly, the data flow analysis phase is refined using a shadow stack, which corresponds to a virtual stack capable of tracking stack variables and updates to the stack pointer register. Secondly, the register propagation algorithms are adapted to handle use-def chains across multiple basic blocks. Lastly, library signatures are generated for the C++ Standard Template Library \cite{readable_decomp}.

% ~~~ [ The REcompile Decompilation Framework ] ~~~~~~~~~~~~~~~~~~~~~~~~~~~~~~~~

\subsubsection{The \texttt{REcompile} Decompilation Framework}

Lorem ipsum dolor sit amet, consectetur adipiscing elit, sed do eiusmod tempor incididunt ut labore et dolore magna aliqua.

\cite{recompile}

% --- [ Open Source Projects ] -------------------------------------------------

\subsection{Open Source Projects}

Lorem ipsum dolor sit amet, consectetur adipiscing elit, sed do eiusmod tempor incididunt ut labore et dolore magna aliqua.

% ~~~ [ Boomerang ] ~~~~~~~~~~~~~~~~~~~~~~~~~~~~~~~~~~~~~~~~~~~~~~~~~~~~~~~~~~~~

\subsubsection{Boomerang}

Lorem ipsum dolor sit amet, consectetur adipiscing elit, sed do eiusmod tempor incididunt ut labore et dolore magna aliqua.

\cite{boomerang}

% ~~~ [ radare ] ~~~~~~~~~~~~~~~~~~~~~~~~~~~~~~~~~~~~~~~~~~~~~~~~~~~~~~~~~~~~~~~

\subsubsection{radare}

Lorem ipsum dolor sit amet, consectetur adipiscing elit, sed do eiusmod tempor incididunt ut labore et dolore magna aliqua.

\cite{radare}

% --- [ Commercial Products ] --------------------------------------------------

\subsection{Commercial Products}

Lorem ipsum dolor sit amet, consectetur adipiscing elit, sed do eiusmod tempor incididunt ut labore et dolore magna aliqua.

% TODO: Add Mac-specific Decompiler.

% ~~~ [ Hex-Rays Decompiler ] ~~~~~~~~~~~~~~~~~~~~~~~~~~~~~~~~~~~~~~~~~~~~~~~~~~

\subsubsection{Hex-Rays Decompiler}

Lorem ipsum dolor sit amet, consectetur adipiscing elit, sed do eiusmod tempor incididunt ut labore et dolore magna aliqua.

\cite{hexrays}

% --- [ Decompilation as a Services ] ------------------------------------------

\subsection{Decompilation as a Services}

Lorem ipsum dolor sit amet, consectetur adipiscing elit, sed do eiusmod tempor incididunt ut labore et dolore magna aliqua.

% ~~~ [ The Retargetable Decompiler ] ~~~~~~~~~~~~~~~~~~~~~~~~~~~~~~~~~~~~~~~~~~

\subsubsection{The Retargetable Decompiler}

Lorem ipsum dolor sit amet, consectetur adipiscing elit, sed do eiusmod tempor incididunt ut labore et dolore magna aliqua.

\cite{retargetable_decomp}

% ==============================================================================
% Design and Implementation
% ==============================================================================

% === [ Requirements ] =========================================================

\section{Requirements}

Lorem ipsum dolor sit amet, consectetur adipiscing elit, sed do eiusmod tempor incididunt ut labore et dolore magna aliqua.

% === [ Methodology ] ==========================================================

\section{Methodology}

Lorem ipsum dolor sit amet, consectetur adipiscing elit, sed do eiusmod tempor incididunt ut labore et dolore magna aliqua.

% TODO(u): Explain why this specific methodology was chosen, and justify the methods. The justification may be required to be quite detailed, with in-text references.

% --- [ Evolutionary Development and Throwaway Prototyping ] -------------------

\subsection{Evolutionary Development and Throwaway Prototyping}

Lorem ipsum dolor sit amet, consectetur adipiscing elit, sed do eiusmod tempor incididunt ut labore et dolore magna aliqua.

% ~~~ [ Revision Control System ] ~~~~~~~~~~~~~~~~~~~~~~~~~~~~~~~~~~~~~~~~~~~~~~

\subsubsection{Revision Control System}

Lorem ipsum dolor sit amet, consectetur adipiscing elit, sed do eiusmod tempor incididunt ut labore et dolore magna aliqua.

% --- [ Public Issue Tracking ] ------------------------------------------------

\subsection{Public Issue Tracking}

Lorem ipsum dolor sit amet, consectetur adipiscing elit, sed do eiusmod tempor incididunt ut labore et dolore magna aliqua.

% --- [ Continuous Integration ] -----------------------------------------------

\subsection{Continuous Integration}

Lorem ipsum dolor sit amet, consectetur adipiscing elit, sed do eiusmod tempor incididunt ut labore et dolore magna aliqua.

% --- [ Test Driven Design ] ---------------------------------------------------

\subsection{Test Driven Design}

Lorem ipsum dolor sit amet, consectetur adipiscing elit, sed do eiusmod tempor incididunt ut labore et dolore magna aliqua.

% === [ Design ] ===============================================================

\section{Design}

Lorem ipsum dolor sit amet, consectetur adipiscing elit, sed do eiusmod tempor incididunt ut labore et dolore magna aliqua.

% --- [ Choice of Programming Language ] ---------------------------------------

\subsection{Choice of Programming Language}

Lorem ipsum dolor sit amet, consectetur adipiscing elit, sed do eiusmod tempor incididunt ut labore et dolore magna aliqua.

% TODO: Mention software composition.

% --- [ Choice of Intermediate Representation ] --------------------------------

% TODO: Find synonym for "Choice of".

\subsection{Choice of Intermediate Representation}

Lorem ipsum dolor sit amet, consectetur adipiscing elit, sed do eiusmod tempor incididunt ut labore et dolore magna aliqua.

% --- [ Decompiler Pipeline ] --------------------------------------------------

\subsection{Decompiler Pipeline}

Lorem ipsum dolor sit amet, consectetur adipiscing elit, sed do eiusmod tempor incididunt ut labore et dolore magna aliqua.

% --- [ System Architecture ] --------------------------------------------------

\subsection{System Architecture}

Lorem ipsum dolor sit amet, consectetur adipiscing elit, sed do eiusmod tempor incididunt ut labore et dolore magna aliqua.

% TODO: Mention package division.

% === [ Development ] ==========================================================

\section{Development}

Lorem ipsum dolor sit amet, consectetur adipiscing elit, sed do eiusmod tempor incididunt ut labore et dolore magna aliqua.

% --- [ Formal Grammar of LLVM IR ] --------------------------------------------

\subsection{Formal Grammar of LLVM IR}

Lorem ipsum dolor sit amet, consectetur adipiscing elit, sed do eiusmod tempor incididunt ut labore et dolore magna aliqua.

% --- [ Documentation ] --------------------------------------------------------

\subsection{Documentation}

Lorem ipsum dolor sit amet, consectetur adipiscing elit, sed do eiusmod tempor incididunt ut labore et dolore magna aliqua.

% --- [ Ideomatic Language Use ] -----------------------------------------------

\subsection{Ideomatic Language Use}

Lorem ipsum dolor sit amet, consectetur adipiscing elit, sed do eiusmod tempor incididunt ut labore et dolore magna aliqua.

% --- [ TODO ] -----------------------------------------------------------------

\subsection{TODO}

Lorem ipsum dolor sit amet, consectetur adipiscing elit, sed do eiusmod tempor incididunt ut labore et dolore magna aliqua.

% --- [ Testing ] --------------------------------------------------------------

\subsection{Testing}

Lorem ipsum dolor sit amet, consectetur adipiscing elit, sed do eiusmod tempor incididunt ut labore et dolore magna aliqua.

% ~~~ [ Fuzzing the LLVM IR Parser ] ~~~~~~~~~~~~~~~~~~~~~~~~~~~~~~~~~~~~~~~~~~~

\subsubsection{Fuzzing the LLVM IR Parser}

Lorem ipsum dolor sit amet, consectetur adipiscing elit, sed do eiusmod tempor incididunt ut labore et dolore magna aliqua.

%%%%%%%%%%%%%%%%%%%%%%%%%%%%%%%%%%%%%%%%%%%%%%%%%%%%%%%%%%%%%%%%%%%%%%%%%%%%%%%%
% Evaluation and Conclusion
%%%%%%%%%%%%%%%%%%%%%%%%%%%%%%%%%%%%%%%%%%%%%%%%%%%%%%%%%%%%%%%%%%%%%%%%%%%%%%%%

% === [ Evaluation ] ===========================================================

\section{Evaluation}

Lorem ipsum dolor sit amet, consectetur adipiscing elit, sed do eiusmod tempor incididunt ut labore et dolore magna aliqua.

% --- [ Evaluation against Requirements ] --------------------------------------

\subsection{Evaluation against Requirements}

Lorem ipsum dolor sit amet, consectetur adipiscing elit, sed do eiusmod tempor incididunt ut labore et dolore magna aliqua.

% --- [ TODO ] -----------------------------------------------------------------

\subsection{TODO}

Lorem ipsum dolor sit amet, consectetur adipiscing elit, sed do eiusmod tempor incididunt ut labore et dolore magna aliqua.

% --- [ Third Party Adaptation ] -----------------------------------------------

\subsection{Third Party Adaptation}

Lorem ipsum dolor sit amet, consectetur adipiscing elit, sed do eiusmod tempor incididunt ut labore et dolore magna aliqua.

% --- [ Profiling and Benchmarks ] ---------------------------------------------

\subsection{Profiling and Benchmarks}

Lorem ipsum dolor sit amet, consectetur adipiscing elit, sed do eiusmod tempor incididunt ut labore et dolore magna aliqua.

% === [ Future Work ] ==========================================================

\section{Future Work}

Lorem ipsum dolor sit amet, consectetur adipiscing elit, sed do eiusmod tempor incididunt ut labore et dolore magna aliqua.

% === [ Conclusion ] ===========================================================

\section{Conclusion}

Lorem ipsum dolor sit amet, consectetur adipiscing elit, sed do eiusmod tempor incididunt ut labore et dolore magna aliqua.

% --- [ Introduction ] ---------------------------------------------------------

\subsection{Introduction}

Lorem ipsum dolor sit amet, consectetur adipiscing elit, sed do eiusmod tempor incididunt ut labore et dolore magna aliqua.

% --- [ Project Summary ] ------------------------------------------------------

\subsection{Project Summary}

% TODO(u): Summarise the key findings of your report. No new information should be included.

Lorem ipsum dolor sit amet, consectetur adipiscing elit, sed do eiusmod tempor incididunt ut labore et dolore magna aliqua.

% --- [ Personal Development ] -------------------------------------------------

\subsection{Personal Development}

Lorem ipsum dolor sit amet, consectetur adipiscing elit, sed do eiusmod tempor incididunt ut labore et dolore magna aliqua.

% --- [ Final Thoughts ] -------------------------------------------------------

\subsection{Final Thoughts}

Lorem ipsum dolor sit amet, consectetur adipiscing elit, sed do eiusmod tempor incididunt ut labore et dolore magna aliqua.

\pagebreak

% === [ References ] ===========================================================

\section{References}

\bibliography{references}

\end{document}
