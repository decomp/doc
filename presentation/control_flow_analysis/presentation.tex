\documentclass[aspectratio=1610]{beamer}

\usetheme{KTH}

% ====>>>>> general

%% The slides should be for the audience, to give them a visual experience.
%% Transitions; one style for the slides and one style for the transitions between topics. e.g. light background, dark text vs. dark background, light text.
%% Don't write too much on the slides.
%% Skip effects.
%% Masking images to direct attention.
%% Simple charts and graphs. If possible, in the same style as the presentation (e.g. fonts, colors).

\begin{document}

% === [ Start page ] ===========================================================

\startpage

\begin{frame}

	\vspace{0.02\textheight}

	\begin{Large}
		Evaluation of Control Flow Recover Methods
	\end{Large}

	\vspace{0.1\textheight}

	\begin{small}
		\textit{Robin Eklind}
	\end{small}
\end{frame}

% === [ Introduction ] =========================================================

% ====>>>>> What?

% What?

% Frame problem at a high level.
% 1-3 minutes.

% Key message you wish to communicate. From the perspective of the audience, what will they gain? What can they do with the information?

\normalpage

\begin{frame}
	\frametitle{What?}

	\begin{block}{bar}
		\begin{itemize}
			\item baz
			\begin{itemize}
				\item qux
				\item fob
			\end{itemize}
		\end{itemize}
	\end{block}
\end{frame}

% ====>>>>> Why?

% Why?

% Then go into depth; both intellectual and emotional arguments for the severity of the problem.
% 15-20 minutes if presenting for 1 hour.

\begin{frame}
	\frametitle{Why?}

	\begin{block}{bar}
		\begin{itemize}
			\item baz
			\begin{itemize}
				\item qux
				\item fob
			\end{itemize}
		\end{itemize}
	\end{block}
\end{frame}

% ====>>>>> How?

% How?

% Give solution, including benefits and drawbacks.

\begin{frame}
	\frametitle{How?}

	\begin{block}{bar}
		\begin{itemize}
			\item baz
			\begin{itemize}
				\item qux
				\item fob
			\end{itemize}
		\end{itemize}
	\end{block}
\end{frame}

\end{document}
