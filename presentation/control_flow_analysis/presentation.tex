\documentclass[aspectratio=1610]{beamer}

\usetheme{KTH}
\usepackage{preamble}

% ====>>>>> general

%% The slides should be for the audience, to give them a visual experience.
%% Transitions; one style for the slides and one style for the transitions between topics. e.g. light background, dark text vs. dark background, light text.
%% Don't write too much on the slides.
%% Skip effects.
%% Masking images to direct attention.
%% Simple charts and graphs. If possible, in the same style as the presentation (e.g. fonts, colors).

\begin{document}

% === [ Start page ] ===========================================================

\startpage

\begin{frame}

	\vspace{0.02\textheight}

	\begin{Large}
		Evaluation of Methods for Effective Control Flow Recovery
	\end{Large}

	\vspace{0.1\textheight}

	\begin{small}
		\textit{Robin Eklind}
	\end{small}
\end{frame}

% === [ Introduction ] =========================================================

% ====>>>>> What?

% What?

% Frame problem at a high level.
% 1-3 minutes.

% Key message you wish to communicate. From the perspective of the audience, what will they gain? What can they do with the information?

\normalpage

\begin{frame}
	\frametitle{What?}

	\begin{block}{Control Flow Recovery}
		Analysis of control flow graphs (CFGs) to recover high-level control flow primitives (e.g. \texttt{if}-statement and \texttt{for}-loops) from assembly or low-level intermediate representations.
	\end{block}
\end{frame}

\begin{frame}
	\frametitle{Control Flow Recovery}

	\begin{figure}[htbp]
		\centering
		\begin{subfigure}[t]{0.27\textwidth}
			\centering
			\lstinputlisting[caption={C source file.}, language=C, style=c, basicstyle=\tiny\ttfamily, breaklines=false, numbers=none]{inc/overview/overview.c}
		\end{subfigure}
		\quad
		\begin{subfigure}[t]{0.52\textwidth}
			\centering
			\lstinputlisting[caption={LLVM IR assembly.}, language=llvm, style=nasm, basicstyle=\tiny\ttfamily, breaklines=false, numbers=none]{inc/overview/overview_mini.ll}
		\end{subfigure}
		\caption{\textit{Reverse compilation}, going from low-level (right) to high-level (left).}
	\end{figure}

\end{frame}

% ====>>>>> Why?

% Why?

% Then go into depth; both intellectual and emotional arguments for the severity of the problem.
% 15-20 minutes if presenting for 1 hour.

\begin{frame}
	\frametitle{Why?}

	\begin{block}{Applications of Control Flow Recovery}
		\begin{itemize}
			\item Malware analysis
			\item Security assessments
			\begin{itemize}
				\item Automated vulnerability scanning
			\end{itemize}
			\item Control-flow aware compiler passes
			\item Verification of compiler output (\textit{Reflections on Trusting Trust})
			\item Reverse compilation
			\begin{itemize}
				\item Transpilation between programming languages ($n + m$)
				\item Migrate proprietary software from legacy architectures
			\end{itemize}
		\end{itemize}
	\end{block}
\end{frame}

\begin{frame}
	\frametitle{Control-flow Aware Compiler Passes}

	\includegraphics[width=0.8\textwidth]{inc/applications/loop_finder.png}
\end{frame}

\begin{frame}
	\frametitle{Issues with State-of-the-Art Reverse Compilation Tools}

	\begin{figure}[htbp]
		\centering
		\begin{subfigure}[ht]{0.30\textwidth}
			\centering
			\includegraphics[height=0.80\textheight]{inc/applications/ida/ida_40B0A5.png}
		\end{subfigure}
		\begin{subfigure}[ht]{0.40\textwidth}
			\centering
			\lstinputlisting[caption={Corresponding Go source code.}, language=go, style=go, basicstyle=\tiny\ttfamily, breaklines=false, numbers=none]{inc/applications/ida/go_40B0A5.go}
		\end{subfigure}
		% IDA Version 7.1.180227
		\caption{IDA output (left) and corresponding Go source code (right).}
	\end{figure}
\end{frame}

% ====>>>>> How?

% How?

% Give solution, including benefits and drawbacks.

\begin{frame}
	\frametitle{How?}

	\begin{block}{Control Flow Recovery Methods}
		\begin{itemize}
			\item Hammock method
			\item Interval method
			\item Pattern-independent method
		\end{itemize}
	\end{block}

	There are benefits and drawbacks with each method.
\end{frame}

\begin{frame}
	\frametitle{Hammock method}

	Model high-level control flow primitives as subgraphs and use \textit{subgraph isomorphism search} to locate the corresponding subgraphs in CFGs.

	\vspace*{2em}

	\begin{block}{Pros}
		\begin{itemize}
			\item \textit{no} false positives
		\end{itemize}
	\end{block}

	\begin{block}{Cons}
		\begin{itemize}
			\item \textit{many} false negatives
			\item requires single-entry/single-exit invariant for subgraphs
		\end{itemize}
	\end{block}

\end{frame}

\begin{frame}
	\frametitle{Interval method}

	Identify intervals in CFGs to determine follow sets of high-level control flow primitives.

	\vspace*{2em}

	\begin{block}{Pros}
		\begin{itemize}
			\item handles multi-level \texttt{continue}- and \texttt{break}-statements in loops
		\end{itemize}
	\end{block}

	\begin{block}{Pros/Cons}
		\begin{itemize}
			\item \textit{some} false positives
			\item \textit{some} false negatives
		\end{itemize}
	\end{block}

% solved by interval method:
% - multi-level break and continue of loops.

\end{frame}

\begin{frame}
	\frametitle{Pattern-independent method}

	Considers the conditions required to reach a node in the CFG rather than modelling explicit patterns. Relies on semantic-preserving transformations to pre-process CFGs.

	\vspace*{2em}

	\begin{block}{Pros}
		\begin{itemize}
			\item \textit{no} false negatives
			\item \texttt{goto}-free output
		\end{itemize}
	\end{block}

	\begin{block}{Cons}
		\begin{itemize}
			\item \textit{many} false positives
			\item introduces auxiliary condition variables (not present in original source)
		\end{itemize}
	\end{block}

% solved by pattern-independent method:

%    if (a & !b) {
%       // code1
%    }
%    if (b) {
%       // code 2
%    }
%    if (!b) {
%       // code 3
%    }

%    if (a & !b) {
%       // code1
%       goto LABEL_1
%    }
%    if (b) {
%       // code 2
%    }
%    if (!b) {
%    LABEL_1:
%       // code 3
%    }

\end{frame}

\end{document}
