% === [ Related Work ] =========================================================

% TODO: Merge with Literature Review section?

\section{Related Work}
\label{sec:related_work}

foo

% --- [ Native Code to LLVM IR ] -----------------------------------------------

\subsection{Native Code to LLVM IR}

There exists several open source projects for translating native code (e.g. x86, ARM) into LLVM IR. This section reviews three such projects.

foo

% ~~~ [ Dagger ] ~~~~~~~~~~~~~~~~~~~~~~~~~~~~~~~~~~~~~~~~~~~~~~~~~~~~~~~~~~~~~~~

\subsubsection{Dagger}

% TODO: Research; http://dagger.repzret.org/

foo

% ~~~ [ Fracture ] ~~~~~~~~~~~~~~~~~~~~~~~~~~~~~~~~~~~~~~~~~~~~~~~~~~~~~~~~~~~~~

\subsubsection{Fracture}

% TODO: Research; https://github.com/draperlaboratory/fracture

foo

% ~~~ [ MC-Semantics ] ~~~~~~~~~~~~~~~~~~~~~~~~~~~~~~~~~~~~~~~~~~~~~~~~~~~~~~~~~

\subsubsection{MC-Semantics}

% TODO: Evaluate and highlight key differences between Dagger, Fracture and McSema.
% Dagger and Fracture rely on TableGen for instruction semantics, McSema does not.

% TODO: Research; https://github.com/trailofbits/mcsema

\cite{mcsema}

foo

% --- [ Decompilers ] ----------------------------------------------------------

\subsection{Decompilers}

foo

% ~~~ [ C-Decompiler ] ~~~~~~~~~~~~~~~~~~~~~~~~~~~~~~~~~~~~~~~~~~~~~~~~~~~~~~~~~

\subsubsection{C-Decompiler}

The \texttt{C-Decompiler} translates machine code into C source code. It focuses primarily on improving the readability of the generated C source code, and does so by extending the traditional decompilation techniques outlined by Cristina Cifuentes in three ways. Firstly, the data flow analysis phase is refined using a shadow stack, which corresponds to a virtual stack capable of tracking stack variables and updates to the stack pointer register. Secondly, the register propagation algorithms are adapted to handle use-def chains across multiple basic blocks. Lastly, library signatures are generated for the C++ Standard Template Library \cite{readable_c_decomp}.

% ~~~ [ Hex-Rays Decompiler ] ~~~~~~~~~~~~~~~~~~~~~~~~~~~~~~~~~~~~~~~~~~~~~~~~~~

\subsubsection{Hex-Rays Decompiler}

foo

\cite{hexrays}

% ~~~ [ The Retargetable Decompiler ] ~~~~~~~~~~~~~~~~~~~~~~~~~~~~~~~~~~~~~~~~~~

\subsubsection{The Retargetable Decompiler}
\label{sec:retargetable_decompiler}

% TODO: Write about Decompilation as a Services.

foo

\cite{retargetable_decomp}
