% --- [ Evaluation against Requirements ] --------------------------------------

\subsection{Evaluation against Requirements}

foo

\subsubsection{LLVM IR Library}

In total four essential (\textbf{R1}, \textbf{R2}, \textbf{R3} and \textbf{R4}), one desirable (\textbf{R5}), and three future (\textbf{R6}, \textbf{R7} and \textbf{R8}) requirements were identified for the LLVM IR library (see section \ref{sec:req_llvm_ir_library}).

foo

\subsubsection{Control Flow Analysis Library}
\label{sec:eval_control_flow_analysis_library}

In total four essential (\textbf{R9}, \textbf{R10}, \textbf{R11} and \textbf{R12}), two important (\textbf{R13} and \textbf{R14}), two desirable (\textbf{R15} and \textbf{R16}), and two future (\textbf{R17} and \textbf{R18}) requirements were identified for the control flow analysis library (see section \ref{sec:req_control_flow_analysis_library}).

The control flow analysis library


The current implementation of the control flow analysis library enforces an invariant on the graph representation of high-level control flow primitives; they must have a single entry and a single exit node. This invariant simplifies the implementation

requires an a single entry and a single exit node

 single-entry, single-exit an invariant of the

\begin{itemize}
	\item \textbf{R9}: Support analysis of reducible graphs
	\item \textbf{R10}: Recover pre-test loops (e.g. \texttt{for}, \texttt{while})
	\item \textbf{R11}: Recover infinite loops (e.g. \texttt{while(TRUE)})
	\item \textbf{R12}: Recover 2-way conditionals (e.g. \texttt{if}, \texttt{if-else})
\end{itemize}

\subsubsection{Control Flow Analysis Tool}

In total two essential (\textbf{R19} and \textbf{R20}) requirements were identified for the control flow analysis tool (see section \ref{sec:req_control_flow_analysis_tool}).

foo
