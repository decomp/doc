% ~~~ [ Important Requirements ] ~~~~~~~~~~~~~~~~~~~~~~~~~~~~~~~~~~~~~~~~~~~~~~~

\subsubsection{Important Requirements}
\label{sec:eval_control_flow_analysis_library_important_requirements}

% * R14 - Recover post-test loops

The successful decompilation of post-test loops (\textbf{R14}) is demonstrated in appendix \ref{app:decompilation_of_post-test_loops}, which provides a contrived example that implicitly uses the same decompilation steps as described above. The final Go output, which is presented on the right side of figure \ref{fig:post-test_comparison} in appendix \ref{app:decompilation_of_post-test_loops}, contains an infinite \texttt{for}-loop with a conditional break statement as the last statement of the loop body (which is semantically equivalent to a post-test loop), thus proving that post-test loops may be recovered. Even though Go does not provide native support for post-test loops, the back-end was capable of translating the source program into semantically equivalent Go code, by combining a set of primitives available in Go. The same approach may be used to support missing primitives for other target programming languages (eg. \texttt{switch}-statements in Python).

% * R15 - Recover n-way conditionals

A data-driven design separates the implementation of the control flow analysis component from the definition of supported high-level control flow primitives, which are expressed in the DOT file format. The design is motivated by the principle of separation of concern (e.g. the control flow analysis may be reused to analyse the control flow of REIL) and extensibility (e.g. support for new high-level control flow primitives may be added without changing the source code), as further described in section \ref{sec:design_control_flow_analysis}. One limitation with this design however, is that it does not support n-way conditionals (\textbf{R15}) or any other high-level control flow primitives with a variable number of nodes in their graph representations, as these cannot be expressed in the DOT file format. A discussion on how to mitigate this issue in the future is provided in section \ref{sec:con_design_validation}.
