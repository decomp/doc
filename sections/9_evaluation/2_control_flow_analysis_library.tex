% --- [ Control Flow Analysis Library ] ----------------------------------------

\subsection{Control Flow Analysis Library}
\label{sec:eval_control_flow_analysis_library}

In total four essential (\textbf{R9}, \textbf{R10}, \textbf{R11} and \textbf{R12}), two important (\textbf{R13} and \textbf{R14}), two desirable (\textbf{R15} and \textbf{R16}), and two future (\textbf{R17} and \textbf{R18}) requirements were identified for the control flow analysis library (see section \ref{sec:req_control_flow_analysis_library}).

%The control flow analysis library

%The current implementation of the control flow analysis library enforces an invariant on the graph representation of high-level control flow primitives; they must have a single entry and a single exit node. This invariant simplifies the implementation

%requires an a single entry and a single exit node

% single-entry, single-exit an invariant of the

%\begin{itemize}
%	\item \textbf{R9}: Support analysis of reducible graphs
%	\item \textbf{R10}: Recover pre-test loops (e.g. \texttt{for}, \texttt{while})
%	\item \textbf{R11}: Recover infinite loops (e.g. \texttt{while(TRUE)})
%	\item \textbf{R12}: Recover 2-way conditionals (e.g. \texttt{if}, \texttt{if-else})
%\end{itemize}

foo
