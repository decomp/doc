\begin{abstract}

% --- [ Background, Purpose and Methodology ] ----------------------------------

% 66 + 95 words.

Decompilation or reverse compilation is the process of translating low-level machine-readable code into high-level human-readable code. The problem is non-trivial due to the amount of information lost during compilation, but it can be divided into several smaller problems which may be solved independently. This report explores the feasibility of composing a decompilation pipeline from independent components, and the potential of exposing those components to the end-user. The components of the decompilation pipeline are conceptually grouped into three modules. Firstly, the front-end translates a source language (e.g. x86 assembly) into LLVM IR; a platform-independent low-level intermediate representation. Secondly, the middle-end structures the LLVM IR by identifying high-level control flow primitives (e.g. pre-test loops, 2-way conditionals). Lastly, the back-end translates the structured LLVM IR into a high-level target programming language (e.g. Go). The control flow analysis stage of the middle-end uses subgraph isomorphism search algorithms to locate control flow primitives in CFGs, both of which are described using Graphviz DOT files.

% --- [ Results and Conclusion ] -----------------------------------------------

% 46 + 82 words.

The decompilation pipeline has been proven capable of recovering nested pre-test and post-test loops (e.g. \texttt{while}, \texttt{do-while}), and 1-way and 2-way conditionals (e.g. \texttt{if}, \texttt{if-else}) from LLVM IR. Furthermore, the data-driven design of the control flow analysis stage facilitates extensions to identify new control flow primitives. There is huge potential for future development. The Go output could be made more idiomatic by adding further post-processing stages; e.g. the \texttt{grind} tool by Russ Cox moves variable declarations closer to their usage. The language-agnostic aspects of the design will be validated by implementing components in other languages; e.g. data flow analysis in Haskell. Additional back-ends (e.g. Python output) will be implemented to verify that the general decompilation tasks (e.g. control flow analysis, data flow analysis) are handled by the middle-end.

\end{abstract}
