% ~~~ [ Post-processsing ] ~~~~~~~~~~~~~~~~~~~~~~~~~~~~~~~~~~~~~~~~~~~~~~~~~~~~~

\subsubsection{Post-processing}
\label{sec:design_post-processing}

The post-processing components seeks to improve the quality of the unpolished Go source code, which was produced by the earlier stages of the decompilation pipeline. One such component is the \texttt{grind} tool by Russ Cox (see section \ref{sec:con_extended_capabilities}), which improves the quality of Go code by moving variable declarations closer to their usage. Another such component is the \texttt{go-post} tool, which improve the quality of Go source code by declaring unresolved identifiers, applying Go conventions for exit status codes, propagating temporary variables into expressions, simplifying binary operations, removing dead assignment statements, and promoting the initialization statement and post-statement of for-loops to the loop header.

%    - Unpolished Go -> Go ([go-post](http://decomp.org/x/cmd/go-post))

foo

The polishing is done by separate tools which fixes potential compilation issues and makes the code more idiomatic.

Currently the \texttt{go-post} replaces return-statements in the \texttt{main} function with calls to \texttt{os.Exit}, which is required since the \texttt{main} function has no return arguments in Go. Instead the Go runtime calls \texttt{os.Exit} with the status-code \texttt{0} once \texttt{main} returns to signal successful termination. This eliminates the need to always end the \texttt{main} function with a \texttt{return 0;} statement as is common practise in C. A future ambition is to make use of and possibly contribute to the \texttt{grind} tool which moves variable declarations closer to their usage, and thus improving readability of the code.
