% === [ Conclusion ] ===========================================================

% <howto>
% * Reflections on what you have personally learned from this and lessons for the future.
% * How well you implemented methodologies.
% * How successful the development was.
% * Reflective writing on what you've done.

% <howto> Relationship between sections.
%
%    Introduction ---> Conclusions
%                         ^
%                         |
%    Literature Review ---+

\section{Conclusion}
\label{sec:conclusion}

% TODO: Recommendations: If I were to redo the project today, I wish someone would have told me that xxx.

This section concludes the project report and includes subjective reflections from the author. For the remainder of this section I will switch to a first person narrative.

% --- [ Project Summary ] ------------------------------------------------------

\subsection{Project Summary}

% TODO: Summarise the key findings of your report. No new information should be included.

% TODO: <note>
% * Reflect on how problems were solved in the conclusions chapter.


foo

% --- [ Future Work ] ----------------------------------------------------------

\subsection{Future Work}

% TODO: Add notes from IDEAS.txt.
%    - grind

% * Finish developing the LLVM IR library and replace the C++ binding with a pure Go implementation. This should decrese build times significantly and simplify the overall complexity of the project by several orders of magnitude.

% * Formal Grammar for a subset of LLVM IR
%    - Mention previous (partial but incomplete) work.

% * Third Party Adaptation

% * Data Flow Analysis.

% * Stress test the design by implementing another back-end written in and for another programming language.

% TODO: Fuzz the LLVM IR parser by generating valid LLVM from the EBNF grammar (similar to gosmith).

foo

% --- [ Personal Development ] -------------------------------------------------

\subsection{Personal Development}

This is the largest project I have undertaken in my life and I feel satisfied with the outcome and proud of what I have been able to accomplished. It has re-enforced my belief that any problem is solvable when broken into smaller subproblems and instilled me with a feeling that anything is possible. The project has allowed me to mature as a software engineer and I now feel more confident in utilizing best practies such as TDD, CI and semantic versioning. I have also matured as a developer and gained experience with implementing a semi-large project and structuring it into several smaller self-contained projects.

% --- [ Final Thoughts ] -------------------------------------------------------

\subsection{Final Thoughts}

My happiest moment during the project was when the larger components started working and could be connected to form a complete system. It feels great having started out with a vague idea of how the decompiler could work, gradually gaining new insights and refining its design after researching and building on the knowledge of others, developing and iteratively reimplementing the various components until they feel just right, finally arriving at a working prototype and seeing the full system in action! If there is one key idea I want to leave you with it is that the composition of independent components, each with a single purpose and well-defined input and output, is a powerful concept for solving complex problems.

% TODO: Include the capabilities of the user when he or she is given direct access to the decompilation pipeline and may exchange any component with one of his or her choice. In contrast to traditional decompilers which are monolithic in nature.
