% ~~~ [ Build Status ] ~~~~~~~~~~~~~~~~~~~~~~~~~~~~~~~~~~~~~~~~~~~~~~~~~~~~~~~~~

\subsubsection{Build Status}

The Go build system relies on a few well-established conventions to remain configuration-free (e.g. no \texttt{Makefile}s, or \texttt{configure} scripts). These conventions specify how to derive the import path from the URL of the source code (e.g. \texttt{github.com/user/repo/pkg}), where the source code of a given import path is stored locally (e.g. \texttt{\$\{GOPATH\}/src/github.com/user/repo/pkg}), and that each directory of the source tree corresponds to a single package. In combination with explicit import declarations in the source code, these conventions enable Go packages and tools to be built using only the information present in the source files~\cite{go_command}.

The CI service, which monitors the build status of each component and their dependencies, may leverage the configuration-free build system of Go to simplify its dependency management. Travis CI has been configured to invoke \texttt{go get}\footnote{Download and install packages and dependencies: \url{https://golang.org/cmd/go/}}, which recursively downloads and installs the dependencies of each component; the location of which is derived from the source code.
