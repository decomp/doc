% --- [ Performance ] ----------------------------------------------------------

\subsection{Performance}

The performance characteristics of the various components have been considered during every stage of the development process, but the initial prototypes have prioritized correctness and simplicity over performance. These prototypes aim to identify suitable data structures and algorithms for the problem, through iterative redesigns and reimplementations. Once the major design decisions stabilize, a production quality prototype is being developed and thoroughly  tested. To limit the risk of premature optimizations, micro-level performance work is intentionally postponed to the later stages of development.

refined

 to after tto limit the risk of premature optimizations.

The focus is on macro level design rather than micro level optimization

As the components mature, through iterative reimplementations and redesigns, the choice of  and

Once a working prototype has been implemented, the focus shifts to

The iterative reimplementations and redesigns of the prototypes aim to identify suitable data structures and algorithms. The prototyping is supplemented with algorithm research

% TODO: Find synoynm for "becomes clear", crystalizes

The prototypes mature, through



As the prototypes mature, through iterative redesigns and reimplementations, the choice of core data structures and algorithms crystalizes.

choice of core data structures and algorithms starts to matter.

The primary focus is to get the data structures and core algorithms

As the prototypes mature, after iterative redesigns and reimplementations,

Once a prototype has been implemented, thoroughly tested, and iteratively redesigned and reimplemented



always focused on correctness of implementation

, but performance has never been the focus of the intial implementations.

The main priority when implementing components have been

 Premature optimization









Components with straight forward implementations (e.g. the LLVM IR lexer) have been profiled to locate bottle necks

The performance of the control flow analysis component

The choice of algorithm has determined the performance ; for which algorithm research and the application of time complexity theory

dominated the performance  been fundamental The performance of algoritmically intensive components, such as the control flow analysis,



Research of algorithms Components which

Code such as the LLVM IR lexer




























% TODO: Rewrite, cleanup and verify (especially the \theta(...) claims!).

\textbf{NOTE}: \textit{The following paragraph is more of a brain-dump. It will be used as a basis for a future rewrite.}

Profiling was put to good use when optimizing the lexer, the code base of which is rather straight forward. When estimating the runtime complexity of the subgraph isomorphism search algorithm however the use of intuition and algorithm research was far more valuable. For this specific task the generic problem (subgraph isomorphism search of arbitrary input graphs) could be simplified (TODO: use generalized instead of simplified?) to a much easier problem as every node of the graph were known to be connected (TODO: find the succinct term in graph theory to express this concept). Exploiting this property lead to an algorithm that had a runtime complexity of $ \Theta(n*m) $ where $ m $ is known to be small rather than $ \Theta(n^m) $ as is the case for a brute force algorithm and $ \Theta(n^log(m)) $ (TODO: Find the correct runtime complexity of the Hillman iso search algo) as was the case of the Hillman subgraph isomorphism search algorithm which is capable of solving the generic case, improving on the brute force approach by applying heuristics (TODO: is heuristics the right word to use here?) to prune the search space.

In summary, using profiling is great for simple problems. Using algorithm research, runtime complexity theory and intuition is needed for complex problems. Knowledge about specific properties of the problem which may be exploited.

foo

% --- [ Subsubsections ] -------------------------------------------------------

% ~~~ [ Profiling ] ~~~~~~~~~~~~~~~~~~~~~~~~~~~~~~~~~~~~~~~~~~~~~~~~~~~~~~~~~~~~

\subsubsection{Profiling}
\label{sec:ver_profiling}

% TODO:
%    - CPU profiling
%       go test -cpuprofile=a.prof
%    - Memory profiling
%       go test -memprofile=a.mprof

The initial implementation of the LLVM IR library focused on correctness, and strived to be as simple and straight forward as possible. Once feature complete and thoroughly tested, the library was profiled for the first time and a major performance bottleneck was discovered; as illustrated in figure \ref{fig:lexer_pprof}.

The original version of the LLVM IR lexer used hash maps instead of arrays to look up the types of keywords. The implementation was intuitive and straight worwa

foo

\texttt{go tool pprof}

\begin{figure}[htbp]
	\begin{center}
		\includegraphics[width=\textwidth]{inc/8_ver/lexer_pprof.png}
		\caption{A major performance bottleneck was located when profiling the LLVM lexer library for the first time. Roughly 70\% of the total execution time was spent doing hash map iterations (i.e. \texttt{runtime.mapiternext}).}
		\label{fig:lexer_pprof}
	\end{center}
\end{figure}

% ~~~ [ Benchmarks ] ~~~~~~~~~~~~~~~~~~~~~~~~~~~~~~~~~~~~~~~~~~~~~~~~~~~~~~~~~~~

\subsubsection{Benchmarks}

percentage delta.

foo

\begin{figure}[htbp]
	\begin{center}
		\begin{verbatim}
$ git checkout old; go test -bench=. > old.txt
$ git checkout new; go test -bench=. > new.txt
$ benchcmp old.txt new.txt
benchmark                old ns/op     new ns/op     delta
BenchmarkParseString     737625        204010        -72.34%
		\end{verbatim}
		\caption{Benchmark run time delta between the original and the optimized version of the LLVM IR lexer, as visualized by \texttt{benchcmp}\protect\footnotemark.}
		\label{fig:benchmark_delta}
	\end{center}
\end{figure}
\footnotetext{Display performance changes between benchmarks: \url{https://golang.org/x/tools/cmd/benchcmp}}

