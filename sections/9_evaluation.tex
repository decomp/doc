% === [ Evaluation ] ===========================================================

% <mark>
% - Description of plan for evaluating outcome.
% - Convincing evidence that artefact meets requirements with explanation where
%   it doesn't.
% - Justification of evaluation method.
% - Shows awareness of limits of evaluation.
%
% - How well does the report describe and justify the means by which the outcome
%   of the project was evaluated?
% - How well is it shown whether the specification of the requirements has been
%   satisfied?
% - How well explained are areas where it hasn't?

% <howto> Relationship between sections.
%
%    Requirements -----> Evaluation
%
% <howto>
% - You should describe how you demonstrated that system works as intended (or not, as the case may be)
% - Include comprehensible summaries of the results of all critical tests that you made.
% - You should try to indicate how confident you are about whatever code you have produced, and also suggest what tests would be required to gain further confidence.
% - You must also critically evaluate your system in the light of these tests, describing its strengths and weaknesses.

\section{Evaluation}
\label{sec:evaluation}

This section evaluates the artefacts of the decompilation system against the requirements outlined in section \ref{sec:requirements}. To assess the capabilities of the individual components, relevant decompilation scenarios have been considered. The current state of each component is summarized in the succeeding paragraphs, and future work to validate the design, improve the reliability, and extend the capabilities of the decompilation pipeline is presented in section \ref{sec:con_future_work}.

The \texttt{ll2dot} component (see section \ref{sec:design_control_flow_graph_generation}) is considered stable, but there are known issues which may affect the reliability and the integrity of the produced CFGs; as further described in section \ref{sec:ver_security_assessment}. Future work which seeks to address these issues is presented in section \ref{sec:con_reliability_improvements}.

The subgraph isomorphism search library (see section \ref{sec:impl_subgraph_isomorphism_search_library}) is considered production quality, and the test cases of the \texttt{iso}\footnote{Subgraph isomorphism search library: \url{https://decomp.org/x/graphs/iso}} package have a code coverage of 94.8\%; as outlined in section \ref{sec:ver_code_coverage}. The restrictions imposed by this library on the subgraph (e.g. single-entry/single-exit invariant and fixed number of nodes) limits infinite loops and n-way conditionals from being modelled, as further discussed in section \ref{sec:design_control_flow_analysis}. Section \ref{sec:con_design_validation} presents a discussion of potential approaches which may relax these restrictions in the future.

The \texttt{restructure} component (see section \ref{sec:design_control_flow_analysis}) is considered production quality, and the test cases of the \texttt{restructure} command have a code coverage of 40.0\%; as outlined in section \ref{sec:ver_code_coverage}. The \texttt{restructure} command is believed to be capable of structuring the CFG of any source program which may be constructred from the set of supported high-level control flow primitives (which are described in figure \ref{fig:graph_representations} of section \ref{sec:lit_review_control_flow_analysis}), including source programs with arbitrarily nested primitives. Any future work which improves the reliability and the capabilities of the subgraph isomorphism search library will directly impact the \texttt{restructure} tool, as it relies entirely on subgraph isomorphism search to recover high-level control flow primitives.

The \texttt{ll2go} component (see section \ref{sec:design_back-end_components}) is considered a \textit{proof of concept} implementation. It was implemented primarily to stress test the design of the decompilation pipeline, and only supports a small subset (e.g. all arithmetic instructions and some terminator instructions) of the LLVM IR language. The \texttt{ll2go} tool is affected by the same reliability issues as the \texttt{ll2dot} command, which are caused by the Go bindings for LLVM; as further described in section \ref{sec:impl_go_bindings_for_llvm}. To address these issues a pure Go library is being developed for interacting with LLVM IR, as further described in section \ref{sec:con_reliability_improvements}. A future version of the \texttt{ll2go} tool would discard the current implementation and start fresh, learning from the mistakes and the building upon the insights.

Lastly, the \texttt{go-post} component (see section \ref{sec:design_post-processing}) is considered alpha quality, and the test cases of the \texttt{go-post} command have a code coverage of 38.0\%; as outlined in section \ref{sec:ver_code_coverage}. The \texttt{go-post} tool was primarily implemented to evaluate the feasibility of applying source code transformations to make the decompiled Go code more idiomatic. Implementing these post-processing rules were surprisingly easy, and it was often possible to go from the conceptual idea of a rewrite rule to a working implementation in a matter of hours. While some rewrite rules are reliable (e.g. the \textit{``mainret''} rewrite rule, which is presented in \ref{fig:rewrite_2} of appendix \ref{app:post-processing_example}), most are considered experimental. For instance, the \textit{``localid''} rewrite rule (see figure \ref{fig:rewrite_3} of appendix \ref{app:post-processing_example}) is known to produce incorrect rewrites when applied to complex programs, but it works for simple programs and provides rudimentary support for expression propagation. A proper implementation of expression propagation would rely on the future implementation of the data flow analysis component, which is mentioned in \ref{sec:con_design_validation}.

% === [ Subsections ] ==========================================================

% --- [ LLVM IR Library ] ------------------------------------------------------

\subsection{LLVM IR Library}

In total four essential (\textbf{R1}, \textbf{R2}, \textbf{R3} and \textbf{R4}), one desirable (\textbf{R5}), and three future (\textbf{R6}, \textbf{R7} and \textbf{R8}) requirements were identified for the LLVM IR library (see section~\ref{sec:req_llvm_ir_library}). The modified Go bindings for LLVM (see section~\ref{sec:impl_go_bindings_for_llvm}), the control flow graph generation component (see section~\ref{sec:design_control_flow_graph_generation}) and the \texttt{dot}\footnote{Drawing Graphs with dot: \url{http://www.graphviz.org/pdf/dotguide.pdf}} tool of the Graphviz project, collectively satisfies all five requirements (not counting future requirements); as summarised in table~\ref{tbl:eval_summary_of_llvm_ir_library}. Section~\ref{sec:eval_llvm_ir_library_essential_requirements} and~\ref{sec:eval_llvm_ir_library_desirable_requirements} provides a detailed evaluation of the essential and the desirable requirements, respectively.

\begin{table}[htbp]
	\begin{center}
		\begin{tabular}{|l|l|l|l|}
			\hline
			Sat. & Req. & Priority & Description \\
			\hline
			\rowcolor{light_green_3}
			Yes & \textbf{R1} & MUST & Read the assembly language representation of LLVM IR \\
			\rowcolor{light_green_3}
			Yes & \textbf{R2} & MUST & Write the assembly language representation of LLVM IR \\
			\rowcolor{light_green_3}
			Yes & \textbf{R3} & MUST & Interact with an in-memory representation of LLVM IR \\
			\rowcolor{light_green_3}
			Yes & \textbf{R4} & MUST & Generate CFGs from LLVM IR basic blocks \\
			\hline
			\rowcolor{light_green_3}
			Yes & \textbf{R5} & COULD & Visualise CFGs using the \texttt{DOT} graph description language \\
			\hline
			N/A & \textbf{R6} & WON'T & Read the bitcode representation of LLVM IR \\
			N/A & \textbf{R7} & WON'T & Write the bitcode representation of LLVM IR \\
			N/A & \textbf{R8} & WON'T & Provide a formal grammar of LLVM IR \\
			\hline
		\end{tabular}
	\end{center}
	\caption{A summary of the evaluation against requirements of the LLVM IR library, which specifies what requirements (abbreviated as ``Req.'') that have been satisfied (abbreviated as ``Sat.'').}
	\label{tbl:eval_summary_of_llvm_ir_library}
\end{table}

% --- [ Subsections ] ----------------------------------------------------------

% ~~~ [ Essential Requirements ] ~~~~~~~~~~~~~~~~~~~~~~~~~~~~~~~~~~~~~~~~~~~~~~~

\subsubsection{Essential Requirements}
\label{sec:eval_llvm_ir_library_essential_requirements}

% * R1 - Read the assembly language representation of LLVM IR
% * R3 - Interact with an in-memory representation of LLVM IR
% * R4 - Generate CFGs from LLVM IR basic blocks

The modified Go bindings for LLVM (see section \ref{sec:impl_go_bindings_for_llvm}) includes read (\textbf{R1}) and write (\textbf{R2}) support for the assembly language representation of LLVM IR, and enables interaction with an in-memory representation of LLVM IR (\textbf{R3}). The \texttt{ll2dot} tool depends on \textbf{R1} and \textbf{R3} for parsing LLVM IR assembly files and inspecting their in-memory representation, which is required to gain access to information about the basic blocks of each function and their terminating instructions. This information determines the node names and the directed edges, when generating CFGs from LLVM IR; as further described in section \ref{sec:design_control_flow_graph_generation}. Appendix \ref{app:control_flow_graph_generation_example} demonstrates that the \texttt{ll2dot} tool is capable of generating CFGs from LLVM IR (\textbf{R4}), thus proving that \textbf{R1}, \textbf{R3} and \textbf{R4} have been satisfied.

% * R2 - Write the assembly language representation of LLVM IR

To support generating CFGs for LLVM IR assembly which contains unnamed basic blocks, the \texttt{ll2dot} tool requires access to the names of unnamed basic blocks. These names are not available from the API of the original Go bindings for LLVM however, as they are generated on the fly by the assembly printer. To work around this issue, the assembly printer of LLVM 3.6 was patched to always print the generated names of unnamed basic blocks (see appendix \ref{app:unnamed_patch}). With this patch in place, the debug facilitites of the modified Go bindings for LLVM could be utilized to write (\textbf{R2}) the assembly to temporary files, which were parsed to gain access to the names of unnamed basic blocks; as further described in section \ref{sec:impl_go_bindings_for_llvm}. The generated CFG presented in appendix \ref{app:control_flow_graph_generation_example} contains the names of unnamed basic blocks (e.g. basic blocks with numeric names), thus proving that \textbf{R2} has been satisfied.

% ~~~ [ Desirable Requirements ] ~~~~~~~~~~~~~~~~~~~~~~~~~~~~~~~~~~~~~~~~~~~~~~~

\subsubsection{Desirable Requirements}
\label{sec:eval_llvm_ir_library_desirable_requirements}

% * R5 - Visualize CFGs using the DOT graph description language.

% TODO: Add footnote with link to the language spec for DOT.

The CFGs generated by the \texttt{ll2dot} tool (see section \ref{sec:design_control_flow_graph_generation}) are described in the DOT graph description language\footnote{foo}. One benefit of expressing CFGs in this format, is that it enables the reuse of existing software to visualize the CFGs (\textbf{R5}). Appendix \ref{app:control_flow_graph_generation_example} demonstrates how the \texttt{dot} tool of the Graphviz project may be utilized to produce an image representation of CFGs, which are expressed in the DOT file format.


% --- [ Control Flow Analysis Library ] ----------------------------------------

\subsection{Control Flow Analysis Library}
\label{sec:eval_control_flow_analysis_library}

In total four essential (\textbf{R9}, \textbf{R10}, \textbf{R11} and \textbf{R12}), two important (\textbf{R13} and \textbf{R14}), two desirable (\textbf{R15} and \textbf{R16}), and two future (\textbf{R17} and \textbf{R18}) requirements were identified for the control flow analysis library (see section \ref{sec:req_control_flow_analysis_library}).

The control flow analysis library


The current implementation of the control flow analysis library enforces an invariant on the graph representation of high-level control flow primitives; they must have a single entry and a single exit node. This invariant simplifies the implementation

requires an a single entry and a single exit node

 single-entry, single-exit an invariant of the

\begin{itemize}
	\item \textbf{R9}: Support analysis of reducible graphs
	\item \textbf{R10}: Recover pre-test loops (e.g. \texttt{for}, \texttt{while})
	\item \textbf{R11}: Recover infinite loops (e.g. \texttt{while(TRUE)})
	\item \textbf{R12}: Recover 2-way conditionals (e.g. \texttt{if}, \texttt{if-else})
\end{itemize}

% --- [ Control Flow Analysis Tool ] -------------------------------------------

\subsection{Control Flow Analysis Tool}

In total two essential (\textbf{R20} and \textbf{R21}) requirements were identified for the control flow analysis tool (see section \ref{sec:req_control_flow_analysis_tool}).

The control flow analysis component (see section \ref{sec:design_control_flow_analysis}) satisfies both requirements (not counting future requirements); as summarized in table \ref{tbl:eval_summary_of_control_flow_analysis_tool}. Section \ref{sec:eval_control_flow_analysis_tool_essential_requirements} provides a detailed evaluation of the essential requirements.

\begin{table}[htbp]
	\begin{center}
		\begin{tabular}{|l|l|l|l|}
			\hline
			Sat. & Req. & Priority & Description \\
			\hline
			\rowcolor{light_green_3}
			Yes & \textbf{R20} & MUST & Identify high-level control flow primitives in LLVM IR \\
			\rowcolor{light_green_3}
			Yes & \textbf{R21} & MUST & Support language-agnostic interaction with other components \\
			\hline
		\end{tabular}
	\end{center}
	\caption{A summary of the evaluation against requirements of the control flow analysis tool, which specifies what requirements (abbreviated as ``Req.'') that have been satisfied (abbreviated as ``Sat.'').}
	\label{tbl:eval_summary_of_control_flow_analysis_tool}
\end{table}

% --- [ Subsections ] ----------------------------------------------------------

% ~~~ [ Essential Requirements ] ~~~~~~~~~~~~~~~~~~~~~~~~~~~~~~~~~~~~~~~~~~~~~~~

\subsubsection{Essential Requirements}
\label{sec:eval_control_flow_analysis_tool_essential_requirements}

% * R20 - Identify high-level control flow primitives in LLVM IR

In collaboration, the \texttt{ll2dot} and \texttt{restructure} tools are capable of identifying high-level control flow primitives in LLVM IR (\textbf{R20}). Firstly, the \texttt{ll2dot} tool generates CFGs (in the DOT file format) for each function of the LLVM IR, as described in section \ref{sec:design_control_flow_graph_generation} and demonstrated in appendix \ref{app:control_flow_graph_generation_example}. Secondly, the \texttt{restructure} tool structures the CFGs to recover high-level control-flow primitives from the underlying LLVM IR, as described in section \ref{sec:design_control_flow_analysis} and demonstrated in appendix \ref{app:restructure_example}.

% * R21 - Support language-agnostic interaction with other components

The components of the decompilation pipeline support language-agnostic interaction with other components (\textbf{R21}), as they only communicate using well-defined input and output; specifically LLVM IR\footnote{LLVM Language Reference Manual: \url{http://llvm.org/docs/LangRef.html}}, DOT\footnote{The DOT Language: \url{http://www.graphviz.org/doc/info/lang.html}}, JSON\footnote{The JSON Data Interchange Format: \url{https://tools.ietf.org/html/rfc7159}} and Go\footnote{The Go Programming Language Specification: \url{https://golang.org/ref/spec}} input and output. The interaction between the components of the decompilation pipeline is demonstrated in four steps, when decompilation LLVM IR to Go. Firstly, the control flow graph generation component (see section \ref{sec:design_control_flow_graph_generation}) parses LLVM IR assembly to produce an unstructured CFG (in the DOT file format); as demonstrated in appendix \ref{app:control_flow_graph_generation_example}. Secondly, the control flow analysis component (see section \ref{sec:design_control_flow_analysis}) analyzes the unstructured CFG (in the DOT file format) to produce a structured CFG (in JSON format); as demonstrated in appendix \ref{app:restructure_example}. Thirdly, the code generation component (see section \ref{sec:design_code_generation}) decompiles the structured LLVM IR assembly into unpolished Go code; as demonstrated in appendix \ref{app:code_generation_example}. Lastly, the post-processing tool (see section \ref{sec:design_post-processing}) improves the quality of the unpolished Go code, by applying a set of source code transformations; as demonstrated in appendix \ref{app:post-processing_example}.


