% === [ Implementation ] =======================================================

\section{Implementation}

% TODO: Note from Janka: The Implementation section should be strictly related to the software itself.

% TODO: Brainstorm about which sections are actually relevant and how they should be structured.

% TODO: Mention the following trivias:
%    - Identify unused tokens (hash and backspace) in the C++ code base and submit a patch which was commited to remove these. (http://reviews.llvm.org/D7248)
%    - Discuss API design with members of the open source community.
%    - Ask experienced LLVM developers of insight into possible inconsistencies with LLVM IR. Some highlighted inconsistent behaviour and some were intended behaviour. (LLVM-dev mailing list)

% Hints for Computer System Design (1983) - Butler Lampson
%    "Handle normal and worst case seperately"

foo

% --- [ LLVM IR Library ] ------------------------------------------------------

\subsection{LLVM IR Library}

foo

% --- [ Control Flow Graph Generation Tool ] -----------------------------------

\subsection{Control Flow Graph Generation Tool}

foo

% --- [ Subgraph Isomorphism Search Algorithm ] --------------------------------

\subsection{Subgraph Isomorphism Search Algorithm}

Implementing the subgraph isomorphism search algorithm was without doubt the most difficult endeavour of the entire project. It took roughly five iterations of implementing, evaluating and rethinking the algorithm to find an approach which felt right and another two iterations to develop a working implementation which passed all the test cases.

foo

% TODO: Incorporate notes from iso_algorithm_notes.txt.

% --- [ Control Flow Analysis Tool ] -------------------------------------------

\subsection{Control Flow Analysis Tool}

foo

% --- [ Decompiler Back-end Tool ] ---------------------------------------------

\subsection{Decompiler Back-end Tool}

foo

% --- [ Documentation ] --------------------------------------------------------

\subsection{Documentation}

% TODO: Add example use cases (perhaps also covering related projects such as McSema)?

A set of source code analysis tools are used to automate the generation and presentation of documentation. The main benefit of this approach is that only one version of the documentation has to be maintained and it is kept within the source code, thus preventing it from falling out of sync with the implementation. UNIX manual pages are generated for command line tools using \texttt{mango} \cite{mango}, which locates the relevant comments and command line flag definitions in the source code. Library documentation is presented using \texttt{godoc} \cite{godoc} (a tool similar to \texttt{doxygen}), and may be accessed through a web or command line interface.

The GoDoc.org server hosts an instance of \texttt{godoc} which presents the documentation of publicly available source code repositories. An online version of the documentation has been made available for each artefact using this service.

\begin{itemize}
	\item Library for interacting with LLVM IR (\textit{work in progress}) \\ \url{https://godoc.org/github.com/mewlang/llvm}
	\item LLVM IR Control Flow Graph generation tool \\ \url{https://godoc.org/github.com/mewrev/ll2dot}
	\item Subgraph isomorphism search algorithms for reconstructing high-level control flow primitives and related tools \\ \url{https://godoc.org/github.com/mewrev/graphs}
	\item Decompiler back-end tool (\textit{proof of concept}) \\ \url{https://godoc.org/github.com/mewrev/ll2go}
\end{itemize}
