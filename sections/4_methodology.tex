% === [ Methodology ] ==========================================================

% <howto>
% * how is development structured? (life cycle)

\section{Methodology}

\textbf{QUESTION}: \textit{Should the methodology be written using future, current or past tense?}

% TODO: <ask> Should the methodology be written using future, current or past tense?

% TODO: Rewrite to make this section feel right. It is currently too succinct. Elaborate on some topics, and potentially split into multiple sections.

\textbf{NOTE}: \textit{The following paragraph has been moved here from the PID and will therefore feel out of place. The intention is to reformulate it.}

No single methodology will be used for this project, but rather a combination of techniques which have been proven to work well in practice. Continuous Integration will be used throughout the development process to monitor the build status, test cases, and code coverage of each component. The API stability of each component will be governed using Semantic Versioning (vMAJOR.MINOR.PATCH). Prior to version 1.0 the components will be written using either throw-away prototyping or iterative development and the API may change drastically. The implementation correctness will be verified using Test-driven Development (TDD) where feasible, with test cases written prior to the actual implementation of the API.

% TODO: Explain why this specific methodology was chosen, and justify the methods. The justification may be required to be quite detailed, with in-text references.

% TODO: Add
%    - Attempt to find flaws in the design by stress testing it through proof of concept implementations.
%    - Once the design is deemed mature, begin an iterative process of implementation and testing.

% --- [ Open Source Development ] ----------------------------------------------

\subsection{Open Source Development}

% TODO: Add note about: Eric Raymond observed in his essay The Cathedral and the Bazaar that announcing the intent for a project is usually inferior to releasing a working project to the public.

%    - Engage in discussions with the open source community during the design process of any library intended for third party use; specifically the LLVM IR libraries.

% TODO: Add
%    - All work was made available on GitHub from day one and the source code was released into the public domain to encourage open source adaptation and interaction.

foo

% ~~~ [ Revision Control System ] ~~~~~~~~~~~~~~~~~~~~~~~~~~~~~~~~~~~~~~~~~~~~~~

\subsubsection{Revision Control System}

% TODO: Add
% - Don not be afraid to throw away code and trying new ideas, that is what revision control systems are for!

foo

% --- [ Public Issue Tracking ] ------------------------------------------------

\subsubsection{Public Issue Tracking}

% TODO: Add
%    - The project plan will be organized using the GitHub issue tracker, where each issue corresponds to a task. Milestones, containing a set of issues, will track the progress of the project and enforce deadlines.
%    - To facilitate time management each task will be tracked using the GitHub issue tracker, and larger tasks will be divided into suitably sized sub-tasks. The smaller sub-tasks help maintain focus and make it easier to establish reasonable deadlines.

% TODO: Add project plan?
%    - The project plan will be managed using the GitHub issue tracker, and actively maintained throughout the project. Each task (research topic, unresolved problem, software feature, etc) will be allocated a dedicated issue tracking its progress. The issues include relevant discussions of identified risks, potential problems and proposed solutions. The project plan provides an overview of the active issues, using milestones to group issues and assign deadlines.

foo

% --- [ Throwaway Prototyping ] ------------------------------------------------

\subsection{Throwaway Prototyping}

foo

% --- [ Evolutionary Prototyping ] ---------------------------------------------

\subsection{Evolutionary Prototyping}

foo

% --- [ Test-driven Development ] ----------------------------------------------

\subsection{Test-driven Development}

foo
