% === [ Methodology ] ==========================================================

% <howto>
% * how is development structured? (life cycle)
% * tense: usually simple past and passive.

\section{Methodology}
\label{sec:methodology}

No single methodology was used for this project, but rather a combination of software development techniques (such as test-driven development and continuous integration) which have been shown to work well in practice for other open source projects. This project has been developed in the open from day one, using public source code repositories and issue trackers. To encourage open source adaptation, the software artefacts and the project report have been released into the public domain, and are made available on GitHub; as further described in section \ref{sec:intro_deliverables}. Throughout the course of the project a public discussion has been held with other members of the open source community to clarify the requirements and validate the design of the LLVM IR library, and to investigate inconsistent behaviours in the LLVM reference implementation; as described in section \ref{sec:impl_llvm_ir_library}.

The API stability of each component have been governed using semantic versioning (e.g. \texttt{vMAJOR.MINOR.PATCH})\footnote{Semantic Versioning: \url{http://semver.org/}}. Prior to version 1.0 the components have been developed using throwaway prototyping (see section \ref{sec:method_throwaway_prototyping}), which facilitated rapid iteration cycles. %When the major design decisions stabilize

% TODO: <note> remove?
% "An evolutionary approach to development of the software artefact has been applied. New ideas have been tested out and explored further if early results are productive."

% The API stability of each component will be governed using Semantic Versioning (vMAJOR.MINOR.PATCH). Prior to version 1.0 the components will be written using either throw-away prototyping or iterative development and the API may change drastically.

% TODO: Explain why this specific methodology was chosen, and justify the methods. The justification may be required to be quite detailed, with in-text references.

% TODO: Add
%    - Attempt to find flaws in the design by stress testing it through proof of concept implementations.
%    - Once the design is deemed mature, begin an iterative process of implementation and testing.

% ### Open Source Development

% TODO: Add note about: Eric Raymond observed in his essay The Cathedral and the Bazaar that announcing the intent for a project is usually inferior to releasing a working project to the public.

%    - Engage in discussions with the open source community during the design process of any library intended for third party use; specifically the LLVM IR libraries.

% TODO: Add
%    - All work was made available on GitHub from day one and the source code was released into the public domain to encourage open source adaptation and interaction.

% ###### Revision Control System

% TODO: Add
% - Don not be afraid to throw away code and trying new ideas, that is what revision control systems are for!

% TODO: Add:
% * feature branches

% ###### Public Issue Tracking

% TODO: Add
%    - The project plan will be organized using the GitHub issue tracker, where each issue corresponds to a task. Milestones, containing a set of issues, will track the progress of the project and enforce deadlines.
%    - To facilitate time management each task will be tracked using the GitHub issue tracker, and larger tasks will be divided into suitably sized sub-tasks. The smaller sub-tasks help maintain focus and make it easier to establish reasonable deadlines.

% TODO: Add project plan?
%    - The project plan will be managed using the GitHub issue tracker, and actively maintained throughout the project. Each task (research topic, unresolved problem, software feature, etc) will be allocated a dedicated issue tracking its progress. The issues include relevant discussions of identified risks, potential problems and proposed solutions. The project plan provides an overview of the active issues, using milestones to group issues and assign deadlines.

% TODO: <note> remove?
% * It is important to decide at the start of the project on the tasks to be carried out (to develop the artefact and to write the report), when they will take place and how long they should take.
%   - It is also essential that you monitor your progress through the project, and show evidence of monitoring in the report.

% TODO: <note> remove?
% * Make a rough project plan; before 26 sep.
%    - Make use of GitHub issues and milestones with deadlines to track development.

% === [ Subsections ] ==========================================================

% --- [ Operational Prototyping ] ----------------------------------------------

\subsection{Operational Prototyping}

The software artefacts were implemented using two distinct stages. The aim of the first stage was to get a better understanding of the problem domain, to identify suitable data structures, and to arrive at a solid approach for solving the problem. To achieve these objectives, a set of throwaway prototypes (see section \ref{sec:method_throwaway_prototyping}) were iteratively implemented, discarded and redesigned until the requirements of the artefact were well understood and a mature design had emerged. The aim of the second stage was to develop a production quality software artefact based on the insights gained from the first stage. To achieve this objective, evolutionary prototyping (see section \ref{sec:method_evolutionary_prototyping}) was used to develop a solid foundation for the software artefact and incrementally extend its capabilities by implementing one feature at the time, starting with the features that were best understood.

This approach is very similar to the operational prototyping methodology, which was proposed by A. Davis in 1992. One important concept in operational prototyping is the notion of a quality baseline, which is implemented using evolutionary prototyping and represents a solid foundation for the software artefact. Throwaway prototypes are implemented on top of the quality baseline for poorly understood parts of the system, to gain further insight into their requirements. The throwaway prototypes are discarded once their part of the system is well-understood, at which point the well-understood parts are carefully reimplemented and incorporated into the evolutionary prototype to establish a new quality baseline \cite{operational_prototyping}. In summary, throwaway prototyping is used to \textit{identify} good solutions to problems, while evolutionary prototyping is used to \textit{implement} identified solutions.

A major benefit with this approach is that it makes it easy to track the evolution of the design, by referring back to the throwaway prototypes which gave new insight into the problem domain; as demonstrated when tracking the evolution of the subgraph isomorphism search algorithm in section \ref{sec:impl_subgraph_isomorphism_search_library}. A concrete risk with operational prototyping is that throwaway prototypes may end up in production systems, if not discarded as intended. As mentioned in section \ref{sec:method_throwaway_prototyping}, the throwaway prototypes enable rapid iteration cycles by ignoring several areas of quality software (e.g. maintainability, efficiency and usability) and should therefore never end up in production systems. The use of revision control systems could help mitigate this risk, as they tracks old versions of the source code which may lower the psychological threshold for removing code (e.g. the code is not permanently removed, and may later be recovered if needed).

% --- [ Subsubsections ] -------------------------------------------------------

% ~~~ [ Throwaway Prototyping ] ~~~~~~~~~~~~~~~~~~~~~~~~~~~~~~~~~~~~~~~~~~~~~~~~

\subsubsection{Throwaway Prototyping}
\label{sec:method_throwaway_prototyping}

Throwaway prototyping may be used in the early stages of development to gain insight into a problem domain, by rapidly implementing \textit{throw away} prototypes which will be discarded upon completion. These prototypes aim to challenge design decisions, stress test implementation strategies, identify further research requirements, and provide a better understanding and intuition for the problem domain and potential solutions. Throwaway prototypes are developed in an informal manner and are not intended to become part of the final artefact. This allows rapid iterations, as several areas of quality software (e.g. maintainability, efficiency and usability) may be ignored. When utilized appropriately, throwaway prototyping makes the development very time effective as costly changes are applied early on \cite{operational_prototyping}.

% TODO: Remove all of the below comments?

% quick-and-dirty throwaway prototypes to gain an understanding of the requirements and an insight into the problem domain.

% rapid prototypes on top of a solid evolutionary base.

% * Build prototypes quickly, used for experimentation.
% * Implements requirements that are poorly understood. (why build a prototype that is already understood, only to throw it away?)
% * Discarded after the desired information is learned.

% After developing the prototype the developer incorporates what was learned. Works well in isolation to verify relatively small parts of complex problems.

% ~~~ [ Evolutionary Prototyping ] ~~~~~~~~~~~~~~~~~~~~~~~~~~~~~~~~~~~~~~~~~~~~~

\subsubsection{Evolutionary Prototyping}
\label{sec:method_evolutionary_prototyping}

Evolutionary prototyping focuses on implementing the parts of the system which are well understood, as acknowledged by the quote from A. Davis presented in figure~\ref{fig:evolutionary_prototyping}. This is in direct contrast to throwaway prototyping (see section~\ref{sec:method_throwaway_prototyping}), which aims to provide insight into the requirements of the poorly understood parts of the system. From the initial implementation, evolutionary prototypes are built as robust systems which evolve over time. The evolutionary prototypes may lack functionality, but the functionality they implement is generally of high enough quality to be used in production systems~\cite{operational_prototyping}.

\begin{figure}[htbp]
	\begin{quote}
		\textit{``… evolutionary prototyping acknowledges that we do not understand all the requirements and builds only those that are well understood.''}
	\end{quote}
	\caption{An extract from \textit{Operational prototyping: A new development approach} by A. Davis in 1992~\cite{operational_prototyping}.}
	\label{fig:evolutionary_prototyping}
\end{figure}


% --- [ Continuous Integration ] -----------------------------------------------

\subsection{Continuous Integration}

The Continuous Integration (CI) practice originated from the Extreme Programming methodology~\cite{extreme_programming} but has reached a much broader audience in recent years. Today most large scale software projects rely on CI server farms to continuously compile and test new versions of the source code. This project makes heavy use of CI to monitor the build status, test cases and code coverage of each software artefact, as further described in section~\ref{sec:ver_continuous_integration}.

