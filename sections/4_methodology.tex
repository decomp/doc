% === [ Methodology ] ==========================================================

% <howto>
% * how is development structured? (life cycle)

\section{Methodology}
\label{sec:methodology}

\textbf{QUESTION}: \textit{Should the methodology be written using future, current or past tense?}

% TODO: <ask> Should the methodology be written using future, current or past tense?

% TODO: Rewrite to make this section feel right. It is currently too succinct. Elaborate on some topics, and potentially split into multiple sections.

\textbf{NOTE}: \textit{The following paragraph has been moved here from the PID and will therefore feel out of place. The intention is to reformulate it.}

No single methodology will be used for this project, but rather a combination of techniques which have been proven to work well in practice. Continuous Integration will be used throughout the development process to monitor the build status, test cases, and code coverage of each component. The API stability of each component will be governed using Semantic Versioning (vMAJOR.MINOR.PATCH). Prior to version 1.0 the components will be written using either throw-away prototyping or iterative development and the API may change drastically. The implementation correctness will be verified using Test-driven Development (TDD) where feasible, with test cases written prior to the actual implementation of the API.

% TODO: Explain why this specific methodology was chosen, and justify the methods. The justification may be required to be quite detailed, with in-text references.

% TODO: Add
%    - Attempt to find flaws in the design by stress testing it through proof of concept implementations.
%    - Once the design is deemed mature, begin an iterative process of implementation and testing.

% --- [ Open Source Development ] ----------------------------------------------

\subsection{Open Source Development}

% TODO: Add note about: Eric Raymond observed in his essay The Cathedral and the Bazaar that announcing the intent for a project is usually inferior to releasing a working project to the public.

%    - Engage in discussions with the open source community during the design process of any library intended for third party use; specifically the LLVM IR libraries.

% TODO: Add
%    - All work was made available on GitHub from day one and the source code was released into the public domain to encourage open source adaptation and interaction.

foo

% ~~~ [ Revision Control System ] ~~~~~~~~~~~~~~~~~~~~~~~~~~~~~~~~~~~~~~~~~~~~~~

\subsubsection{Revision Control System}

% TODO: Add
% - Don not be afraid to throw away code and trying new ideas, that is what revision control systems are for!

foo

% --- [ Public Issue Tracking ] ------------------------------------------------

\subsubsection{Public Issue Tracking}

% TODO: Add
%    - The project plan will be organized using the GitHub issue tracker, where each issue corresponds to a task. Milestones, containing a set of issues, will track the progress of the project and enforce deadlines.
%    - To facilitate time management each task will be tracked using the GitHub issue tracker, and larger tasks will be divided into suitably sized sub-tasks. The smaller sub-tasks help maintain focus and make it easier to establish reasonable deadlines.

% TODO: Add project plan?
%    - The project plan will be managed using the GitHub issue tracker, and actively maintained throughout the project. Each task (research topic, unresolved problem, software feature, etc) will be allocated a dedicated issue tracking its progress. The issues include relevant discussions of identified risks, potential problems and proposed solutions. The project plan provides an overview of the active issues, using milestones to group issues and assign deadlines.

foo

% --- [ Prototyping ] ----------------------------------------------------------

\subsection{Prototyping}

% Prototypes are partial implementations of systems and they are built to learn more about a specific problem domain or a solution to a problem in this domain.

% Prototypes are partial implementations of systems which may be implemented quickly to gain insight into the feasibility of an envisioned design,

% Prototypes are partial implementations of systems which are developed rapidly to gain insight into a problem domain and stress test the feasibility of an envisioned design or solution.

% TODO: Add; Prototyping is valuable as developers quickly gain insight into the feasibility of an envisioned design, get a more realistic sense of the required time to implement full functionality and may then re-allocate time and rethink the design early to achive a functional software artefact before the deadline.

% Proof of concept.

% Refine the design.

% Showcase development (software demonstrations).

% pro: Reduced time and cost.
% con: Insufficient analysis: Risk of loosing track of the larger picture.
% con: Developer attachment: The efforts of producing a prototype may endear them developers. Risk of attempting to convert a prototype into a final system, even if it had a poor underlying architecture. Solution: revision control system. Throw it away, start over and if the new solution ends up worse, simply revert to the working prototype.

% It has been suggested that prototyping is ill-fitted for systems with little user interaction and heavy computational tasks; which is the nature of the artefact of this project. The difference between throwaway and evolutionary prototyping are however substantial, and for each component of the system the author has carefully considered which strategy would be most beneficial to employ.
%ref: John Crinnion: Evolutionary Systems Development, a practical guide to the use of prototyping within a structured systems methodology. Plenum Press, New York, 1991. Page 18.

foo

% Operational Prototyping

% Alan Davis proposed the methodology.

% "It offers the best of both the quick-and-dirty and conventional-development worlds in a sensible manner. Designers develop only well-understood features in building the evolutionary baseline, while using throwaway prototyping to experiment with the poorly understood features." \cite{operational_prototyping}

% ~~~ [ Throwaway Prototyping ] ~~~~~~~~~~~~~~~~~~~~~~~~~~~~~~~~~~~~~~~~~~~~~~~~

\subsubsection{Throwaway Prototyping}

% TODO: Use throwaway prototyping when a good solution is not known in advance. The code is intended to be thrown away, and works to stress test a concept, verify if an idea for a potential solution works and get a better understanding and intuition about the problem domain and potential solutions.

% Throwaway prototyping allows quick iterations and the rapid ..

% Discard, never intended to become part of the final artefact.

% Early stage, Informal approach; Gain insight into the difficulties of the problem domain; re-examine design decisions, implementation strategies, need for further research. The prototype is discarded, or \textit{thrown away},

% Major benefit is the rapid approach

% Speed is achived by ignoring common areas for producing quality software, such as extensibility, maintainability, etc.
% - Maintainability.
% - Dependability.
% - Efficiency.
% - Usability.


% Make changes early; time effective.

foo

% ~~~ [ Evolutionary Prototyping ] ~~~~~~~~~~~~~~~~~~~~~~~~~~~~~~~~~~~~~~~~~~~~~

\subsubsection{Evolutionary Prototyping}

% TODO: Use evolutionary prototyping when the developer has good knowledge about how to implement a part of the system, and continue to build the system starting with the components that are well known and where the developer has a good idea of how to implement. As these parts are developed the developer will gain insight into how other parts of the system may interact and therefore gain confidence in implementing those components.

% Goal is to build a robust prototype, in a structured manner; which may be refined continuously. system contiunously refined and rebuilt.

\begin{quote}
	\textit{"… evolutionary prototyping acknowledges that we do not understand all the requirements and builds only those that are well understood."} \cite{operational_prototyping}
\end{quote}

% In contrast to throwaway prototypes, evolutionary prototypes are stable and may be used straight away by other parts of the system. They may lack functionality, but the functionality that is implemented is generally of high quality.

% Developers can focus on developing parts of the system that they understand. Do not implement poorly understood features. (reformulate!)

foo

% --- [ Test-driven Development ] ----------------------------------------------

\subsection{Test-driven Development}

foo
