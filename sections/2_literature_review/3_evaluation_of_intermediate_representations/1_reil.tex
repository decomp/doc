% ~~~ [ REIL ] ~~~~~~~~~~~~~~~~~~~~~~~~~~~~~~~~~~~~~~~~~~~~~~~~~~~~~~~~~~~~~~~~~

\subsubsection{REIL}
\label{sec:lit_review_reil}

The Reverse Engineering Intermediate Language (REIL) is a very simple and platform-independent assembly language. The REIL instruction set contains only 17 different instructions, each with exactly three (possibly empty) operands. The first two operands are always used for input and the third for output (except for the conditional jump instruction which uses the third operand as the jump target). Furthermore, each instruction has at most one effect on the global state and never any side-effects (such as setting flags) \cite{reil_paper,reil_spec}. Thanks to the simplicity of REIL a full definition of its instruction set has been provided in appendix \ref{app:reil_instructions}, which includes examples of each instruction and defines their syntax and semantics (in pseudo C-code).

When translating native assembly (e.g. x86) into REIL, the original addresses of each instruction is left shifted by 8 bits to allow 256 REIL instructions per address. Each native instruction may therefore be translated into one or more REIL instructions (at most 256), which is required to correctly map the semantics of complex instructions with side-effects. This systematic approach of deriving instruction addresses has a fundamental implication, REIL supports indirect branches (e.g. \texttt{call rax}) by design.

The language was originally designed to assist static code analysis and translators from native assembly (x86, PowerPC-32 and ARM-32) to REIL are commercially available. However, the project home page has not been updated since Google acquired zynamics in 2011. Since then approximately 10 papers have been published which references REIL and the adaptation of the language within the open source community seems limited. As of 2015-01-04 only three implementations existed on GitHub (two in Python\footnote{Binary Analysis and RE Framework: \url{https://github.com/programa-stic/barf-project}}\footnote{REIL translation library: \url{https://github.com/c01db33f/pyreil}} and one in C\footnote{Binary introspection toolkit: \url{https://github.com/aoikonomopoulos/bit}}), and the most popular had less than 25 watchers, 80 stars and 15 forks.

A fourth implementation was released at the 15th of March 2015 however, and in less than two weeks OpenREIL had become the most popular REIL implementation on GitHub. The OpenREIL project extends the original REIL instruction set with signed versions of the multiplication, division and modulo instructions, and includes convenience instructions for common comparison and binary operations. OpenREIL is currently capable of translating x86 executables to REIL, and aims to include support for ARM and x86-64 in the future. Furthermore, the OpenREIL project intends to implement support for translating REIL to LLVM IR, thus bridging the two intermediate representations \cite{openreil}.
