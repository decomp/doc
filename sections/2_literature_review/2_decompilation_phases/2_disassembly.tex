% ~~~ [ Disassembly ] ~~~~~~~~~~~~~~~~~~~~~~~~~~~~~~~~~~~~~~~~~~~~~~~~~~~~~~~~~~

\subsubsection{Disassembly}
\label{sec:lit_review_disassembly}

The disassembly phase (referred to as the \textit{syntactic analysis phase} by C. Cifuentes) is responsible for decoding the raw machine instructions of the executable segments into assembly. The computer architecture dictates how the assembly instructions and their associated operands are encoded. Generally CISC architectures (e.g. x86) use variable length instruction encoding (e.g. instructions occupy between 1 and 15 bytes in x86) and allow memory addressing modes for most instructions (e.g. arithmetic instructions may refer to memory locations in x86) \cite{x86_manual}. In contract, RISC architectures (e.g. ARM) generally use fixed-length instruction encoding (e.g. instructions always occupy 4 bytes in AArch64) and only allow memory access through load-store instructions (e.g. arithmetic instructions may only refer to registers or immediate values in ARM) \cite{arm_manual}.

One of the main problems of the disassembly phase is how to separate code from data. In the Von Neumann architecture the same memory unit may contain both code and data. Furthermore, the data stored in a given memory location may be interpreted as code by one part of the program, and as data by another part. In contrast, the Harvard architecture uses separate memory units for code and data \cite{von_neumann_vs_harvard}. Since the use of the Von Neumann architecture is wide spread, solving this problem is fundamental for successful disassemblers.

The most basic disassemblers (e.g. \texttt{objdump} and \texttt{ndisasm}) use linear descent when decoding instructions. Linear descent disassemblers decode instructions consecutively from a given entry point, and contain no logic for tracking the flow of execution. This approach may produce incorrect disassembly when code and data are intermixed (e.g. switch tables stored in executable segments) \cite{reverse_comp}; as illustrated in figure \ref{fig:comparison_between_linear_and_recursive_descent}. More advanced disassemblers (e.g. IDA) often use recursive descent when decoding instructions, to mitigate this issue.

Recursive descent disassemblers track the flow of execution and decode instructions from a set of locations known to be reachable from a given entry point. The set of reachable locations is initially populated with the entry points of the binary (e.g. the \texttt{start} or \texttt{main} function of executables and the \texttt{DllMain} function of shared libraries). To disassemble programs, the recursive descent algorithm will recursively pop a location from the reachable set, decode its corresponding instruction, and add new reachable locations from the decoded instruction to the reachable set, until the reachable set is empty. When decoding non-branching instructions (e.g. \texttt{add}, \texttt{xor}), the immediately succeeding instruction is known to be reachable (as it will be executed after the non-branching instruction) and its location is therefore added to the reachable set. Similarly, when decoding branching instructions (e.g. \texttt{br}, \texttt{ret}), each target branch (e.g. the conditional branch and the default branch of conditional branch instructions) is known to be reachable and therefore added to the reachable set; unless the instruction has no target branches, as is the case with return instructions. This approach is applied recursively until all paths have reached an end-point, such as a return instruction, and the reachable set is empty. To prevent cycles, the reachable locations are tracked and only added once to the reachable set.

\begin{figure}[htbp]
	\centering
	\begin{subfigure}[t]{0.49\textwidth}
		\lstinputlisting[language=nasm, style=nasm, tabsize=2]{inc/2_lit_review/hello/hello_linear.asm}
		\caption{Disassembly from \texttt{objdump} and \texttt{ndisasm}\protect\footnotemark.}
	\end{subfigure}
	\qquad
	\begin{subfigure}[t]{0.35\textwidth}
		\lstinputlisting[language=nasm, style=nasm, tabsize=2]{inc/2_lit_review/hello/hello_recursive.asm}
		\caption{Disassembly from IDA.}
	\end{subfigure}
	\caption{The disassembly produced by a linear descent parser (left) and a recursive descent parser (right) when analysing a simple \textit{``hello world''} program that stores the \texttt{hello} string in the executable segment.}
	\label{fig:comparison_between_linear_and_recursive_descent}
\end{figure}
\footnotetext{The Netwide Disassembler: \url{http://www.nasm.us/doc/nasmdoca.html}}

A limitation with recursive descent disassemblers is that they cannot track indirect branches (e.g. branch to the address stored in a register) without additional information, as it is impossible to know the branch target of indirect branch instructions only by inspecting individual instructions (e.g. \texttt{jmp eax} gives no information about the value of \texttt{eax}). One solution to this problem is to utilize symbolic execution engines, which emulate the CPU and execute the instructions along each path to give information about the values stored in registers and memory locations. Using this approach, the target of indirect branch instructions may be derived from the symbolic execution engine by inspecting the values of registers and memory locations at the invocation site \cite{mcsema}. Symbolic execution engines are no silver bullets, and introduce a new range of problems; such as cycle accurate modelling of the CPU, idiosyncrasies related to memory caches and instruction pipelining, and potentially performance and security issues.

Malicious software often utilize anti-disassembly techniques to hinder malware analysis. One such technique exploits the fact that recursive descent parsers follow both the conditional and the default branch of conditional branch instructions, as demonstrated in figure \ref{fig:anti-disassembly}. The recursive descent parser cannot decode the target instructions of both the conditional branch (i.e. \texttt{fake+1}) and the default branch (i.e. \texttt{fake}) of the conditional branch instruction at line 3, because the conditional branch targets the middle of a \texttt{jmp} instruction which would be decoded if traversing the default branch. As both branches cannot be decoded, the recursive descent parser is forced to choose one of them; and in this case the \texttt{fake} branch was disassembled, thus disguising the potentially malicious code of the conditional branch \cite{anti_disassembly}.

\begin{figure}[htbp]
	\centering
	\begin{subfigure}[t]{0.59\textwidth}
		\lstinputlisting[language=nasm, style=nasm, tabsize=2]{inc/2_lit_review/hello/anti_orig.asm}
		\caption{Original assembly.}
	\end{subfigure}
	\qquad
	\begin{subfigure}[t]{0.34\textwidth}
		\lstinputlisting[language=nasm, style=nasm, tabsize=2]{inc/2_lit_review/hello/anti_fail.asm}
		\caption{Disassembly from IDA.}
	\end{subfigure}
	\caption{The original assembly (left) contains an anti-disassembly trick which causes the recursive descent parser to fail (right).}
	\label{fig:anti-disassembly}
\end{figure}

The anti-disassembly technique presented in figure \ref{fig:anti-disassembly} may be mitigated using symbolic execution. The symbolic execution engine could verify that the conditional branch instruction at line 3 always branches to the conditional branch (i.e. \texttt{fake+1}) and never to the default branch (i.e. \texttt{fake}). The conditional branch instruction may therefore be replaced with an unconditional branch instruction to \texttt{fake+1}, the target of which corresponds to the \texttt{mov} instruction at line 6. Please note that this is inherently a game of cat-and-mouse, as the anti-disassembly techniques could be extended to rely on network activity, file contents, or other external sources which would require the symbolic execution environment to be extended to handle such cases.

To conclude, the disassembly phase deals with non-trivial problems, some of which are very difficult to automate. Interactive disassemblers (such as IDA) automate what may reasonably be automated, and rely on human intuition and problem solving skills to resolve any ambiguities and instruct the disassembler on how to deal with corner cases; as further described in section \ref{sec:rel_work_hex-rays_decompiler}.
