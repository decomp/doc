% --- [ Future Work ] ----------------------------------------------------------

\subsection{Future Work}

This design enables the end-user to replace, refine, and interact with each stage of the decompilation pipeline and to use the self-contained components for other purposes, not yet envisioned by the component authors.

The middle-end module of the decompilation pipeline uses an intermediate representation (LLVM IR) to separate heavy decompilation tasks (e.g. control flow analysis) from concerns related to the source (e.g. x86 assembly) and target (e.g. Go) language; thus making it easy to implement additional front-ends (e.g. support for MIPS assembly) or back-ends (e.g. support for Python output).

% TODO: Add notes from IDEAS.txt.
%    - grind

% * Finish developing the LLVM IR library and replace the C++ binding with a pure Go implementation. This should decrese build times significantly and simplify the overall complexity of the project by several orders of magnitude.

% * Formal Grammar for a subset of LLVM IR
%    - Mention previous (partial but incomplete) work.

% * Third Party Adaptation

% * Data Flow Analysis.

% TODO: Proof-of-concept. Implement a back-end for another language and written in another language. This would stress test the language-agnostic aspects of the design, thus making sure that the heavy-lifting is done in the middle-end and not in ll2go.

% TODO: Add ref to rsc's grind tool.

% * Stress test the design by implementing another back-end written in and for another programming language.

% TODO: Fuzz the LLVM IR parser by generating valid LLVM from the EBNF grammar (similar to gosmith).

% TODO: Investigate various ways to mitigate the limitations of the control flow analysis design.

foo
