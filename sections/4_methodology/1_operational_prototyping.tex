% --- [ Operational Prototyping ] ----------------------------------------------

\subsection{Operational Prototyping}

% TODO: reformulate!
% Construct quick-and-dirty throwaway prototypes on top of the quality baseline software.

% After new insight has been gained the developer throws away the prototype and incorporates the new features into the evolutionary prototype using conventional development practices; thus creating a new quality baseline.

% Benefits, it is easy to track the reason for a change to the baseline; and see how the throwaway prototypes gave insight into the problem domain and influences the underlying decisions taken when developing the quality baseline.

% Fighting the urge to keep throwaway prototypes!



% TODO: Use throwaway prototyping when a good solution is not known in advance.

% Prototypes are partial implementations of systems and they are built to learn more about a specific problem domain or a solution to a problem in this domain.

% Prototypes are partial implementations of systems which may be implemented quickly to gain insight into the feasibility of an envisioned design,

% Prototypes are partial implementations of systems which are developed rapidly to gain insight into a problem domain and stress test the feasibility of an envisioned design or solution.

% TODO: Add; Prototyping is valuable as developers quickly gain insight into the feasibility of an envisioned design, get a more realistic sense of the required time to implement full functionality and may then re-allocate time and rethink the design early to achive a functional software artefact before the deadline.

% Proof of concept.

% Refine the design.

% Showcase development (software demonstrations).

% pro: Reduced time and cost.
% con: Insufficient analysis: Risk of loosing track of the larger picture.
% con: Developer attachment: The efforts of producing a prototype may endear them developers. Risk of attempting to convert a prototype into a final system, even if it had a poor underlying architecture. Solution: revision control system. Throw it away, start over and if the new solution ends up worse, simply revert to the working prototype.

% It has been suggested that prototyping is ill-fitted for systems with little user interaction and heavy computational tasks; which is the nature of the artefact of this project. The difference between throwaway and evolutionary prototyping are however substantial, and for each component of the system the author has carefully considered which strategy would be most beneficial to employ.
%ref: John Crinnion: Evolutionary Systems Development, a practical guide to the use of prototyping within a structured systems methodology. Plenum Press, New York, 1991. Page 18.

foo

% Operational Prototyping

% Alan Davis proposed the methodology.

% "It offers the best of both the quick-and-dirty and conventional-development worlds in a sensible manner. Designers develop only well-understood features in building the evolutionary baseline, while using throwaway prototyping to experiment with the poorly understood features." \cite{operational_prototyping}

Develop iteratively when the solution is known, and use throw away prototyping when the solution is unknown.

% --- [ Subsubsections ] -------------------------------------------------------

% ~~~ [ Throwaway Prototyping ] ~~~~~~~~~~~~~~~~~~~~~~~~~~~~~~~~~~~~~~~~~~~~~~~~

\subsubsection{Throwaway Prototyping}
\label{sec:method_throwaway_prototyping}

Throwaway prototyping may be used in the early stages of development to gain insight into a problem domain, by rapidly implementing \textit{throw away} prototypes which will be discarded upon completion. These prototypes aim to challenge design decisions, stress test implementation strategies, identify further research requirements, and provide a better understanding and intuition for the problem domain and potential solutions. Throwaway prototypes are developed in an informal manner and are not intended to become part of the final artefact. This allows rapid iterations, as several areas of quality software (e.g. maintainability, efficiency and usability) may be ignored. When utilized appropriately, throwaway prototyping makes the development very time effective as costly changes are applied early on \cite{operational_prototyping}.

% TODO: Remove all of the below comments?

% quick-and-dirty throwaway prototypes to gain an understanding of the requirements and an insight into the problem domain.

% rapid prototypes on top of a solid evolutionary base.

% * Build prototypes quickly, used for experimentation.
% * Implements requirements that are poorly understood. (why build a prototype that is already understood, only to throw it away?)
% * Discarded after the desired information is learned.

% After developing the prototype the developer incorporates what was learned. Works well in isolation to verify relatively small parts of complex problems.

% ~~~ [ Evolutionary Prototyping ] ~~~~~~~~~~~~~~~~~~~~~~~~~~~~~~~~~~~~~~~~~~~~~

\subsubsection{Evolutionary Prototyping}
\label{sec:method_evolutionary_prototyping}

Evolutionary prototyping focuses on implementing the parts of the system which are well understood, as acknowledged by the quote from A. Davis presented in figure~\ref{fig:evolutionary_prototyping}. This is in direct contrast to throwaway prototyping (see section~\ref{sec:method_throwaway_prototyping}), which aims to provide insight into the requirements of the poorly understood parts of the system. From the initial implementation, evolutionary prototypes are built as robust systems which evolve over time. The evolutionary prototypes may lack functionality, but the functionality they implement is generally of high enough quality to be used in production systems~\cite{operational_prototyping}.

\begin{figure}[htbp]
	\begin{quote}
		\textit{``… evolutionary prototyping acknowledges that we do not understand all the requirements and builds only those that are well understood.''}
	\end{quote}
	\caption{An extract from \textit{Operational prototyping: A new development approach} by A. Davis in 1992~\cite{operational_prototyping}.}
	\label{fig:evolutionary_prototyping}
\end{figure}

