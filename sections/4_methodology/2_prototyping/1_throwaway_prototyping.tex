% ~~~ [ Throwaway Prototyping ] ~~~~~~~~~~~~~~~~~~~~~~~~~~~~~~~~~~~~~~~~~~~~~~~~

\subsubsection{Throwaway Prototyping}

% TODO: Use throwaway prototyping when a good solution is not known in advance. The code is intended to be thrown away, and works to stress test a concept, verify if an idea for a potential solution works and get a better understanding and intuition about the problem domain and potential solutions.

% Throwaway prototyping allows quick iterations and the rapid ..

% Discard, never intended to become part of the final artefact.

% Early stage, Informal approach; Gain insight into the difficulties of the problem domain; re-examine design decisions, implementation strategies, need for further research. The prototype is discarded, or \textit{thrown away},

% Major benefit is the rapid approach

% Speed is achived by ignoring common areas for producing quality software, such as extensibility, maintainability, etc.
% - Maintainability.
% - Dependability.
% - Efficiency.
% - Usability.


% Make changes early; time effective.

foo
