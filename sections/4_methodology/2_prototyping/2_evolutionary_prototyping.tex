% ~~~ [ Evolutionary Prototyping ] ~~~~~~~~~~~~~~~~~~~~~~~~~~~~~~~~~~~~~~~~~~~~~

\subsubsection{Evolutionary Prototyping}

Evolutionary prototyping focuses on implementing the parts of the system which are well understood, as illustrated in figure \ref{fig:evolutionary_prototyping}. This is in direct contrast to throwaway prototyping (see section \ref{sec:throwaway_prototyping}), which aims to gain insight by implementing the parts of the system which are poorly understood.

\begin{figure}[htbp]
	\begin{quote}
		\textit{``… evolutionary prototyping acknowledges that we do not understand all the requirements and builds only those that are well understood.''} \cite{operational_prototyping}
	\end{quote}
	\caption{An extract from \textit{Operational prototyping: A new development approach} by A. Davis in 1992.}
	\label{fig:evolutionary_prototyping}
\end{figure}





























As the components mature, through iterative re-implementations and re-designs, the macro-level design decisions regarding core data structures and algorithms becomes clear.


% TODO: Use evolutionary prototyping when the developer has good knowledge about how to implement a part of the system, and continue to build the system starting with the components that are well known and where the developer has a good idea of how to implement. As these parts are developed the developer will gain insight into how other parts of the system may interact and therefore gain confidence in implementing those components.

% Goal is to build a robust prototype, in a structured manner; which may be refined continuously. system contiunously refined and rebuilt.

% In contrast to throwaway prototypes, evolutionary prototypes are stable and may be used straight away by other parts of the system. They may lack functionality, but the functionality that is implemented is generally of high quality.

% Developers can focus on developing parts of the system that they understand. Do not implement poorly understood features. (reformulate!)

foo
