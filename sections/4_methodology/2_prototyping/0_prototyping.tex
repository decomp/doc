% --- [ Prototyping ] ----------------------------------------------------------

\subsection{Prototyping}

% Prototypes are partial implementations of systems and they are built to learn more about a specific problem domain or a solution to a problem in this domain.

% Prototypes are partial implementations of systems which may be implemented quickly to gain insight into the feasibility of an envisioned design,

% Prototypes are partial implementations of systems which are developed rapidly to gain insight into a problem domain and stress test the feasibility of an envisioned design or solution.

% TODO: Add; Prototyping is valuable as developers quickly gain insight into the feasibility of an envisioned design, get a more realistic sense of the required time to implement full functionality and may then re-allocate time and rethink the design early to achive a functional software artefact before the deadline.

% Proof of concept.

% Refine the design.

% Showcase development (software demonstrations).

% pro: Reduced time and cost.
% con: Insufficient analysis: Risk of loosing track of the larger picture.
% con: Developer attachment: The efforts of producing a prototype may endear them developers. Risk of attempting to convert a prototype into a final system, even if it had a poor underlying architecture. Solution: revision control system. Throw it away, start over and if the new solution ends up worse, simply revert to the working prototype.

% It has been suggested that prototyping is ill-fitted for systems with little user interaction and heavy computational tasks; which is the nature of the artefact of this project. The difference between throwaway and evolutionary prototyping are however substantial, and for each component of the system the author has carefully considered which strategy would be most beneficial to employ.
%ref: John Crinnion: Evolutionary Systems Development, a practical guide to the use of prototyping within a structured systems methodology. Plenum Press, New York, 1991. Page 18.

foo

% Operational Prototyping

% Alan Davis proposed the methodology.

% "It offers the best of both the quick-and-dirty and conventional-development worlds in a sensible manner. Designers develop only well-understood features in building the evolutionary baseline, while using throwaway prototyping to experiment with the poorly understood features." \cite{operational_prototyping}
