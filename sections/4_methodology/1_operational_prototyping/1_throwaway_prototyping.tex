% ~~~ [ Throwaway Prototyping ] ~~~~~~~~~~~~~~~~~~~~~~~~~~~~~~~~~~~~~~~~~~~~~~~~

\subsubsection{Throwaway Prototyping}
\label{sec:method_throwaway_prototyping}

Throwaway prototyping may be used in the early stages of development to gain insight into a problem domain, by rapidly implementing prototypes which will be discarded upon completion. These prototypes aim to challenge design decisions, stress test implementation strategies, identify further research requirements, and provide a better understanding and intuition for the problem domain and potential solutions. Throwaway prototypes are developed in an informal manner and are not intended to become part of the final artefact. This allows rapid iterations, as several areas of quality software (e.g. maintainability, efficiency and usability) may be ignored. When utilised appropriately, throwaway prototyping makes the development very time effective as costly changes are applied early on \cite{operational_prototyping}.
