% --- [ Prototyping ] ----------------------------------------------------------

\subsection{Prototyping}

% Prototypes are partial implementations of systems and they are built to learn more about a specific problem domain or a solution to a problem in this domain.

% Prototypes are partial implementations of systems which may be implemented quickly to gain insight into the feasibility of an envisioned design,

% Prototypes are partial implementations of systems which are developed rapidly to gain insight into a problem domain and stress test the feasibility of an envisioned design or solution.

% TODO: Add; Prototyping is valuable as developers quickly gain insight into the feasibility of an envisioned design, get a more realistic sense of the required time to implement full functionality and may then re-allocate time and rethink the design early to achive a functional software artefact before the deadline.

% Proof of concept.

% Refine the design.

% Showcase development (software demonstrations).

% pro: Reduced time and cost.
% con: Insufficient analysis: Risk of loosing track of the larger picture.
% con: Developer attachment: The efforts of producing a prototype may endear them developers. Risk of attempting to convert a prototype into a final system, even if it had a poor underlying architecture. Solution: revision control system. Throw it away, start over and if the new solution ends up worse, simply revert to the working prototype.

% It has been suggested that prototyping is ill-fitted for systems with little user interaction and heavy computational tasks; which is the nature of the artefact of this project. The difference between throwaway and evolutionary prototyping are however substantial, and for each component of the system the author has carefully considered which strategy would be most beneficial to employ.
%ref: John Crinnion: Evolutionary Systems Development, a practical guide to the use of prototyping within a structured systems methodology. Plenum Press, New York, 1991. Page 18.

foo

% Operational Prototyping

% Alan Davis proposed the methodology.

% "It offers the best of both the quick-and-dirty and conventional-development worlds in a sensible manner. Designers develop only well-understood features in building the evolutionary baseline, while using throwaway prototyping to experiment with the poorly understood features." \cite{operational_prototyping}

% --- [ Subsubsections ] -------------------------------------------------------

% ~~~ [ Throwaway Prototyping ] ~~~~~~~~~~~~~~~~~~~~~~~~~~~~~~~~~~~~~~~~~~~~~~~~

\subsubsection{Throwaway Prototyping}
\label{sec:throwaway_prototyping}

Throwaway prototyping may be used in the early stages of development to gain insight into a problem domain, by rapidly implementing \textit{throw away} prototypes which will be discarded upon completion. These prototypes aim to challenge design decisions, stress test implementation strategies, identify further research requirements, and provide a better understanding and intuition for the problem domain and potential solutions. Throwaway prototypes are developed in an informal manner and are not intended to become part of the final artefact. This allows rapid iterations, as several areas of quality software (e.g. maintainability, efficiency and usability) may be ignored. When utilized appropriately, throwaway prototyping makes the development very time effective as costly changes are applied early on \cite{operational_prototyping}.

% ~~~ [ Evolutionary Prototyping ] ~~~~~~~~~~~~~~~~~~~~~~~~~~~~~~~~~~~~~~~~~~~~~

\subsubsection{Evolutionary Prototyping}

Evolutionary prototyping focuses on implementing the parts of the system which are well understood (see figure \ref{fig:evolutionary_prototyping}). This is in direct contrast to throwaway prototyping (see section \ref{sec:throwaway_prototyping}), which aims to gain insight by implementing the parts of the system which are poorly understood.

\begin{figure}[htbp]
	\begin{quote}
		\textit{``… evolutionary prototyping acknowledges that we do not understand all the requirements and builds only those that are well understood.''} \cite{operational_prototyping}
	\end{quote}
	\caption{An extract from \textit{Operational prototyping: A new development approach} by A. Davis in 1992.}
	\label{fig:evolutionary_prototyping}
\end{figure}





























As the components mature, through iterative re-implementations and re-designs, the macro-level design decisions regarding core data structures and algorithms becomes clear.


% TODO: Use evolutionary prototyping when the developer has good knowledge about how to implement a part of the system, and continue to build the system starting with the components that are well known and where the developer has a good idea of how to implement. As these parts are developed the developer will gain insight into how other parts of the system may interact and therefore gain confidence in implementing those components.

% Goal is to build a robust prototype, in a structured manner; which may be refined continuously. system contiunously refined and rebuilt.

% In contrast to throwaway prototypes, evolutionary prototypes are stable and may be used straight away by other parts of the system. They may lack functionality, but the functionality that is implemented is generally of high quality.

% Developers can focus on developing parts of the system that they understand. Do not implement poorly understood features. (reformulate!)

foo

