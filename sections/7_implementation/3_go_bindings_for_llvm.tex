% --- [ Go Bindings for LLVM ] -------------------------------------------------

\subsection{Go Bindings for LLVM}
\label{sec:go_bindings_for_llvm}

% TODO: Mention both the rice and the roses. Using the Go bindings made it possible to implement working artefacts very quickly. Especially nice to automatically gain support for the new syntax introduced in LLVM IR 3.6.0 (e.g. changes to how metadata is handled).

% TODO: Mention the need for using the Go bindings for LLVM. (time constraints).
% TODO: Mention the unnamed.patch hack. Needed access to the name of anonymous basic blocks. This name is computed by the assembly printer, and never stored.

Several components (e.g. \texttt{ll2dot}, \texttt{ll2go}) use the Go bindings for LLVM\footnote{Go bindings for LLVM: \url{https://godoc.org/llvm.org/llvm/bindings/go/llvm}} to interact with LLVM IR assembly. The API of the Go bindings mimics that of the C bindings, which feels awkward as it bypasses the safety of the type system. A fundamental concept of the API is the notion of a \textit{value}, which represents a computed value that may be used as an operand of other values. This recursive description captures the semantics of instructions and constants. The API provides a unified \texttt{Value} type, which defines 145 methods for interacting with the underlying value types. It is the callers responsibility to only invoke the subset of methods actually implemented by the underlying type. In practise this approach results in fragile applications which may crash during runtime; with assertions such as \textit{``cast<Ty>() argument of incompatible type!''} when invoking the \texttt{Opcode} method instead of the \texttt{InstructionOpcode} method on an instruction value. The former method may only be invoked on constants, which uses a subset of the instruction opcodes for constant expressions. A more sound approach would solve these issues by refining the \texttt{Value} interface be the \textit{intersect} rather than the \textit{union} of all methods implemented on the underlying value types.
