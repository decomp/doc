% --- [ LLVM IR Library ] ------------------------------------------------------

\subsection{LLVM IR Library}
\label{sec:impl_llvm_ir_library}

Early on in the project it was believed that the control flow analysis stage would operate on CFGs that were tightly coupled with an in-memory representation of LLVM IR. This motivated the search for a LLVM IR library with a carefully considered API and set of data structures (objective \ref{itm:obj_ir_library}). While there existed a library which provides Go bindings for LLVM, the API of this library felt awkward to use and was too heavily influenced by the underlying C and C++ libraries; as further described in section \ref{sec:impl_go_bindings_for_llvm}. The interaction with the LLVM IR library would critically influence the design and implementation of the decompilation components. For this reason it was decided that a set of pure Go libraries would be implemented for interacting with LLVM IR, even if it would require a considerable amount of work.

The LLVM IR libraries were intentially developed as reusable components for compilers, decompilers and other semantic analysis tools. To assess the requirements of LLVM-based compilers, a public discussion was held with the developers of the open source \texttt{llgo} compiler, who clarified its specific requirements\footnote{Requirements: \url{https://github.com/llir/llvm/issues/3}}. Fredrik Ehnbom, who is one of the \texttt{llgo} developers, has remained involved with the development of the LLVM IR libraries, by participating in API discussions, conducting code reviews, and submitting patches for performance improvements\footnote{Use binary search for keyword lexing: \url{https://github.com/llir/llvm/pull/11}} (these code changes have not yet been merged, as the project artefacts are required to be developed independently).

The first component to be implemented was the LLVM IR lexer, which tokenizes LLVM IR assembly. In addition to the LLVM IR language specification, the implementation of the reference lexer in LLVM was reviewed to establish a full listing of the valid tokens in LLVM IR. This review uncovered two tokens which were defined but never used in the code base of LLVM. In collaboration with members of the LLVM community, a patch was commited to the LLVM code base which removed these tokens\footnote{Remove unused tokens from AsmParser: \url{http://reviews.llvm.org/D7248}}.

The design of the LLVM IR lexer has been hugely influenced by a talk given by Rob Pike in 2011, titled \textit{``Lexical Scanning in Go''} \cite{lexical_scanning_in_go}. In this talk, Pike introduces the notion of a state function which is a function that returns a state function. The state of the lexer is represented by the active state function, which may transition into another stage by returning the corresponding state function. For instance, when \texttt{lexLineComment} is active, the context of the lexer is known to be a line comment. Any character is valid within line comments, except new lines which terminate the token; at which point an unknown token on the succeeding line is lexed by returning the \texttt{lexToken} state function.

The LLVM IR assembly language requires no separators (e.g. whitespace characters, semicolons) between tokens. This made it very difficult to determine where one token ends and another starts, as further indicated by inconsistent behaviour for separating tokens in the reference implementation of the LLVM lexer. The solution to this issue was inspired by the Go language specification\footnote{The Go Programming Language Specification: \url{https://golang.org/ref/spec}} which states that \textit{``the next token is the longest sequence of characters that form a valid token''}, thus defining a consistent behaviour.

A formal grammar of the LLVM IR language would have facilitate the implementation of the LLVM IR libraries. As no such grammar had been officially endorsed, other sources were cross-referenced to learn about the low-level details of the language, such as its token set and details regarding the type system. This work uncovered a number of potential inconsistencies between the language reference, the implementation, and official blog posts. After discussing these issues with more experienced LLVM developers on the LLVM-dev mailing list, it could be concluded that some issues highlighted inconsistent behaviours while others were working as intended.

The cheer size of LLVM IR was at times discouraging and the project time constrains forced the implementation of subsets within every aspect of the language. LLVM IR may have started out as a simple, minimalistic and platform-independent low-level IR, but this is no longer the case. As time went by and as the project rose in popularity, more and more developers joined the project. In 2014 more than five hundred commits were submitted to the LLVM code base each month. Keeping these changes consistent and the overall system simple is a massive challenge.

It was at times very tempting to switch to REIL instead of LLVM IR, as REIL is a minimal, consistent and clean language. The adaptation of REIL in the open source community was however limited, and there had been no public news announcements since Google acquired the company back in 2011. Furthermore, the REIL language lacks the notion of basic blocks which would complicate the control flow analysis.

After months of development it had become clear that the task of implementing libraries for LLVM IR was way more time consuming than initially anticipated. The project time constrains forced the re-evaluation of using the Go bindings for LLVM, which gave rise to a seemingly small but hugely influential idea. The control flow analysis stage should operate entirely on graph data structures, thus making it unaware of LLVM IR. This idea effectively mitigated the risk of being influenced by the API of the Go bindings for LLVM, and gave rise to the data-driven design of the control flow analysis component, as further described in section \ref{sec:design_control_flow_analysis}. From this point on, the focus shifted to implementing working artefacts which utilized the Go bindings for LLVM IR, as further described in section \ref{sec:impl_go_bindings_for_llvm}.
