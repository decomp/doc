% === [ Introduction ] =========================================================

\section{Introduction}
\label{sec:introduction}

A compiler is a piece of software which translates human readable high-level programming languages (e.g. C) to machine readable low-level languages (e.g. Assembly). In the usual flow of compilation, code is lowered through a set of transformations from a high-level to a low-level representation. The decompilation process (originally referred to as reverse compilation \cite{reverse_comp}) moves in the opposite direction by lifting code from a low-level to a high-level representation.

Decompilation enables source code reconstruction of binary applications and libraries. Both security researchers and software engineers may benefit from decompilation as it facilitates analysis, modification and reconstruction of object code. The applications of decompilation are versatile, and may include one of the following uses:

\begin{itemize}
	\item Analyse malware
	\item Recover source code
	\item Migrate software from legacy platforms or programming languages
	\item Optimise existing binary applications
	\item Discover and mitigate bugs and security vulnerabilities
	\item Verify compiler output with regards to correctness
	\item Analyse proprietary algorithms
	\item Improve interoperability with other software
	\item Add new features to existing software
\end{itemize}

As recognised by Edsger W. Dijkstra in his 1972 ACM Turing Lecture (an extract from which is presented in figure \ref{fig:dijkstra_lecture}) one of the most powerful tools for solving complex problems in Computer Science is the use of abstractions and separation of concerns. This paper explores a compositional approach to decompilation which facilitates abstractions to create a pipeline of self-contained components. Since each component interacts through language-agnostic interfaces (well-defined input and output) they may be written in a variety of programming languages. Furthermore, for each component of the decompilation pipeline there may exist multiple implementations with their respective advantages and limitations. The end user (e.g. malware analyst, security researcher, reverse engineer) may select the components which solves their task most efficiently.

\begin{figure}[htbp]
	\begin{quote}
		\textit{``We all know that the only mental tool by means of which a very finite piece of reasoning can cover a myriad cases is called ``abstraction''; as a result the effective exploitation of their powers of abstraction must be regarded as one of the most vital activities of a competent programmer. In this connection it might be worthwhile to point out that the purpose of abstracting is not to be vague, but to create a new semantic level in which one can be absolutely precise.''} \cite{abstractions_quote}
	\end{quote}
	\caption{An extract from the ACM Turing Lecture given by Edsger W. Dijkstra in 1972.}
	\label{fig:dijkstra_lecture}
\end{figure}

% === [ Subsections ] ==========================================================

% --- [ Project Aim and Objectives ] -------------------------------------------

\subsection{Project Aim and Objectives}
\label{sec:intro_project_aim_and_objectives}

The aim of this project is to facilitate decompilation workflows using composition of language-agnostic decompilation passes; specifically the reconstruction of high-level control structures and, as a future ambition, expressions.

To achieve this aim, the following objectives have been identified:

\begin{enumerate}
	\item Review traditional decompilation techniques, including control flow analysis and data flow analysis.
	\label{itm:obj_review_decomp_techniques}
	\item Critically evaluate a set of Intermediate Representations (IRs), which describes low-, medium- and high-level language semantics, to identify one or more suitable for the decompilation pipeline.
	\label{itm:obj_review_suitable_ir}
	\item Analyse the formal grammar (language specification) of the IR to verify that it is unambiguous. If the grammar is ambiguous or if no formal grammar exists, produce a formal grammar. This objective is critical for language-independence, as the IR works as a bridge between different programming languages.
	\label{itm:obj_formal_ir}
	\item Determine if any existing library for the IR satisfies the requirements; and if not develop one. The requirements would include a suitable in-memory representation, and support for on-disk file storage and arbitrary manipulations (e.g. inject, delete) of the IR.
	\label{itm:obj_ir_library}
	\item Design and develop components which identify the control flow patterns of high-level control structures using control flow analysis of the IR.
	\label{itm:obj_control_flow_analysis_component}
	\item Develop tools which perform one or more decompilation passes on a given IR. The tools will be reusable by other programming language environments as their input and output is specified by a formally defined IR.
	\label{itm:obj_decomp_pass_tool}
	\item As a future ambition, design and develop components which perform expression propagation using data flow analysis of the IR.
	\label{itm:obj_data_analysis_library}
\end{enumerate}

% --- [ Deliverables ] ---------------------------------------------------------

\subsection{Deliverables}

The source code and the report of this project have been released into the public domain\footnote{CC0 1.0 Universal: \url{https://creativecommons.org/publicdomain/zero/1.0/}} and are made available on GitHub.

The following documents have been produced:
\begin{itemize}
	\item Project report; refer to objective \ref{itm:obj_review_decomp_techniques} and \ref{itm:obj_review_suitable_ir} \\ \url{https://github.com/mewpaper/decompilation}
\end{itemize}

And the following system artefacts have been developed:
\begin{itemize}
	\item Library for interacting with LLVM IR (\textit{work in progress}); refer to objective \ref{itm:obj_ir_library} \\ \url{https://github.com/llir/llvm}
	\item Control flow graph generation tool; refer to objective \ref{itm:obj_control_flow_analysis_component} \\ \url{https://github.com/decomp/ll2dot}
	\item Subgraph isomorphism search algorithms and related tools; refer to objective \ref{itm:obj_control_flow_analysis_component} \\ \url{https://github.com/decomp/graphs}
	\item Control flow analysis tool; refer to objective \ref{itm:obj_decomp_pass_tool} \\ \url{https://github.com/decomp/restructure}
	\item Go code generation tool (\textit{proof of concept}); refer to objective \ref{itm:obj_decomp_pass_tool} \\ \url{https://github.com/decomp/ll2go}
	\item Go post-processing tool; refer to objective \ref{itm:obj_decomp_pass_tool} \\ \url{https://github.com/decomp/go-post}
\end{itemize}

% --- [ Disposition ] ----------------------------------------------------------

\subsection{Disposition}

This report details every stage of the project from conceptualisation to successful completion. It follows a logical structure and outlines the major stages in chronological order. A brief summary of each section is presented in the list below.

\begin{itemize}
	\item Section \ref{sec:introduction} - \textbf{Introduction} \\ \textit{Introduces the concept of decompilation and its applications, outlines the project aim and objectives, and summarises its deliverables.}
	\item Section \ref{sec:literature_review} - \textbf{Literature Review} \\ \textit{Details the problem domain, reviews traditional decompilation techniques, and evaluates potential intermediate representations for the decompilation pipeline of the project.}
	\item Section \ref{sec:related_work} - \textbf{Related Work} \\ \textit{Evaluates projects for translating native code to LLVM IR, and reviews the design of modern decompilers.}
	\item Section \ref{sec:methodology} - \textbf{Methodology} \\ \textit{Surveys methodologies and best practices for software construction, and relates them to the specific problem domain.}
	\item Section \ref{sec:requirements} - \textbf{Requirements} \\ \textit{Specifies and prioritises the requirements of the project artefacts.}
	\item Section \ref{sec:design} - \textbf{Design} \\ \textit{Discusses the system architecture and the design of each component, motivates the choice of core algorithms and data structures, and highlights strengths and limitations of the design.}
	\item Section \ref{sec:implementation} - \textbf{Implementation} \\ \textit{Discusses language considerations, describes the implementation process, and showcases how set-backs were dealt with.}
	\item Section \ref{sec:verification} - \textbf{Verification} \\ \textit{Describes the approaches taken to validate the correctness, performance and security of the artefacts.}
\clearpage % <layout>
	\item Section \ref{sec:evaluation} - \textbf{Evaluation} \\ \textit{Assesses the outcome of the project and evaluates the artefacts against the requirements.}
	\item Section \ref{sec:conclusion} - \textbf{Conclusion} \\ \textit{Summarises the project outcomes, presents ideas for future work, reflects on personal development, and concludes with an attribution to the key idea of this project.}
\end{itemize}

