% === [ Design ] ===============================================================

\section{Design}

% TODO: Reformulate to fit the Design section.

\textbf{NOTE}: \textit{The following paragraph has been moved here from a previous version of the abstract and will therefore feel out of place. The intention is to reformulate it.}

A dynamic approach to problem solving is the composition of independent and specialized components which communicate using well-defined interfaces. Several smaller components may conceptually be arranged in a pipeline of stages which transform, massage or interpret the input in a certain way to solve larger tasks. A well composed pipeline is capable of solving more complex problems than each of its components, problems which may not even have been envisioned by the original component authors. This idea is embodied in the Unix philosophy and it has influenced software construction profoundly. Systems which expose their individual components to end-users facilitate dynamic workflows, as they enable users to adapt and extend each part of the system by adding, removing, replacing or refining components in one or more stages of the pipeline. Meanwhile, the monolithic nature of most production-quality decompilers prevents reverse engineers from utilizing such workflows and may leave them with scripting and plugin support as a substitute.

% TODO: Add design notes
%    - The LLVM IR libraries are developed as reusable components for compilers, decompilers, and other semantic analysis tools. They aim to support generic semantic analysis applications, while satisfying the explicit requirements [1] of the third party llgo compiler.
%
% [1]: https://github.com/go-llvm/llvm/issues/40

% TODO: Add
%   - A decompilation system composed of individual components and based on the principle of separation of concerns.
%   - The system must be language-agnostic so that decompilation passes can be reused from other programming language environments.

foo

% --- [ Programming Language Considerations ] ----------------------------------

\subsection{Programming Language Considerations}

% TODO: Clarify the benefits and drawbacks of using Go over C++ which would be the obvious choice for LLVM IR heavy projects. Choosing not to use C++ validates the language-agnostic aspects of the design.
% - Compilation speed. ll2dot takes > 1.5m whereas a regular Go program takes < 1s.

% TODO: Mention software composition.

% TODO: Add and tie in to the pragmatic aspects of Go?
%    * tooling?
%       - "go get" can locate all dependencies.
%       - compilation time is linear rather than exponential with regards to dependencies.
%    * automation?

foo

% --- [ Idiomatic Coding ] -----------------------------------------------------

\subsection{Idiomatic Coding}

% TODO: Adapt the "Ideomatic Coding" section to fit the Design section. Choice of section title? Merge with the Programming Language Considerations section? Probably.

\textbf{NOTE}: \textit{Still not sure where this section fits. It has been moved here from the Implementation section, but it needs to be reworked somehow. Perhaps it can be merged with the Programming Language Considerations section and adapted to fit better.}

Making effective use of a programming language requires more than simply learning its syntax and key features. With a reasonable understanding of the underlying design decisions behind a programming language and the historic factors which drove its development one may infer the governing principles and key beliefs of its developers. These principles and beliefs influence every aspect of the software development process; they determine how programs are structured and how problems are solved. One of the primary driving forces behind the development of the Go programming language is pragmatism, the idea of which was captured in the following quote by Samuel Tesla shortly after Go was released in 2009:

\begin{quote}
	\textit{``Go is not meant to innovate programming theory. It's meant to innovate programming practice.''} - Samuel Tesla, Dec 2009 \cite{pragmatic}
\end{quote}

In 2012 Rob Pike (one of the Go language inventors) gave a talk titled \textit{``Less is exponentially more''} which included a personal description of the historic events leading up to the inception of Go. The starting point of the language was C, not C++, which Go aimed to simplify further by removing cruft. This stands in direct contrast to the direction of C++ which gains more features with each passing release. The \textit{less is more} mindset is deeply rooted in the mentality of Go developers and there is a strong emphasis on the use of composition to solve problems.

\begin{quote}
	\itshape
	``If C++ and Java are about type hierarchies and the taxonomy of types, Go is about composition.

	Doug McIlroy, the eventual inventor of Unix pipes, wrote in 1964 (!):

	\begin{quote}
		We should have some ways of coupling programs like garden hose--screw in another segment when it becomes necessary to massage data in another way. This is the way of IO also.
	\end{quote}

	That is the way of Go also. Go takes that idea and pushes it very far. It is a language of composition and coupling.''
	\normalfont
	- Rob Pike, 2012 \cite{less_is_more}
\end{quote}

Every aspect of Go development embodies the UNIX philosophy (see \cref{fig:unix_philosophy}) which comes as no surprise since Ken Thompson (one of the original inventors of UNIX) is part of the core Go team.

\begin{figure}[htbp]
	\begin{center}
		\begin{quote}
			\textit{``Write programs that do one thing and do it well. Write programs to work together.''} \cite{art_of_unix}
		\end{quote}
		\caption{The UNIX philosophy.}
		\label{fig:unix_philosophy}
	\end{center}
\end{figure}

% --- [ Decompiler Pipeline ] --------------------------------------------------

\subsection{Decompiler Pipeline}

% TODO: Describe where the "restructure" component fits in the overall decompilation pipeline. Mention which projects and tools that may be used to fill the gaps. bin_descend and IDA python script of MC-Semantics -> Google Protocol Buffer -> cfg_to_bc -> LLVM IR

% TODO: Rewrite and clarify.

\textbf{NOTE}: \textit{The following paragraph is more of a brain-dump. It is intended to be used as a basis for a future rewrite.}

The decompilation pipeline is made of up several components which are conceptually grouped into three modules. The front-end module translates a variety of inputs (such as binary files and source code) into LLVM IR by utilizing a collection of tools developed by several independent open source projects. The middle-end lifts the LLVM IR to a high-level representation by conducting a control flow analysis which generates a structured CFG of each function. The back-end generates high-level control flow primitives such as if-statements and for-loops based on the structured CFG. In addition it translates the individual instructions of the LLVM IR to expressions and statements of the target programming language (in this case Go). The interaction between the front-end, middle-end and back-end modules is visualized in \cref{fig:decompilation_pipeline}.

% * Front-end
%    - binary -> LLVM IR ([MC-Semantics](https://github.com/trailofbits/mcsema), [Dagger](http://dagger.repzret.org/) or [Fracture](https://github.com/draperlaboratory/fracture))
%    - source code -> LLVM IR (clang, ghc, rustc, ...)
% * Middle-end
%    - LLVM IR -> Unstructured CFG ([ll2dot](https://github.com/mewrev/ll2dot))
%    - Unstructured CFG -> Structured CFG ([iso](https://github.com/mewrev/graphs) and [merge](https://github.com/mewrev/graphs).)
%       + Truthfully `ll2go` doesn't make direct use of `iso` or `merge` but rather the graph libraries. A future ambition is to allow `iso` and `merge` to output JSON such that they may be used directly by `ll2go`.
% * Back-end
%    - Structured CFG -> Go ([ll2go](https://github.com/mewrev/ll2go))


\begin{figure}[htbp]
	\begin{center}
		\includegraphics[width=\textwidth]{inc/decompilation_pipeline.png}
		\caption{foo}
		\label{fig:decompilation_pipeline}
	\end{center}
\end{figure}


\subsubsection{Front-end}

% TODO: Rewrite and clarify.

\textbf{NOTE}: \textit{The following paragraph is more of a brain-dump. It is intended to be used as a basis for a future rewrite.}

The front-end module is responsible for converting the input into LLVM IR. Two common scenarios exists, converting binary files (e.g. executables, shared libraries and relocatable object code) and converting source code (e.g. C, Haskell, Rust, …) into LLVM IR. The first scenario is presented in \cref{fig:front-end_binary} and the second in \cref{fig:front-end_source}.

% TODO: Mention opt --mem2reg.

\begin{figure}[htbp]
	\begin{center}
		\includegraphics[width=\textwidth]{inc/front-end_binary.png}
		\caption{foo}
		\label{fig:front-end_binary}
	\end{center}
\end{figure}

\begin{figure}[htbp]
	\begin{center}
		\includegraphics[width=\textwidth]{inc/front-end_source.png}
		\caption{foo}
		\label{fig:front-end_source}
	\end{center}
\end{figure}

\subsubsection{Middle-end}

% TODO: Rewrite and clarify.

% TODO: Create a restructure tool which replaces the use of iso.

\textbf{NOTE}: \textit{The following paragraph is more of a brain-dump. It is intended to be used as a basis for a future rewrite.}

The middle-end is responsible for lifting the LLVM IR to a high-level representation through a series of decompilation passes. The \texttt{ll2dot} tool generates a CFG (in the DOT file format) for each function of a given LLVM IR input file. The \texttt{iso} tool searches for subgraph isomorphisms of control flow primitives in a given CFG. Once located the nodes identified subgraph are merged into a single node which is labeled with the high-level control flow primitive. Successive iterations continue to simplify the CFG until only one node is left, at which point the high-level control flow structure has been recovered. Should the \texttt{iso} tool fail to reduce the graph into a single node, the graph is considered irreducible with regards to the supported high-level control flow primitives. The interaction between the front-end, the \texttt{ll2dot} and \texttt{iso} tools of the middle-end and the back-end is illustrated in \cref{fig:middle-end}.

\begin{figure}[htbp]
	\begin{center}
		\includegraphics[width=\textwidth]{inc/middle-end.png}
		\caption{foo}
		\label{fig:middle-end}
	\end{center}
\end{figure}

\subsubsection{Back-end}

% TODO: Rewrite and clarify.

% TODO: Add ref to rsc's grind tool.

% TODO: Proof-of-concept. Implement a back-end for another language and written in another language. This would stress test the language-agnostic aspects of the design, thus making sure that the heavy-lifting is done in the middle-end and not in ll2go.

\textbf{NOTE}: \textit{The following paragraph is more of a brain-dump. It is intended to be used as a basis for a future rewrite.}

The back-end is responsible for translating the structured control flow graph of the LLVM IR into a target programming language. The \texttt{ll2go} tool is a proof of concept back-end which produces unpolished Go source code. The polishing is done by separate tools which fixes potential compilation issues and makes the code more idiomatic. The interaction between the middle-end and the back-end is illustrated in \cref{fig:back-end}. Currently the \texttt{ll2gofix} replaces return-statements in the \texttt{main} function with calls to \texttt{os.Exit}, which is required since the \texttt{main} function has no return arguments in Go. Instead the Go runtime calls \texttt{os.Exit} with the status-code \texttt{0} once \texttt{main} returns to signal successful termination. This eliminates the need to always end the \texttt{main} function with a \texttt{return 0;} statement as is common practise in C. A future ambition is to make use of and possibly contribute to the \texttt{grind} tool which moves variable declarations closer to their usage, and thus improving readability of the code. Generally the aim is to keep the \texttt{ll2go} tool as simple as possible. The middle-end is responsible for the structural analysis, and as a future ambition the data flow analysis. Since the complexity of the back-end is kept to a minimum it should be trivial to implement support for other output languages.

\begin{figure}[htbp]
	\begin{center}
		\includegraphics[width=\textwidth]{inc/back-end.png}
		\caption{foo}
		\label{fig:back-end}
	\end{center}
\end{figure}

% --- [ System Architecture ] --------------------------------------------------

\subsection{System Architecture}

% TODO: Visualize the dependency graph of the "restructure" tool and describe in detail what input it expects and what output it produces.

% TODO: Write about. Input and output LLVM IR to operate well with components written in other languages. Output LLVM IR with information about high-level control structures stored in the basic block names or in metadata.

foo

% TODO: Mention package division.

\subsection{Data-driven Design}

foo

% TODO: Mention: CFG invariants (e.g. single-entry, single-exit)

% TODO: Add notes about the use of DOT-files to describe control flow primitives. Think if and how this could be pushed further to facilitate the development of future back-ends.

\subsection{Limitations}

% TODO: Add limitations related to the design choices. Which limitations are easily solvable given more time and which are fundamentally part of the design.
%    - No support for n-way conditionals (e.g.switch-statements).

foo
