% === [ Design ] ===============================================================

% <howto>
% * Justify, evaluate, recognize limitations of.
%
% * Analytical writing
%    - Why did you do X that way?
%    - Why did you do Y but not Z?
%    - What was important and what not?

% <howto>
% * If you were to implement the system in another language, which aspects of
%   the design would remain?
% * Which are the guiding design principles?
% * Describe the general system architecture; which components interact, how,
%   and why?
% * How are the individual components designed? (once again, which aspects
%   remain if you implemented them in another language?)

\begin{quote}
	\textit{``The whole is more than the sum of its parts.''} - Anonymous
\end{quote}

\section{Design}
\label{sec:design}

The principle of separation of concern has had a core influence on the design of the decompilation system. It has motivated a system architecture based on the composition of independent and self-contained components. End-users may either use the individual component in separation, or combine a set of components into a custom decompilation pipeline.

Several smaller components may conceptually be arranged in a pipeline of stages which transform, massage or interpret the input in a certain way to solve larger tasks. A well composed pipeline is capable of solving more complex problems than each of its components, problems which may not even have been envisioned by the original component authors~\cite{simplicity_and_collaboration}. This idea is embodied in the Unix philosophy and it has influenced software construction profoundly~\cite{art_of_unix}. Furthermore, systems which expose their individual components to end-users facilitate dynamic workflows, as they enable users to adapt and extend each part of the system by adding, removing, replacing or refining components in one or more stages of the pipeline.

To enforce a strict separation of concerns, each component is given access to the least amount of information required to successfully accomplish its task (e.g. the control flow analysis stage operates on CFGs and is unaware of the underlying code).

The design of the decompilation system must allow language-agnostic interaction between components written in different programming languages (refer to the aim of the project in section~\ref{sec:intro_project_aim_and_objectives}). This requirement has been satisfied by communicating through well-defined input and output (e.g. JSON, DOT, LLVM IR). A more detailed view of the system architecture is presented in section~\ref{sec:design_system_architecture}.

% === [ Subsections ] ==========================================================

% --- [ System Architecture ] --------------------------------------------------

% <howto>
% * the overall structure of the software system (architecture)

% <howto>
% * Software architecture is concerned with deciding what has to be done, and which program component is going to do it (how something is done is left to the detailed design phase, below)
% * It effectively defines the interface between the programs of the system.
% * This stage does not need to consider non-functional requirements (e.g. response time, reliability, maintainability).

\subsection{System Architecture}
\label{sec:design_system_architecture}

The decompilation pipeline conceptually consists of three modules which separate the general decompilation tasks (e.g. control flow analysis) from concerns related to the source language and the target language. Firstly, the front-end translates a variety of source languages (e.g. x86 or ARM assembly, C or Haskell source code, …) to LLVM IR by utilizing several independent open source projects. Secondly, the middle-end structures the LLVM IR by identifying high-level control flow primitives in the CFGs generated from the intermediate representation. Lastly, the back-end translates the structured LLVM IR into a high-level target programming language (e.g. Go). The interaction between these modules is visualized in figure \ref{fig:decompilation_pipeline}, and the individual components of the front-end, middle-end and back-end modules are further described in section \ref{sec:design_front-end_components}, \ref{sec:design_middle-end_components} and \ref{sec:design_back-end_components} respectively.

\begin{figure}[htbp]
	\begin{center}
		\includegraphics[width=\textwidth]{inc/6_design/decompilation_pipeline.png}
		\caption{The front-end of the decompilation pipeline translates a variety of inputs (e.g. native code or source code) to LLVM IR; the middle-end structures the LLVM IR through control flow analysis; and the back-end translates the structured LLVM IR to a high-level programming language (e.g. Go).}
		\label{fig:decompilation_pipeline}
	\end{center}
\end{figure}

The main benefit with this decompiler architecture is that it scales well when implementing support for additional source languages (e.g. MIPS or PowerPC assembly) and target languages (e.g. Python), as the general decompilation tasks only have to be implemented once. The decompiler architecture is an adaptation of the one presented by C. Cifuentes back in 1994 (as described in section \ref{sec:lit_review_decompilation_phases}), which was heavily inspired by the architecture of compilers that separated general optimization tasks (e.g. constant propagation) from concerns related to the source programming language (e.g. C) and the target computer architecture (e.g. x86). The compiler architecture has been proven so effective at separating concerns that it remains in use today by several production-quality compilers \cite{llvm_architecture,gcc_architecture}.

% --- [ Front-end Components ] -------------------------------------------------

\subsection{Front-end Components}
\label{sec:front-end_components}

The front-end module is responsible for converting the input into LLVM IR. Two common scenarios exists, converting binary files (e.g. executables, shared libraries and relocatable object code) and converting source code (e.g. C, Haskell, Rust, …) into LLVM IR. The first scenario is presented in section \ref{sec:design_native_code_to_llvm_ir} and the second in section \ref{sec:compilers}.

% --- [ Subsubsections ] -------------------------------------------------------

% ~~~ [ Native Code to LLVM IR ] ~~~~~~~~~~~~~~~~~~~~~~~~~~~~~~~~~~~~~~~~~~~~~~~

\subsubsection{Native Code to LLVM IR}

%    - binary -> LLVM IR ([MC-Semantics](https://github.com/trailofbits/mcsema), [Dagger](http://dagger.repzret.org/) or [Fracture](https://github.com/draperlaboratory/fracture))

foo

% ~~~ [ Compilers ] ~~~~~~~~~~~~~~~~~~~~~~~~~~~~~~~~~~~~~~~~~~~~~~~~~~~~~~~~~~~~

\subsubsection{Compilers}
\label{sec:compilers}

One important aspect of utilizing the IR of a compiler framework, is that the decompilation pipeline automatically gains support for transpilation (i.e. translating one programming language into another) in addition to reverse compilation. An increasing number of open source compilers (e.g. Clang, GHC, \texttt{rustc}) are capable of translating a range of source languages (e.g. C, Haskell, Rust) into LLVM IR. These compilers may be used as-is by the front-end module (see figure \ref{fig:front-end_source}), thereby extending the supported source languages of the decompilation pipeline. Using this approach, the decompilation pipeline may translate $ n $ source languages into $ m $ target languages by implementing $ n + m $ front-end and back-end modules, instead of $ n \cdot m $ transpilers.

\begin{figure}[htbp]
	\begin{center}
		\includegraphics[width=\textwidth]{inc/front-end_source.png}
		\caption{Several open source compilers translate high-level programming languages to LLVM IR. Three such compilers are Clang, the Glasgow Haskell Compiler and the Rust compiler which translate C, Haskell and Rust respectively to LLVM IR.}
		\label{fig:front-end_source}
	\end{center}
\end{figure}

Another important aspect of utilizing LLVM IR, is that a wide range of optimizations have been implemented already by the LLVM compiler framework. This allows the front-end components to focus on translating the source languages into LLVM IR, without having to worry about producing highly optimized output. The LLVM IR may later be optimized by invoking the \texttt{opt} tool of LLVM to remove dead code, propagate constants, and promote memory accesses to registers.


% --- [ Middle-end Components ] ------------------------------------------------

% <howto>
% * more detailed design of individual components (design)

% <howto>
% * The intention is that the design should be detailed enough to provide a good guide for actual coding, including details of any particular algorithms to be used.

\subsection{Middle-end Components}
\label{sec:middle-end_components}

The middle-end module is responsible for lifting the low-level IR generated by the front-end to a higher level. This is achieved through a set of decompilation stages, which identify high-level control flow primitives and, as a future ambition, propagate expressions. The former decompilation stage consists of two self-contained components which separate concerns related to the control flow analysis from the underlying details of LLVM IR. The first component generates unstructured CFGs from LLVM IR, as further described in section \ref{sec:design_control_flow_graph_generation}. And the second component structures the generated CFGs by identifying high-level control flow primitives, as further described in section \ref{sec:design_control_flow_analysis}. The interaction between the front-end, the \texttt{ll2dot} and \texttt{restructure} tools of the middle-end and the back-end is illustrated in figure \ref{fig:middle-end}.

\begin{figure}[htbp]
	\begin{center}
		\includegraphics[width=\textwidth]{inc/middle-end.png}
		\caption{The middle-end module performs a control flow analysis on the LLVM IR in two steps. Firstly, the \texttt{ll2dot} tool generates unstructured CFGs (in the DOT file format) from LLVM IR. Secondly, the \texttt{restructure} tool produces a structured CFG (in JSON format) by identifying high-level control flow primitives in the unstructured CFG.}
		\label{fig:middle-end}
	\end{center}
\end{figure}

% --- [ Subsubsections ] -------------------------------------------------------

% ~~~ [ Control Flow Graph Generation ] ~~~~~~~~~~~~~~~~~~~~~~~~~~~~~~~~~~~~~~~~

\subsubsection{Control Flow Graph Generation}
\label{sec:design_control_flow_graph_generation}

%    - LLVM IR -> Unstructured CFG ([ll2dot](http://decomp.org/x/cmd/ll2dot))

foo

% ~~~ [ Control Flow Analysis ] ~~~~~~~~~~~~~~~~~~~~~~~~~~~~~~~~~~~~~~~~~~~~~~~~

\subsubsection{Control Flow Analysis}
\label{sec:design_control_flow_analysis}

The key idea behind the control flow analysis (see section \ref{sec:control_flow_analysis}), is that high-level control flow primitives may be represented using directed graphs. The problem of structuring low-level code may therefore be rephrased as the problem of identifying subgraphs (e.g. the graph representation of high-level control flow primitives) in graphs (e.g. the CFGs of low-level code) without considering node names, as illustrated in figure \ref{fig:representation_and_identification_of_primitive}. This problem is generally referred to as \textit{subgraph isomorphism search} and has been well studied \cite{subgraph_isomorphism_algorithms}. Rephrasing the problem in this manner aligns with the design principle of giving each component access to the least amount of information required to successfully accomplish its task. The control flow analysis component is only given access to control flow information (e.g. CFGs), and is oblivious of the underlying LLVM IR. This enables the component to be reused as-is when analyzing the control flow of other languages, such as REIL.

\begin{figure}[htbp]
	\centering
	\begin{subfigure}[ht]{0.10\textwidth}
		\lstinputlisting[language=go, style=go, breaklines=false, numbers=none]{poster/inc/if.c}
		\includegraphics[width=\textwidth]{poster/inc/if.png}
	\end{subfigure}
	\enskip
	\begin{subfigure}[ht]{0.18\textwidth}
		\includegraphics[width=\textwidth]{poster/inc/foo.png}
	\end{subfigure}
	\caption{The left side contains the pseudo-code (top left) and graph representation (bottom left) of an if-statement; if \texttt{A} is true then do \texttt{B} followed by \texttt{C}, otherwise do \texttt{C}. The right side highlights (in red) an identified isomorphism of the if-statement's graph representation, in the CFG of the \texttt{main} function presented in appendix \ref{app:clang_example}.}
	\label{fig:representation_and_identification_of_primitive}
\end{figure}

The \texttt{restructure} tool uses subgraph isomorphism search algorithms to locate isomorphisms of the graph representations of high-level control flow primitives in the CFG of a given function. The CFG is simplified by recursively replacing the identified subgraphs with single nodes until the entire CFG has been reduced into a single node; a step-by-step demonstration of which is presented in appendix \ref{app:control_flow_analysis_example}. By recoding the node names of the identified subgraph isomorphisms and the name of their corresponding high-level control flow primitives, a structured CFG may be produced in which all nodes are known to belong to a high-level control flow primitive; as demonstrated in appendix \ref{app:restructure_example}.

The pseudo-code and graph representations of the supported high-level control flow primitives are presented in figure \ref{fig:graph_representations} of section \ref{sec:control_flow_analysis}. Should the control flow analysis fail to reduce a CFG into a single node, the CFG is considered irreducible with regards to the supported high-level control flow primitives, in which case a structured CFG cannot be produced.

The \texttt{restructure} tool relies entirely on subgraph isomorphism search to produce structured CFGs (in JSON format) from unstructured CFGs (in the DOT file format). The supported high-level control flow primitives are defined using DOT files, thus promoting a data-driven design which separates data regarding the primitives from the implementation of the \texttt{restructure} tool. A major benefit with this approach is that the \texttt{restructure} tool may -- without modification -- search for any high-level control flow primitive that can be expressed in the DOT file format.

One limitation with this approach is that it does not support graph representations of high-level control flow primitives with a variable number of nodes, as they cannot be described in the DOT file format. For this reason, the \texttt{restructure} tool does not support the recovery of n-way conditionals (e.g. \texttt{switch}-statements). Furthermore, the current design enforces a single-entry/single-exit invariant on the graph representation of high-level control flow primitives. This prevents the recovery of infinite loops, as their graph representation has no exit node. Section \ref{sec:design_validation} discusses how these issues may be mitigated in the future.


\input{sections/6_design/4_back-end_components/0_back-end_components}
% ~~~ [ Code Generation ] ~~~~~~~~~~~~~~~~~~~~~~~~~~~~~~~~~~~~~~~~~~~~~~~~~~~~~~

\subsubsection{Code Generation}
\label{sec:design_code_generation}

%    - Structured CFG -> Go ([ll2go](http://decomp.org/x/cmd/ll2go))
%       + Truthfully `ll2go` does not make direct use of `restructure` but rather the graph libraries.

The code generation component translates structured LLVM IR into unpolished Go code. It creates and populates an abstract syntax tree (AST)

% Generates an abstract syntax tree (AST)

foo

% ~~~ [ Post-processsing ] ~~~~~~~~~~~~~~~~~~~~~~~~~~~~~~~~~~~~~~~~~~~~~~~~~~~~~

\subsubsection{Post-processing}

%    - Unpolished Go -> Go ([go-post](http://decomp.org/x/cmd/go-post))

foo

The polishing is done by separate tools which fixes potential compilation issues and makes the code more idiomatic.

Currently the \texttt{go-post} replaces return-statements in the \texttt{main} function with calls to \texttt{os.Exit}, which is required since the \texttt{main} function has no return arguments in Go. Instead the Go runtime calls \texttt{os.Exit} with the status-code \texttt{0} once \texttt{main} returns to signal successful termination. This eliminates the need to always end the \texttt{main} function with a \texttt{return 0;} statement as is common practise in C. A future ambition is to make use of and possibly contribute to the \texttt{grind} tool which moves variable declarations closer to their usage, and thus improving readability of the code.


