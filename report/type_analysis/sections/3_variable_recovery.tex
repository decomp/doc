% === [ Variable Recovery ] ====================================================

\section{Variable Recovery}

Variable and typing information is lost during the process of compilation, as local and global variables, function arguments and function parameters are lowered from source code to machine code and mapped onto type-less registers and memory locations. Debug information of binary executables may record this mapping. When debug information is limited or absent however, the source variables have to be recovered from low-level code using variable recovery methods. Since type analysis is based on inference between type typing relations of variables, variable recovery is required for type recovery.

To explain different aspects of variable recovery and type analysis, two running examples are used throughout this paper. The first example (see figure \ref{fig:running_example} of appendix \ref{app:running_example}) details a function \texttt{f} which keeps track of local variables using a base pointer (\texttt{ebp}). The second example (see figure \ref{fig:struct_example} of appendix \ref{app:struct_example}) details a function \texttt{g} which operates on an array of structures, and a function \texttt{h} which operates on a structure pointer. Subsequent in text references to \texttt{f}, \texttt{g} and \texttt{h} correspond to these three functions.

% === [ Subsections ] ==========================================================

% --- [ Value Set Analysis ] ---------------------------------------------------

% Results: objective communication of data

\subsection{Value Set Analysis}

\todo{TODO: describe value set analysis}

upper bound, lower bound and stride for memory locations. \cite{wysinwyx}

% --- [ Function Signature Recovery ] ------------------------------------------

\subsection{Function Signature Recovery}

\todo{TODO: add note about call conventions}

% TODO: rephrase.

The following set of steps are used for function signature recovery (function parameters and returns arguments) in SecondWrite \cite{second_write_scalable_type_detection}.

\begin{enumerate}
	\item assume all registers are arguments and no register are return arguments.
	\item registers written to are potential return arguments.
	\item callee saved registers (through push in function prologue and pop in function epilogue) (\textit{DeadStores}) are pruned from potential return set.
	\item prune arguments not actually used (e.g. not part of \textit{DeadStore} or PHI instructions).
	\item prune return registers not actually used by callers.
\end{enumerate}

