% === [ Type Recovery ] ========================================================

\section{Type Recovery}

% Key: many problems in binary analysis can be seen as subproblems of type analysis.

% binary analysis,

% floating-point stack
%    - track stack top (indirect calls/external calls)
% pointer analysis
%    - global memory region
%    - stack memory region
%    - heap memory region
% structures/arrays

\todo{TODO: give introduction to type recovery as part of type analysis of low-level code.} \cite{mycroft_type_based_decompilation}

\paragraph{Value-based type inference} \todo{TODO: describe value-based type inference. E.g. if value is \texttt{42}, infer that it is an integer, if value is \texttt{0x401005} infer that it is a pointer. Bring up problems with this approach, misclassification.} \cite{type_inference_on_executables}

\paragraph{Flow-based type inference}

\todo{TODO: describe type inference, Algorithm W} \cite{milner_algorithmw}

\paragraph{Type propagation}

\todo{TODO: describe type propagation, type constraints, forward and backward propagation, constraint solving}

\paragraph{Unification}

\todo{TODO: describe the unification step, as many algorithms rely on this to discover structure types, etc.}

\paragraph{Instruction type sources}

Also known as type revealing instruction.

\todo{TODO: describe type revealing instructions. Infer from the instruction that the operand is of integer type, memory, floating-point, etc.}

\paragraph{Function type sources} Type information derived for arguments in calls to functions with known functions signatures (e.g. those of the standard library). \cite{type_inference_on_executables}

\paragraph{Type sink} A type sink is a type which can be resolved directly and is known to be correct. For instance, types for arguments inferred from function type sources.

\paragraph{Pointer analysis}

\todo{TODO: give brief introduction to pointer analysis, points-to sets, pointer aliasing, and the technique to merge call sites such that memory allocated at different allocation call sites (e.g. calls to \texttt{malloc}) may have the same type if proven that through type propagation, that the returned pointers are used at other parts of the program for the same variables.}
