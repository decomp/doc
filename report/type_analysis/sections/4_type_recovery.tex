% === [ Type Recovery ] ========================================================

\section{Type Recovery}

% Key: many problems in binary analysis can be seen as subproblems of type analysis.

% binary analysis,

% floating-point stack
%    - track stack top (indirect calls/external calls)
% pointer analysis
%    - global memory region
%    - stack memory region
%    - heap memory region
% structures/arrays

The key concepts used in type recovery are \textit{type inference}, \textit{type propagation} and \textit{unification} \cite{mycroft_type_based_decompilation}; and their shared representation is \textit{type constraints} (e.g. upper and lower bounds in a primitive type lattice) associated with the variables determined through variable recovery (see section \ref{sec:variable_recovery}).

\begin{itemize}
	\item Type inference infers type constraints of variables (based on the values they may contain, their use in instructions and as function arguments).
	\item Type propagation propagates the type constraints between variables which interact (based on data flow analysis).
	\item Unification unifies the type constraints of the entire program, and may be seen as calculating the set of solutions (which may be empty) to the type constraint equations.
\end{itemize}

% === [ Subsections ] ==========================================================

% --- [ Type Inference ] -------------------------------------------------------

\subsection{Type Inference}

% ~~~ [ Value-based type inference ] ~~~~~~~~~~~~~~~~~~~~~~~~~~~~~~~~~~~~~~~~~~~

\paragraph{Value-based type inference}

The type of a variable may be inferred from the values stored in its register (e.g. 42) or memory location. For instance, the type of the global variable \texttt{unk\_8000040} (see line 32 in figure \ref{fig:local_variable_example_asm}) may be inferred from the sequence of bytes stored at its memory location; given that the characters of the byte sequence (\texttt{'foo', 0}) are within ASCII range (0-127) and ends with a zero, the source type may be a NULL-terminated string.

One issue with value-based type inference is that it is difficult to distinguish pointer values from integer values or arbitrary sequences of bytes (e.g. \texttt{0x414243} could either refer to an address, the integer value \texttt{4276803} or the byte sequence \texttt{ABC}). Therefore, value-based type inference may lead to misclassification of types \cite{type_inference_on_executables}.

One benefit of value-based type inference is that it only requires examining the contents of registers and memory. Given that no extensive code analysis is required, this approach fits dynamic analysis well.

% ~~~ [ Instruction type sources ] ~~~~~~~~~~~~~~~~~~~~~~~~~~~~~~~~~~~~~~~~~~~~~

\paragraph{Instruction type sources}

Also known as \textit{type revealing instruction}.

The type of a variable may be inferred from the operational semantics (e.g. assignment, comparison, arithmetic) of instructions.

For instance, a variable assigned the value \texttt{321} may be classified as a signed or unsigned integer of bit size at least 16 (since the value \texttt{321} does not fit in 8 bits). More concretely, the \texttt{mov} instruction at \texttt{f:14} (see figure \ref{fig:local_variable_example_asm}) constrains the local variable at \texttt{ebp-4} to either \textit{num16} or \textit{num32} in the primitive type lattice on a 32-bit system (see figure \ref{fig:base_type_lattice}).

% (e.g. assignment, comparison, arithmetic)

\todo{TODO: describe type revealing instructions. Infer from the instruction that the operand is of integer type, memory, floating-point, etc.}

% ~~~ [ Function type sources ] ~~~~~~~~~~~~~~~~~~~~~~~~~~~~~~~~~~~~~~~~~~~~~~~~

\paragraph{Function type sources}

Type information derived for arguments in calls to functions with known functions signatures (e.g. those of the standard library) \cite{type_inference_on_executables}.

% ~~~ [ Type sink ] ~~~~~~~~~~~~~~~~~~~~~~~~~~~~~~~~~~~~~~~~~~~~~~~~~~~~~~~~~~~~

\paragraph{Type sink}

A type sink is a type which can be resolved directly and is known to be correct. For instance, the types of arguments inferred from function type sources.

% ~~~ [ Format-string based type inference ] ~~~~~~~~~~~~~~~~~~~~~~~~~~~~~~~~~~~

\paragraph{Format-string based type inference}

\todo{TODO: describe format-string based type inference.}

Type of arguments to \texttt{printf} call may be inferred by format-string verbs (e.g. \texttt{\%d}).

\todo{TODO: use example of eax register in NASM example above to illustrate the need to track life ranges}.

% --- [ Type Propagation ] -----------------------------------------------------

\subsection{Type Propagation}

\todo{TODO: describe type propagation, type constraints, forward and backward propagation, constraint solving}

\todo{TODO: describe type inference, Algorithm W} \cite{milner_algorithmw}

% --- [ Unification ] ----------------------------------------------------------

\subsection{Unification}

\todo{TODO: describe the unification step, as many algorithms rely on this to discover structure types, etc.}


% TODO: skip pointer analysis?
%% --- [ Pointer analysis ] -----------------------------------------------------

\subsection{Pointer analysis}

\todo{TODO: give brief introduction to pointer analysis, points-to sets, pointer aliasing, and the technique to merge call sites such that memory allocated at different allocation call sites (e.g. calls to \texttt{malloc}) may have the same type if proven that through type propagation, that the returned pointers are used at other parts of the program for the same variables.}

