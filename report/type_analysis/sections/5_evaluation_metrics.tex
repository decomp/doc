% === [ Evaluation Metrics ] ===================================================

\section{Evaluation Metrics}

\todo{TODO: add note on the use of debug information.}

\todo{TODO: mention that SPEC2006 and Coreutils are often used in benchmarks.}

To enable objective comparison of different type recovery methods, a shared set of evaluation metrics are required. For this purpose, TIE proposed two evaluation metrics: \textit{distance} and \textit{conservativeness} \cite{tie_reverse_engineering_of_types}.

% --- [ Distance ] -------------------------------------------------------------

\subsection{Distance}

The distance in height between the recovered and the source type in the primitive type lattice (see figure \ref{fig:base_type_lattice}) for subtypes (e.g. \texttt{int32} is a subtype of \texttt{num32} at distance 1), and the maximum lattice height otherwise.

% --- [ Conservativeness ] -----------------------------------------------------

\subsection{Conservativeness}

Reports whether the source type exists within the type range (lower and upper bound) of the recovered type.

The distance measurement as defined by TIE did not distinguish between multi-level pointers (e.g. \texttt{int*} and \texttt{int**} are equivalent, i.e. on the same level in the type lattice). Thus SecondWrite proposed a refinement to measure the ratio of between the recovered pointer level and the source pointer level \cite{second_write_scalable_type_detection}.

Other metrics have been proposed to compare class hierarchies.
