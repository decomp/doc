% --- [ Type Inference ] -------------------------------------------------------

\subsection{Type Inference}

% ~~~ [ Value-based type inference ] ~~~~~~~~~~~~~~~~~~~~~~~~~~~~~~~~~~~~~~~~~~~

\paragraph{Value-based type inference}

The type of a variable may be inferred from the values stored in its register (e.g. 42) or memory location. For instance, the type of the global variable \texttt{unk\_8000040} (see line 32 in figure \ref{fig:local_variable_example_asm}) may be inferred from the sequence of bytes stored at its memory location; given that the characters of the byte sequence (\texttt{'foo', 0}) are within ASCII range (0-127) and ends with a zero, the source type may be a NULL-terminated string.

One issue with value-based type inference is that it is difficult to distinguish pointer values from integer values or arbitrary sequences of bytes (e.g. \texttt{0x414243} could either refer to an address, the integer value \texttt{4276803} or the byte sequence \texttt{ABC}). Therefore, value-based type inference may lead to misclassification of types \cite{type_inference_on_executables}.

One benefit of value-based type inference is that it only requires examining the contents of registers and memory. Given that no extensive code analysis is required, this approach fits dynamic analysis well.

% ~~~ [ Instruction type sources ] ~~~~~~~~~~~~~~~~~~~~~~~~~~~~~~~~~~~~~~~~~~~~~

\paragraph{Instruction type sources}

The type of a variable may be inferred from the operational semantics (e.g. assignment, comparison, arithmetic) of instructions (also known as \textit{type revealing instruction}). An assignment instruction may reveal the size of a memory location, a comparison instruction the signedness of an integer variable, and an arithmetic instruction the relationship between variables. Further more, the use of FPU registers may reveal the source type of a floating-point variable.

Compared to value-based type inference, instruction type source may extract more information. For instance, given value-based inference a variable assigned the value \texttt{321} may be classified as a signed or unsigned integer of bit size at least 16 (since the value \texttt{321} does not fit in 8 bits). More concretely, the \texttt{mov} instruction at \texttt{f:14} (see figure \ref{fig:local_variable_example_asm}) constrains the local variable at \texttt{ebp-4} to either \textit{num16} or \textit{num32} in the primitive type lattice on a 32-bit system (see figure \ref{fig:primitive_type_lattice}). However, there is more information contained within the type revealing \texttt{mov} instruction at \texttt{f:14}, namely the size of the memory location (\texttt{DWORD}) used in the assignment operation. With this added knowledge, the only valid type constraint is \texttt{num32}.

One issue with relying on instruction type sources to determine the size of variables in memory, is that optimizing compilers may produce executables containing distinct instructions accessing the same memory location using different data size specifiers. This is common when using \texttt{memcpy} and \texttt{memmove} on structure types in binaries produced by optimizing compilers; as further illustrated in figure \ref{fig:struct_example_asm} where \texttt{g} and \texttt{h} refer to the structure type \texttt{T} using different size specifiers (i.e. \texttt{DWORD} for offset 4 at \texttt{h:49} and \texttt{BYTE} for offset 4 at \texttt{g:21}).

% ~~~ [ Function type sources ] ~~~~~~~~~~~~~~~~~~~~~~~~~~~~~~~~~~~~~~~~~~~~~~~~

\paragraph{Function type sources}

Type information derived for arguments in calls to functions with known functions signatures (e.g. those of the standard library) \cite{type_inference_on_executables}.

% ~~~ [ Type sink ] ~~~~~~~~~~~~~~~~~~~~~~~~~~~~~~~~~~~~~~~~~~~~~~~~~~~~~~~~~~~~

\paragraph{Type sink}

A type sink is a type which can be resolved directly and is known to be correct. For instance, the type of an argument in a function call to a standard library function.

% ~~~ [ Format-string based type inference ] ~~~~~~~~~~~~~~~~~~~~~~~~~~~~~~~~~~~

\paragraph{Format-string based type inference}

The types of arguments in function calls to the \texttt{printf} family of functions in the standard library may be inferred from the corresponding format string verbs (e.g. \texttt{\%d} for integer and \texttt{\%f} for floating-point numbers).

For instance, based on the format string \texttt{``\%s, \%d''} used in the call to \texttt{printf} at \texttt{f:20} (see figure \ref{fig:local_variable_example_asm}) the types of the corresponding variables \texttt{ebp-8} and \texttt{ebp-4} may be inferred from the string verbs \texttt{\%s} for NULL-terminated string and \texttt{\%d} for integer number, respectively.

Here, it is important to note that the \texttt{eax} register containing the address to the format string at \texttt{f:19} is used at two other locations in \texttt{f}, each referring to distinct variables. At \texttt{f:15}, \texttt{eax} refers to the string literal \texttt{``foo''} stored in the global memory region, and at \texttt{f:25}, \texttt{eax} refers to the integer return value. To separate these variables from one another, SSA-form is used to track live ranges and create distinct \textit{versions} of the \texttt{eax} register.
