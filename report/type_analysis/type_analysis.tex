\documentclass[10pt, a4paper, sigplan, authordraft]{acmart}
% authordraft

%                     Reflektera
%                     +---------+
%           Recensera |         |
%           +---------+         |
%  Referera |                   |
% +---------+                   |
% |                             |
% +-----------------------------+

% TODO: Update title.
% TODO: check use of of.

\usepackage{preamble}

\title{Type Analysis of Low-level Code}

\author{Robin Eklind}
\affiliation{
	\institution{Royal Institute of Technology (KTH)}
	\city{Stockholm}
	\country{Sweden}
}
\orcid{0000-0003-0275-5514}

% TODO: remove some keywords

\keywords{type analysis, type recovery, type inference, type constraints, type lattice, variable recovery, decompilation, reverse engineering, low-level code, assembly, LLVM IR, SSA, formal verification, binary analysis, static analysis, dynamic analysis}

\begin{document}

% === [ Front matter ] =========================================================

% --- [ Abstract ] -------------------------------------------------------------

%\begin{abstract}
%foo
%\end{abstract}

\begin{quote}
\todo{Note, this is an early draft}. As such most sections are incomplete. I would very much appreciate early feedback on the \textit{overall structure} of the paper, what sections to include, how many levels of subsections to use, if some sections should be merged, etc. As this is the first time I'm writing a conference paper, any feedback specifically related to this format of writing would be invaluable. The specific conference is the ACM SIGPLAN programming language conference. Throughout this paper, I've used \todo{yellow labels} for TODO notes to myself (often used in sections that are far from complete). \question{Blue labels} are used for specific questions, for which I'd love to get some feedback. Look forward to seeing you on Wednesday Jamie. \\
Kindly, Robin \\

Including a Table of Contents below, to get a sense of the report structure. The table of contents will not be included in the final paper.
\end{quote}

\tableofcontents

\clearpage

% --- [ Teaser image ] ---------------------------------------------------------

\begin{teaserfigure}
	\centering
	\includegraphics[width=0.45\textwidth]{inc/tie_primitive_type_lattice.png}
	\caption{Primitive type lattice of TIE}
	\label{fig:base_type_lattice}
\end{teaserfigure}

% --- [ Title ] ----------------------------------------------------------------

\maketitle

% === [ Main matter ] ==========================================================

\question{QUESTION: Include abstract? The final paper will be around 6 pages.}

% === [ Introduction ] =========================================================

\section{Introduction}
\label{sec:introduction}

A compiler is a piece of software which translates human readable high-level programming languages (e.g. C) to machine readable low-level languages (e.g. Assembly). In the usual flow of compilation, code is lowered through a set of transformations from a high-level to a low-level representation. The decompilation process (originally referred to as reverse compilation \cite{reverse_comp}) moves in the opposite direction by lifting code from a low-level to a high-level representation.

Decompilation enables source code reconstruction of binary applications and libraries. Both security researchers and software engineers may benefit from decompilation as it facilitates analysis, modification and reconstruction of object code. The applications of decompilation are versatile, and may include one of the following uses:

\begin{itemize}
	\item Analyze malware
	\item Recover source code
	\item Migrate software from legacy platforms or programming languages
	\item Optimize existing binary applications
	\item Discover and mitigate bugs and security vulnerabilities
	\item Verify compiler output with regards to correctness
	\item Analyze proprietary algorithms
	\item Improve interoperability with other software
	\item Add new features to existing software
\end{itemize}

As recognized by Edsger W. Dijkstra in his 1972 ACM Turing Lecture (an extract from which is presented in figure \ref{fig:dijkstra_lecture}) one of the most powerful tools for solving complex problems in Computer Science is the use of abstractions and separation of concerns. This paper explores a compositional approach to decompilation which facilitates abstractions to create a pipeline of self-contained components. Since each component interacts through language-agnostic interfaces (well-defined input and output) they may be written in a variety of programming languages. Furthermore, for each component of the decompilation pipeline there may exist multiple implementations with their individual strengths and weaknesses. The end user (e.g. malware analyst, security researcher, reverse engineer) may select the components which solves their task most efficiently.

\begin{figure}[htbp]
	\begin{quote}
		\textit{``We all know that the only mental tool by means of which a very finite piece of reasoning can cover a myriad cases is called ``abstraction''; as a result the effective exploitation of their powers of abstraction must be regarded as one of the most vital activities of a competent programmer. In this connection it might be worthwhile to point out that the purpose of abstracting is not to be vague, but to create a new semantic level in which one can be absolutely precise. Of course I have tried to find a fundamental cause that would prevent our abstraction mechanisms from being sufficiently effective. But no matter how hard I tried, I did not find such a cause. As a result I tend to the assumption -- up till now not disproved by experience -- that by suitable application of our powers of abstraction, the intellectual effort needed to conceive or to understand a program need not grow more than proportional to program length.''} \cite{abstractions_quote}
	\end{quote}
	\caption{An extract from the ACM Turing Lecture given by Edsger W. Dijkstra in 1972.}
	\label{fig:dijkstra_lecture}
\end{figure}

\pagebreak % <layout>

% --- [ Project Aim and Objectives ] -------------------------------------------

\subsection{Project Aim and Objectives}
\label{sec:intro_project_aim_and_objectives}

The aim of this project is to facilitate decompilation workflows using composition of language-agnostic decompilation passes; specifically the reconstruction of high-level control structures and, as a future ambition, expressions.

To achieve this aim, the following objectives have been identified:

\begin{enumerate}
	\item Review traditional decompilation techniques, including control flow analysis and data flow analysis.
	\label{itm:obj_review_decomp_techniques}
	\item Critically evaluate a set of Intermediate Representations (IRs), which describes low-, medium- and high-level language semantics, to identify one or more suitable for the decompilation pipeline.
	\label{itm:obj_review_suitable_ir}
	\item Analyse the formal grammar (language specification) of the IR to verify that it is unambiguous. If the grammar is ambiguous or if no formal grammar exists, produce a formal grammar. This objective is critical for language-independence, as the IR works as a bridge between different programming languages.
	\label{itm:obj_formal_ir}
	\item Determine if any existing library for the IR satisfies the requirements; and if not develop one. The requirements would include a suitable in-memory representation, and support for on-disk file storage and arbitrary manipulations (e.g. inject, delete) of the IR.
	\label{itm:obj_ir_library}
	\item Design and develop components which identify the control flow patterns of high-level control structures using control flow analysis of the IR.
	\label{itm:obj_control_flow_analysis_component}
	\item Develop tools which perform one or more decompilation passes on a given IR. The tools will be reusable by other programming language environments as their input and output is specified by a formally defined IR.
	\label{itm:obj_decomp_pass_tool}
	\item As a future ambition, design and develop components which perform expression propagation using data flow analysis of the IR.
	\label{itm:obj_data_analysis_library}
\end{enumerate}

% --- [ Deliverables ] ---------------------------------------------------------

\subsection{Deliverables}

The source code and the report of this project have been released into the public domain\footnote{CC0 1.0 Universal: \url{https://creativecommons.org/publicdomain/zero/1.0/}} and are made available on GitHub.

The following documents have been produced:
\begin{itemize}
	\item Project report; refer to objective \ref{itm:obj_review_decomp_techniques} and \ref{itm:obj_review_suitable_ir} \\ \url{https://github.com/mewpaper/decompilation}
\end{itemize}

And the following system artefacts have been developed:
\begin{itemize}
	\item Library for interacting with LLVM IR (\textit{work in progress}); refer to objective \ref{itm:obj_ir_library} \\ \url{https://github.com/llir/llvm}
	\item Control flow graph generation tool; refer to objective \ref{itm:obj_control_flow_analysis_component} \\ \url{https://github.com/decomp/ll2dot}
	\item Subgraph isomorphism search algorithms and related tools; refer to objective \ref{itm:obj_control_flow_analysis_component} \\ \url{https://github.com/decomp/graphs}
	\item Control flow analysis tool; refer to objective \ref{itm:obj_decomp_pass_tool} \\ \url{https://github.com/decomp/restructure}
	\item Go code generation tool (\textit{proof of concept}); refer to objective \ref{itm:obj_decomp_pass_tool} \\ \url{https://github.com/decomp/ll2go}
	\item Go post-processing tool; refer to objective \ref{itm:obj_decomp_pass_tool} \\ \url{https://github.com/decomp/go-post}
\end{itemize}

% --- [ Disposition ] ----------------------------------------------------------

\subsection{Disposition}

This report details every stage of the project from conceptualisation to successful completion. It follows a logical structure and outlines the major stages in chronological order. A brief summary of each section is presented in the list below.

\begin{itemize}
	\item Section \ref{sec:introduction} - \textbf{Introduction} \\ \textit{Introduces the concept of decompilation and its applications, outlines the project aim and objectives, and summarises its deliverables.}
	\item Section \ref{sec:literature_review} - \textbf{Literature Review} \\ \textit{Details the problem domain, reviews traditional decompilation techniques, and evaluates potential intermediate representations for the decompilation pipeline of the project.}
	\item Section \ref{sec:related_work} - \textbf{Related Work} \\ \textit{Evaluates projects for translating native code to LLVM IR, and reviews the design of modern decompilers.}
	\item Section \ref{sec:methodology} - \textbf{Methodology} \\ \textit{Surveys methodologies and best practices for software construction, and relates them to the specific problem domain.}
	\item Section \ref{sec:requirements} - \textbf{Requirements} \\ \textit{Specifies and prioritises the requirements of the project artefacts.}
	\item Section \ref{sec:design} - \textbf{Design} \\ \textit{Discusses the system architecture and the design of each component, motivates the choice of core algorithms and data structures, and highlights strengths and limitations of the design.}
	\item Section \ref{sec:implementation} - \textbf{Implementation} \\ \textit{Discusses language considerations, describes the implementation process, and showcases how set-backs were dealt with.}
	\item Section \ref{sec:verification} - \textbf{Verification} \\ \textit{Describes the approaches taken to validate the correctness, performance and security of the artefacts.}
\clearpage % <layout>
	\item Section \ref{sec:evaluation} - \textbf{Evaluation} \\ \textit{Assesses the outcome of the project and evaluates the artefacts against the requirements.}
	\item Section \ref{sec:conclusion} - \textbf{Conclusion} \\ \textit{Summarises the project outcomes, presents ideas for future work, reflects on personal development, and concludes with an attribution to the key idea of this project.}
\end{itemize}


% === [ Background ] ===========================================================

\section{Background}

This section provides a brief background on important terminology and concepts related to type analysis and binary analysis.

To help explain different aspects of variable recovery and type analysis, two running examples are used throughout this paper. The first example (see figure \ref{fig:local_variable_example} of appendix \ref{app:local_variable_example}) details a function \texttt{f} which allocates local variables on the function stack frame. The second example (see figure \ref{fig:struct_example} of appendix \ref{app:struct_example}) details a function \texttt{g} which operates on an array of structures, and a function \texttt{h} which operates on a structure pointer. Subsequent in-text references to \texttt{f}, \texttt{g} and \texttt{h} correspond to these three functions, and the \texttt{f:12} notation represents a reference to the function \texttt{f} at line number 12 in the source listing.

% === [ Subsections ] ==========================================================

% --- [ Static Single Assignment ] ---------------------------------------------

\subsection{Static Single Assignment}

\todo{TODO: write about static single assignment}

% registers

Track live ranges of variables stored in registers; SSA.

\todo{TODO: describe PHI instructions. merges data flow at joining point of control flow graph.}

\begin{figure}[htbp]
	\centering
	\begin{subfigure}[ht]{0.50\textwidth}
		\lstinputlisting[language=llvm, style=nasm, breaklines=false]{inc/ssa.ll}
		\caption{LLVM IR in SSA-form: each value is assigned exactly once.}
		\label{fig:ssa_form}
	\end{subfigure}
	\begin{subfigure}[ht]{0.50\textwidth}
		\centering
		\includegraphics[width=0.50\textwidth]{inc/phi.png}
		\caption{$\Phi$-node semantics: value of \texttt{\%x.0} in \texttt{\%ret} depend on branch taken.}
		\label{fig:phi_node_semantics}
	\end{subfigure}
	\caption{LLVM IR in SSA-form and associated CFG.}
\end{figure}

% --- [ Type Lattice ] ---------------------------------------------------------

\subsection{Type Lattice}

A type lattice may be thought of as a set of subtyping relationships, represented as a directed graph from the \textit{top} type $\top$ to the \textit{bottom} type $\bot$; where every type is a subtype of $\top$, and no type is a subtype of $\bot$.

\begin{itemize}
	\item $\top$: any type
	\item $\bot$: inconsistent type
\end{itemize}

In type primitive type lattice of TIE (see figure \ref{fig:primitive_type_lattice}), for instance, both signed and unsigned 32-bit integers (\texttt{int32} and \texttt{uint32}, respectively) are subtypes of 32-bit integers (\texttt{num32}) \cite{tie_reverse_engineering_of_types}.

\begin{figure}[htbp]
	\centering
	\includegraphics[width=0.40\textwidth]{inc/tie_primitive_type_lattice.png}
	\caption{Primitive type lattice of TIE.}
	\label{fig:primitive_type_lattice}
\end{figure}

In the context of type recovery, a type lattice may specify the set of possible types for the variable through upper and lower bounds; thus imposing type constraints on the variable.


% TODO: Skip static and dynamic analysis?
%% --- [ Static Analysis ] ------------------------------------------------------

\subsection{Static Analysis}

\todo{TODO: describe relevant aspects of static analysis. Covers all code paths but does not known concrete values of pointers, and some techniques of memory graph analysis are thus limited. (Verify if this is actually true, claimed in Type Inference on Executables. However Symbolic execution may mitigate this limitation even in static analysis.)} \cite{type_inference_on_executables}

%% --- [ Dynamic Analysis ] -----------------------------------------------------

\subsection{Dynamic Analysis}

\todo{TODO: describe relevant aspects of dynamic analysis. Does not cover all code paths (as static analysis does), but still relevant and widely used in Malware forensics. Particularly useful for memory access analysis, to recover structure types as their members are accessed during execution.} \cite{dynstruct}



\question{QUESTION: change order of Variable Recovery section and Type Analysis section? Alternatively, merge the two sections.}

% === [ Variable Recovery ] ====================================================

\section{Variable Recovery}

\question{QUESTION: change order of Variable Recovery section and Type Analysis section? Alternatively, merge the two sections.}

\todo{TODO: add meta-text; prior to type analysis, the set of variables must first be recovered, as they are to be assigned types.}

% === [ Subsections ] ==========================================================

% --- [ Value Set Analysis ] ---------------------------------------------------

% Results: objective communication of data

\subsection{Value Set Analysis}

\todo{TODO: describe value set analysis}

upper bound, lower bound and stride for memory locations. \cite{wysinwyx}

% --- [ Function Signature Recovery ] ------------------------------------------

\subsection{Function Signature Recovery}

\todo{TODO: add note about call conventions}

% TODO: rephrase.

The following set of steps are used for function signature recovery (function parameters and returns arguments) in SecondWrite \cite{second_write_scalable_type_detection}.

\begin{enumerate}
	\item assume all registers are arguments and no register are return arguments.
	\item registers written to are potential return arguments.
	\item callee saved registers (through push in function prologue and pop in function epilogue) (\textit{DeadStores}) are pruned from potential return set.
	\item prune arguments not actually used (e.g. not part of \textit{DeadStore} or PHI instructions).
	\item prune return registers not actually used by callers.
\end{enumerate}


% === [ Type Recovery ] ========================================================

\section{Type Recovery}

% Key: many problems in binary analysis can be seen as subproblems of type analysis.

% binary analysis,

% floating-point stack
%    - track stack top (indirect calls/external calls)
% pointer analysis
%    - global memory region
%    - stack memory region
%    - heap memory region
% structures/arrays

\todo{TODO: give introduction to type recovery as part of type analysis of low-level code.} \cite{mycroft_type_based_decompilation}

% === [ Subsections ] ==========================================================

% --- [ Type Inference ] -------------------------------------------------------

\subsection{Type Inference}

% ~~~ [ Value-based type inference ] ~~~~~~~~~~~~~~~~~~~~~~~~~~~~~~~~~~~~~~~~~~~

\paragraph{Value-based type inference}

The type of a variable may be inferred from the values stored in its register (e.g. 42) or memory location. For instance, the type of the global variable \texttt{unk\_8000040} (see line 32 in figure \ref{fig:local_variable_example_asm}) may be inferred from the sequence of bytes stored at its memory location; given that the characters of the byte sequence (\texttt{'foo', 0}) are within ASCII range (0-127) and ends with a zero, the source type may be a NULL-terminated string.

One issue with value-based type inference is that it is difficult to distinguish pointer values from integer values or arbitrary sequences of bytes (e.g. \texttt{0x414243} could either refer to an address, the integer value \texttt{4276803} or the byte sequence \texttt{ABC}). Therefore, value-based type inference may lead to misclassification of types \cite{type_inference_on_executables}.

One benefit of value-based type inference is that it only requires examining the contents of registers and memory. Given that no extensive code analysis is required, this approach fits dynamic analysis well.

% ~~~ [ Instruction type sources ] ~~~~~~~~~~~~~~~~~~~~~~~~~~~~~~~~~~~~~~~~~~~~~

\paragraph{Instruction type sources}

Also known as \textit{type revealing instruction}.

The type of a variable may be inferred from the operational semantics (e.g. assignment, comparison, arithmetic) of instructions.

For instance, a variable assigned the value \texttt{321} may be classified as a signed or unsigned integer of bit size at least 16 (since the value \texttt{321} does not fit in 8 bits). More concretely, the \texttt{mov} instruction at \texttt{f:14} (see figure \ref{fig:local_variable_example_asm}) constrains the local variable at \texttt{ebp-4} to either \textit{num16} or \textit{num32} in the primitive type lattice on a 32-bit system (see figure \ref{fig:base_type_lattice}).

% (e.g. assignment, comparison, arithmetic)

\todo{TODO: describe type revealing instructions. Infer from the instruction that the operand is of integer type, memory, floating-point, etc.}

% ~~~ [ Function type sources ] ~~~~~~~~~~~~~~~~~~~~~~~~~~~~~~~~~~~~~~~~~~~~~~~~

\paragraph{Function type sources}

Type information derived for arguments in calls to functions with known functions signatures (e.g. those of the standard library) \cite{type_inference_on_executables}.

% ~~~ [ Type sink ] ~~~~~~~~~~~~~~~~~~~~~~~~~~~~~~~~~~~~~~~~~~~~~~~~~~~~~~~~~~~~

\paragraph{Type sink}

A type sink is a type which can be resolved directly and is known to be correct. For instance, the types of arguments inferred from function type sources.

% ~~~ [ Format-string based type inference ] ~~~~~~~~~~~~~~~~~~~~~~~~~~~~~~~~~~~

\paragraph{Format-string based type inference}

\todo{TODO: describe format-string based type inference.}

Type of arguments to \texttt{printf} call may be inferred by format-string verbs (e.g. \texttt{\%d}).

\todo{TODO: use example of eax register in NASM example above to illustrate the need to track life ranges}.

% --- [ Type Propagation ] -----------------------------------------------------

\subsection{Type Propagation}

\todo{TODO: describe type propagation, type constraints, forward and backward propagation, constraint solving}

\todo{TODO: describe type inference, Algorithm W} \cite{milner_algorithmw}

% --- [ Unification ] ----------------------------------------------------------

\subsection{Unification}

\todo{TODO: describe the unification step, as many algorithms rely on this to discover structure types, etc.}

% --- [ Pointer analysis ] -----------------------------------------------------

\subsection{Pointer analysis}

\todo{TODO: give brief introduction to pointer analysis, points-to sets, pointer aliasing, and the technique to merge call sites such that memory allocated at different allocation call sites (e.g. calls to \texttt{malloc}) may have the same type if proven that through type propagation, that the returned pointers are used at other parts of the program for the same variables.}


% === [ Evaluation Metrics ] ===================================================

\section{Evaluation Metrics}

To enable objective comparison of different type recovery methods, a set of shared evaluation metrics is required. For this purpose, TIE proposed two evaluation metrics: \textit{distance} and \textit{conservativeness} \cite{tie_reverse_engineering_of_types}.

The open source tools of Coreutils are often used in benchmark comparisons of different type recovery methods. The source code may be compiled to include debug information of source variables and types in the executables. This information is only used for validation, and no type recovery method assessed in this paper depend on the information to be available.

% === [ Subsections ] ==========================================================

% --- [ Distance ] -------------------------------------------------------------

\subsection{Distance}

The distance metric measures the distance in height between the recovered type and the source type in the primitive type lattice (see figure \ref{fig:primitive_type_lattice}). The calculation of distance is meaningful only for subtypes (e.g. \texttt{int32} is a subtype of \texttt{reg32} at distance 2), otherwise the maximum lattice height is used.

% --- [ Conservativeness ] -----------------------------------------------------

\subsection{Conservativeness}

Reports whether the source type exists within the type range (lower and upper bound) of the recovered type.

The distance measurement as defined by TIE did not distinguish between multi-level pointers (e.g. \texttt{int*} and \texttt{int**} are equivalent, i.e. on the same level in the type lattice). Thus SecondWrite proposed a refinement to measure the ratio between the recovered pointer level and the source pointer level \cite{second_write_scalable_type_detection}.

Other metrics have been proposed to compare class hierarchies.


% === [ Discussion ] ===========================================================

\section{Discussion}

% (Reminder: method, research results.)

\todo{TODO: critically assess results.}

% * make claims
% * interpret results (interpretation of your data)
%    - what limitations are there?
%    - can we depend on this?
%    - relevance to your research question?
%    - relevance to your field of study?


% TODO: skip? \cite{bintype}
% TODO: skip? \cite{polymorphic_type_inference_for_machine_code}

% === [ Back matter ] ==========================================================

% --- [ References ] -----------------------------------------------------------

\bibliography{references}

%\appendix
%\section{Appendix}
%
%foo

\end{document}
