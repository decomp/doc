% --- [ Documentation ] --------------------------------------------------------

\subsection{Documentation}

A set of source code analysis tools are used to automate the generation and presentation of documentation. The main benefit of this approach is that only one version of the documentation has to be maintained and it is kept within the source code, thus preventing it from falling out of sync with the implementation. Unix manual pages are generated for command line tools using \texttt{mango}\footnote{Generate Man pages from Go source: \url{https://github.com/slyrz/mango}}, which locates the relevant comments and command line flag definitions in the source code. Library documentation is presented using \texttt{godoc}\footnote{Godoc extracts and generates documentation for Go programs: \url{https://golang.org/cmd/godoc}} (a tool similar to \texttt{doxygen}), and may be accessed through a web or command line interface.

The GoDoc.org server hosts an instance of \texttt{godoc} which presents the documentation of publicly available source code repositories. An online version of the documentation has been made available for each artefact using this service.

\begin{itemize}
	\item Library for interacting with LLVM IR (\textit{work in progress}) \\ \url{https://godoc.org/github.com/llir/llvm}
	\item Control flow graph generation tool \\ \url{https://godoc.org/decomp.org/x/cmd/ll2dot}
	\item Subgraph isomorphism search algorithms and related tools \\ \url{https://godoc.org/decomp.org/x/graphs}
	\item Control flow recovery tool \\ \url{https://godoc.org/decomp.org/x/cmd/restructure}
	\item Go code generation tool (\textit{proof of concept}) \\ \url{https://godoc.org/decomp.org/x/cmd/ll2go}
	\item Go post-processing tool \\ \url{https://godoc.org/decomp.org/x/cmd/go-post}
\end{itemize}
