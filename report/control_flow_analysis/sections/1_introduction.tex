% === [ Introduction ] =========================================================

\section{Introduction}

Control flow recovery is the process of identifying structures of high-level control flow primitives, such as 2-way conditionals (i.e. \textit{if-else} statements) and pre-test loops (i.e. \textit{while}-loops), in unstructured control flow graphs (CFGs).

Information of high-level control flow primitives in the original source code is lost during compilation, and the recovery of such information presents interesting challenges since compiler optimisations (such as jump threading and code relocation) may produce irreducible graphs \cite{cifuentes_reverse_comp}. Solutions to this problem domain therefore include trade-offs between preservation of the original CFG and introduction of additional nodes (e.g. auxiliary conditional nodes \cite{no_more_gotos} or code duplication through node splitting \cite{node_splitting}) to recovery high-level control flow primitives from otherwise irreducible CFGs.

The applications of control flow recovery are versatile, and as part of binary analysis and decompilation tasks include:

\begin{itemize}
	\item malware analysis
	\item security assessments
	\begin{itemize}
		\item automated vulnerability scanning
		\item discovery and mitigation of bugs and security vulnerabilities
	\end{itemize}
	\item control-flow aware compiler passes
	\item enabling high-level data-flow analysis on IR (Rosen's method \cite{rosen_method,advanced_compiler_design_implementation_book})
	\item verification of compiler output (\textit{Reflections on Trusting Trust} \cite{trusting_trust})
	\item decompilation (reverse compilation)
	\begin{itemize}
		\item transpilation between programming languages\footnote{Transpilation between $n$ source and $m$ target langauges requires $n + m$ combinations when utilizing an IR, instead of $n \cdot m$ combinations.}
		\item migrate proprietary software from legacy architectures
		\item re-optimization of software where source code or tool chain is missing
	\end{itemize}
\end{itemize}

% === [ Subsections ] ==========================================================

% --- [ Project Aim and Objectives ] -------------------------------------------

\subsection{Project Aim and Objectives}
\label{sec:intro_project_aim_and_objectives}

The aim of this project is to facilitate decompilation workflows using composition of language-agnostic decompilation passes; specifically the reconstruction of high-level control structures and, as a future ambition, expressions.

To achieve this aim, the following objectives have been identified:

\begin{enumerate}
	\item Review traditional decompilation techniques, including control flow analysis and data flow analysis.
	\label{itm:obj_review_decomp_techniques}
	\item Critically evaluate a set of Intermediate Representations (IRs), which describes low-, medium- and high-level language semantics, to identify one or more suitable for the decompilation pipeline.
	\label{itm:obj_review_suitable_ir}
	\item Analyse the formal grammar (language specification) of the IR to verify that it is unambiguous. If the grammar is ambiguous or if no formal grammar exists, produce a formal grammar. This objective is critical for language-independence, as the IR works as a bridge between different programming languages.
	\label{itm:obj_formal_ir}
	\item Determine if any existing library for the IR satisfies the requirements; and if not develop one. The requirements would include a suitable in-memory representation, and support for on-disk file storage and arbitrary manipulations (e.g. inject, delete) of the IR.
	\label{itm:obj_ir_library}
	\item Design and develop components which identify the control flow patterns of high-level control structures using control flow analysis of the IR.
	\label{itm:obj_control_flow_analysis_component}
	\item Develop tools which perform one or more decompilation passes on a given IR. The tools will be reusable by other programming language environments as their input and output is specified by a formally defined IR.
	\label{itm:obj_decomp_pass_tool}
	\item As a future ambition, design and develop components which perform expression propagation using data flow analysis of the IR.
	\label{itm:obj_data_analysis_library}
\end{enumerate}

% --- [ Project Thesis ] -------------------------------------------------------

\subsection{Project Thesis}

% TODO: progression/spectrum?

To evaluate the set of control flow recovery methods, a notion of \textit{false positive} and \textit{false negative} recovery is used; as defined below, and further described in section \ref{sec:cfa_metric}.

\begin{description}
	\item[false positive recovery] denotes a control flow primitive \textit{recovered} but \textit{not present} in the original source code.
	\item[false negative recovery] denotes a control flow primitive \textit{present} in the original source code but \textit{not recovered}.
\end{description}

The thesis of this project is that upon evaluation of three distinct methods for control flow recovery (see section \ref{sec:hammock_method}, \ref{sec:interval_method} and \ref{sec:pattern-independent_method}), the methods will fall on a progression from \textit{many} false negatives and \textit{no} false positives for the Hammock method; to \textit{some} false negatives and \textit{some} false positives for the Interval method; to \textit{no} false negatives and \textit{many} false positives for the Pattern-independent method.

Put more concretely,
\begin{itemize}
	\item the \textbf{Hammock method} will only recover \textit{some} control flow primitives present in the original source code, but will \textit{never} introduce control flow primitives not present in the original source code;
	\item the \textbf{Interval method} will recover \textit{most} control flow primitives present in the original source code, but will introduce \textit{some} control flow primitives not present in the original source code; and
	\item the \textbf{Pattern-independent method} will recover \textit{all} control flow primitives present in the original source code, but will introduce \textit{many} control flow primitives not present in the original source code.
\end{itemize}

% --- [ Project Deliverables ] -------------------------------------------------

\subsection{Project Deliverables}

Documents to be produced:

\begin{itemize}
    \item Project report.
\end{itemize}

System artefacts to be developed:

\begin{itemize}
    \item Library for control flow analysis with support for a set of control flow recovery methods; refer to objective 3.
    \item Tool for performing control flow recovery on a given CFG; refer to objective 4.
\end{itemize}

