% --- [ Evaluation Process ] ---------------------------------------------------

\subsection{Evaluation Process}

% TODO: rephrase

A preliminary metric for the effectiveness of control flow recovery is how many of the original high-level control flow primitives that are recovered and how many are omitted. The metric for effectiveness should also includes a notion of false positives for high-level primitives that are recovered but are not part of the original source code.

For each test program, the evaluation process consists of the following steps.

\begin{enumerate}
	\item Parse the original C source code of the test program to determine the occurrence of each high-level control flow primitive, and store this truth table in JSON format.
	\item Convert C source code to LLVM IR using Clang. For test programs consisting of several source files, the LLVM IR is produced using the WLLVM\footnote{Whole Program LLVM: \url{https://github.com/travitch/whole-program-llvm}} project which provides tools for building whole-program LLVM bitcode files from unmodified C or C++ source code.
	\item Generate a control flow graph for each function defined in the LLVM IR, and store the CFGs in Graphviz DOT format.
	\item Apply each control flow recovery method on the generated CFGs, and compare the recovered high-level primitives against the truth table produced in step 1.
\end{enumerate}

Special consideration has been taken to ensure that the C source code is parsed identically in step 1 and 2, since pre-processor directives and compiler specific macros may otherwise introduce uncertainty into the results. The Python script which generates truth tables in step 1 parses C source files using the official bindings for the Clang library, and the C source files are converted to LLVM IR in step 2 using Clang (also when invoked through WLLVM).
