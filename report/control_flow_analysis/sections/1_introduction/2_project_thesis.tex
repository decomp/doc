% --- [ Project Thesis ] -------------------------------------------------------

\subsection{Project Thesis}

% TODO: progression/spectrum?

To evaluate the set of control flow recovery methods, a notion of \textit{false positive} and \textit{false negative} recovery is used; as defined below, and further described in section \ref{sec:cfa_metric}.

\begin{description}
	\item[false positive recovery] denotes a control flow primitive \textit{recovered} but \textit{not present} in the original source code.
	\item[false negative recovery] denotes a control flow primitive \textit{present} in the original source code but \textit{not recovered}.
\end{description}

The thesis of this project is that upon evaluation of three distinct methods for control flow recovery (see section \ref{sec:hammock_method}, \ref{sec:interval_method} and \ref{sec:pattern-independent_method}), the methods will fall on a progression from \textit{many} false negatives and \textit{no} false positives for the Hammock method; to \textit{some} false negatives and \textit{some} false positives for the Interval method; to \textit{no} false negatives and \textit{many} false positives for the Pattern-independent method.

Put more concretely,
\begin{itemize}
	\item the \textbf{Hammock method} will only recover \textit{some} control flow primitives present in the original source code, but will \textit{never} introduce control flow primitives not present in the original source code;
	\item the \textbf{Interval method} will recover \textit{most} control flow primitives present in the original source code, but will introduce \textit{some} control flow primitives not present in the original source code; and
	\item the \textbf{Pattern-independent method} will recover \textit{all} control flow primitives present in the original source code, but will introduce \textit{many} control flow primitives not present in the original source code.
\end{itemize}
