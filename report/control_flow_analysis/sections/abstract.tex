\begin{abstract}

% At most 300 words.

% --- [ Background, Purpose and Methodology ] ----------------------------------

% 66 + 93 words.

% What?

Control flow recovery is the process of analysing control flow graphs to recover information about the underlying high-level control flow primitives of unstructured machine code or low-level intermediate representations.

% Why?

The applications of control flow recovery are versatile. Among others, it is used within compilers, automated vulnerability scanners and decompilers to enable high-level data flow analysis of IR, facilitate binary analysis and reconstruct structured source code from binary applications.

% How?

A number of control flow recovery methods have been proposed by the scientific community. However, each method has its respective benefits and drawbacks, and there is no one method that outperforms every other method in all metrics. Therefore, gaining insight into the strengths and limitations of different control flow recovery methods is valuable.

% Technical contributions

To provide insight, this project distils key ideas from three primary methods of control flow recovery, showcasing their strengths and providing intuition for their failure modes.

% Future work

Based on these insights, ideas for further improvements and future research are presented.


%The three primary control flow recovery methods that have been evaluated are the Hammock method, the Interval method and the Patter-independent method.

%The evaluated methods fall on a progression from many false negatives and no false positives, to some false negatives and some false positives, to no false negatives and many false positives.

%However, each method has their respective benefits and drawbacks, and knowing when to use which method requires insight into the concrete problem instances and the specific methods. To help facilitate gaining these insights, a set of tools have been developed that help visualize...

% --- [ Results and Conclusion ] -----------------------------------------------

% 46 + 84 words.

\end{abstract}
