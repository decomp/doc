% === [ Evaluation ] ===========================================================

% <mark>
% - Description of plan for evaluating outcome.
% - Convincing evidence that artefact meets requirements with explanation where
%   it doesn't.
% - Justification of evaluation method.
% - Shows awareness of limits of evaluation.
%
% - How well does the report describe and justify the means by which the outcome
%   of the project was evaluated?
% - How well is it shown whether the specification of the requirements has been
%   satisfied?
% - How well explained are areas where it hasn't?

% <howto> Relationship between sections.
%
%    Requirements -----> Evaluation
%
% <howto>
% - You should describe how you demonstrated that system works as intended (or not, as the case may be)
% - Include comprehensible summaries of the results of all critical tests that you made.
% - You should try to indicate how confident you are about whatever code you have produced, and also suggest what tests would be required to gain further confidence.
% - You must also critically evaluate your system in the light of these tests, describing its strengths and weaknesses.

\section{Evaluation}
\label{sec:evaluation}

This section evaluates the artefacts of the decompilation system against the requirements outlined in section~\ref{sec:requirements}. To assess the capabilities of the individual components, relevant decompilation scenarios have been considered. The current state of each component is summarised in the succeeding paragraphs, and future work to validate the design, improve the reliability, and extend the capabilities of the decompilation pipeline is presented in section~\ref{sec:con_future_work}.

The \texttt{ll2dot} component (see section~\ref{sec:design_control_flow_graph_generation}) is considered stable, but there are known issues which may affect the reliability and the integrity of the produced CFGs; as further described in section~\ref{sec:ver_security_assessment}. Future work which seeks to address these issues is presented in section~\ref{sec:con_reliability_improvements}.

The subgraph isomorphism search library (see section~\ref{sec:impl_subgraph_isomorphism_search_library}) is considered production quality, and the test cases of the \texttt{iso} package have a code coverage of 94.8\%; as outlined in section~\ref{sec:ver_code_coverage}. The restrictions imposed by this library on the subgraph (e.g. single-entry/single-exit invariant and fixed number of nodes) limits infinite loops and n-way conditionals from being modelled, as further discussed in section~\ref{sec:design_control_flow_analysis}. Section~\ref{sec:con_design_validation} presents a discussion of potential approaches which may relax these restrictions in the future.

The \texttt{restructure} component (see section~\ref{sec:design_control_flow_analysis}) is considered production quality, and the test cases of the \texttt{restructure} command have a code coverage of 40.0\%; as outlined in section~\ref{sec:ver_code_coverage}. The \texttt{restructure} command is believed to be capable of structuring the CFG of any source program which may be constructed from the set of supported high-level control flow primitives (which are described in figure~\ref{fig:graph_representations} of section~\ref{sec:lit_review_control_flow_analysis}), including source programs with arbitrarily nested primitives. Any future work which improves the reliability and the capabilities of the subgraph isomorphism search library will directly impact the \texttt{restructure} tool, as it relies entirely on subgraph isomorphism search to recover high-level control flow primitives.

The \texttt{ll2go} component (see section~\ref{sec:design_back-end_components}) is considered a \textit{proof of concept} implementation. It was implemented primarily to stress test the design of the decompilation pipeline, and only supports a small subset (e.g. all arithmetic instructions and some terminator instructions) of the LLVM IR language. The \texttt{ll2go} tool is affected by the same reliability issues as the \texttt{ll2dot} command, which are caused by the Go bindings for LLVM; as further described in section~\ref{sec:impl_go_bindings_for_llvm}. To address these issues a pure Go library is being developed for interacting with LLVM IR, as further described in section~\ref{sec:con_reliability_improvements}. A future version of the \texttt{ll2go} tool would discard the current implementation and start fresh, learning from the mistakes and the building upon the insights.

Lastly, the \texttt{go-post} component (see section~\ref{sec:design_post-processing}) is considered alpha quality, and the test cases of the \texttt{go-post} command have a code coverage of 38.0\%; as outlined in section~\ref{sec:ver_code_coverage}. The \texttt{go-post} tool was primarily implemented to evaluate the feasibility of applying source code transformations to make the decompiled Go code more idiomatic. Implementing these post-processing rules were surprisingly easy, and it was often possible to go from the conceptual idea of a rewrite rule to a working implementation in a matter of hours. While some rewrite rules are reliable (e.g. the \textit{``mainret''} rewrite rule, which is presented in~\ref{fig:rewrite_2} of appendix~\ref{app:post-processing_example}), most are considered experimental. For instance, the \textit{``localid''} rewrite rule (see figure~\ref{fig:rewrite_3} of appendix~\ref{app:post-processing_example}) is known to produce incorrect rewrites when applied to complex programs, but it works for simple programs and provides rudimentary support for expression propagation. A proper implementation of expression propagation would rely on the future implementation of the data flow analysis component, which is mentioned in~\ref{sec:con_design_validation}.

% === [ Subsections ] ==========================================================

% --- [ LLVM IR Library ] ------------------------------------------------------

\subsection{LLVM IR Library}

In total four essential (\textbf{R1}, \textbf{R2}, \textbf{R3} and \textbf{R4}), one desirable (\textbf{R5}), and three future (\textbf{R6}, \textbf{R7} and \textbf{R8}) requirements were identified for the LLVM IR library (see section \ref{sec:req_llvm_ir_library}).

foo

% --- [ Control Flow Analysis Library ] ----------------------------------------

\subsection{Control Flow Analysis Library}
\label{sec:eval_control_flow_analysis_library}

In total five essential (\textbf{R9}, \textbf{R10}, \textbf{R11}, \textbf{R12} and \textbf{R13}), two important (\textbf{R14} and \textbf{R15}), two desirable (\textbf{R16} and \textbf{R17}), and two future (\textbf{R18} and \textbf{R19}) requirements were identified for the control flow analysis library (see section \ref{sec:req_control_flow_analysis_library}). The current implementation of the control flow analysis library satisfies six out of nine requirements (not counting future requirements), and fails to satisfy one essential, one important, and one desirable requirement; as summarized in table \ref{tbl:eval_summary_of_control_flow_analysis_library}. Section \ref{sec:eval_control_flow_analysis_library_essential_requirements}, \ref{sec:eval_control_flow_analysis_library_important_requirements} and \ref{sec:eval_control_flow_analysis_library_desirable_requirements} provides a detailed evaluation of the essential, important and desirable requirements, respectively.

\begin{table}[htbp]
	\begin{center}
		\begin{tabular}{|l|l|l|l|}
			\hline
			Satisfied? & Req. & Priority & Description \\
			\hline
			\rowcolor{light_green_3}
			Yes & \textbf{R9} & MUST & Support analysis of reducible graphs \\
			\rowcolor{light_green_3}
			Yes & \textbf{R10} & MUST & Recover pre-test loops (e.g. \texttt{for}, \texttt{while}) \\
			\rowcolor{light_red_3}
			No & \textbf{R11} & MUST & Recover infinite loops (e.g. \texttt{while(TRUE)}) \\
			\rowcolor{light_green_3}
			Yes & \textbf{R12} & MUST & Recover 1-way conditionals (e.g. \texttt{if}) \\
			\rowcolor{light_green_3}
			Yes & \textbf{R13} & MUST & Recover 2-way conditionals (e.g. \texttt{if-else}) \\
			\hline
			\rowcolor{light_green_3}
			Yes & \textbf{R14} & SHOULD & Recover post-test loops (e.g. \texttt{do-while}) \\
			\rowcolor{light_red_3}
			No & \textbf{R15} & SHOULD & Recover n-way conditionals (e.g. \texttt{switch}) \\
			\hline
			\rowcolor{light_red_3}
			No & \textbf{R16} & COULD & Recover multi-exit loops \\
			\rowcolor{light_green_3}
			Yes & \textbf{R17} & COULD & Recover nested loops \\
			\hline
			N/A & \textbf{R18} & WON'T & Support analysis of irreducible graphs \\
			N/A & \textbf{R19} & WON'T & Recover compound boolean expressions \\
			\hline
		\end{tabular}
	\end{center}
	\caption{A summary of the evaluation against requirements of the control flow analysis library.}
	\label{tbl:eval_summary_of_control_flow_analysis_library}
\end{table}

% ~~~ [ Essential Requirements ] ~~~~~~~~~~~~~~~~~~~~~~~~~~~~~~~~~~~~~~~~~~~~~~~

\subsubsection{Essential Requirements}
\label{sec:eval_control_flow_analysis_library_essential_requirements}

% * R9 - Support analysis of reducible graphs

The current implementation of the control flow analysis library supports analysis of reducible graphs (\textbf{R9}), as demonstrated by the step-by-step analysis of a reducible CFG in appendix \ref{app:control_flow_analysis_example}.

% * R10 - Recover pre-test loops
% * R12 - Recover 1-way conditionals

The successful recovery of pre-test loops (\textbf{R10}) and 1-way conditionals (\textbf{R12}) is demonstrated in four steps, through the use of components which depend on the control flow analysis library. Firstly, the \texttt{ll2dot} tool (see section \ref{sec:design_control_flow_graph_generation}) is used to generate an unstructured CFG for each function of an LLVM IR assmebly file; as demonstrated in appendix \ref{app:control_flow_graph_generation_example}. Secondly, the \texttt{restructure} tool (see section \ref{sec:design_control_flow_analysis}) analyzes the unstructured CFG of an LLVM IR assembly function to produce a structured CFG; as demonstrated in appendix \ref{app:restructure_example}. Thirdly, the \texttt{ll2go} tool (see section \ref{sec:design_code_generation}) uses the high-level control flow information of the structured CFG to decompile the LLVM IR function into unpolished Go code; as demonstrated in appendix \ref{app:code_generation_example}. Lastly, the \texttt{go-post} tool improves the quality of the unpolished Go code, by applying a set of source code transformations; as demonstrated in appendix \ref{app:post-processing_example}. The final Go output, which is presented on the right side of figure \ref{fig:example1_comparison} in appendix \ref{app:post-processing_example}, contains both a \texttt{for}-loop and an \texttt{if}-statement, thus proving that pre-test loops and 1-way conditionals may be recovered.

% * R13 - Recover 2-way conditionals

The successful decompilation of 2-way conditionals (\textbf{R13}) is demonstrated in appendix \ref{app:decompilation_of_nested_primitives}, which provides a contrived example that implicitly uses the same decompilation steps as described above. The final Go output, which is presented on the right side of figure \ref{fig:nested_comparison} in appendix \ref{app:decompilation_of_nested_primitives}, contains an \texttt{if-else} statement, thus proving that 2-way conditionals may be recovered.

% * R11 - Recover infinite loops

The current design of the control flow analyis stage enforces a single-entry/single-exit invariant on the graph representation of high-level control flow primitives. In other words, high-level control flow primitives must be modelled as directed graphs with a single entry and a single exit node. This invariant simplifies the control flow analysis, as it allows identified subgraphs to be merged into single nodes, which inherit the predecessors of the entry node and the successors of the exit node; as demonstrated by the step-by-step simplification of the CFG in appendix \ref{app:control_flow_analysis_example}. This restriction prevents infinite loops (\textbf{R11}) from being modelled however, as they have no exit node. Future work will try to determine if this invariant may be relaxed to include single-entry/no-exit graphs, as further described in section \ref{sec:con_design_validation}.

% ~~~ [ Important Requirements ] ~~~~~~~~~~~~~~~~~~~~~~~~~~~~~~~~~~~~~~~~~~~~~~~

\subsubsection{Important Requirements}
\label{sec:eval_control_flow_analysis_library_important_requirements}

% * R14 - Recover post-test loops

The successful decompilation of post-test loops (\textbf{R14}) is demonstrated in appendix \ref{app:decompilation_of_post-test_loops}, which provides a contrived example that implicitly uses the same decompilation steps as described above. The final Go output, which is presented on the right side of figure \ref{fig:post-test_comparison} in appendix \ref{app:decompilation_of_post-test_loops}, contains an infinite \texttt{for}-loop with a conditional break statement as the last statement of the loop body (which is semantically equivalent to a post-test loop), thus proving that post-test loops may be recovered. Even though Go does not provide native support for post-test loops, the back-end was capable of translating the source program into semantically equivalent Go code, by combining a set of primitives available in Go. The same approach may be used to support missing primitives for other target programming languages (eg. \texttt{switch}-statements in Python).

% * R15 - Recover n-way conditionals

A data-driven design separates the implementation of the control flow analysis component from the definition of supported high-level control flow primitives, which are expressed in the DOT file format. The design is motivated by the principle of separation of concern (e.g. the control flow analysis may be reused to analyze the control flow of REIL) and extensibility (e.g. support for new high-level control flow primitives may be added without changing the source code), as further described in section \ref{sec:design_control_flow_analysis}. One limitation with this design however, is that it does not support n-way conditionals (\textbf{R15}) or any other high-level control flow primitives with a variable number of nodes in their graph representations, as these cannot be expressed in the DOT file format. A discussion on how to mitigate this issue in the future is provided in section \ref{sec:con_design_validation}.

% ~~~ [ Desirable Requirements ] ~~~~~~~~~~~~~~~~~~~~~~~~~~~~~~~~~~~~~~~~~~~~~~~

\subsubsection{Desirable Requirements}
\label{sec:eval_control_flow_analysis_library_desirable_requirements}

% * R16 - Recover multi-exit loops

Implementation strategies for desirable requirements were only considered as time permitted. The support for multi-exit loops (\textbf{R16}) was intentionally omitted from this release, to allocate time for the essential requirements. More research is required to determine how the current design of the control flow analysis stage may be refined to support the recovery of multi-exit loops.

% * R17 - Recover nested loops

The successful decompilation of nested loops (\textbf{R17}) is demonstrated in appendix \ref{app:decompilation_of_nested_primitives}, which provides a contrived example that implicitly uses the same decompilation steps as described above. The final Go output, which is presented on the right side of figure \ref{fig:nested_comparison} in appendix \ref{app:decompilation_of_nested_primitives}, contains nested \texttt{for}-loops (one inner loop and one outer loop), thus proving that nested loops may be recovered.

% --- [ Control Flow Recovery Tool ] -------------------------------------------

\subsection{Control Flow Recovery Tool}

In total two essential (\textbf{R20} and \textbf{R21}) requirements were identified for the control flow recovery tool (see section~\ref{sec:req_control_flow_recovery_tool}).

The control flow analysis component (see section~\ref{sec:design_control_flow_analysis}) satisfies both requirements (not counting future requirements); as summarised in table~\ref{tbl:eval_summary_of_control_flow_recovery_tool}. Section~\ref{sec:eval_control_flow_recovery_tool_essential_requirements} provides a detailed evaluation of the essential requirements.

\begin{table}[htbp]
	\begin{center}
		\begin{tabular}{|l|l|l|l|}
			\hline
			Sat. & Req. & Priority & Description \\
			\hline
			\rowcolor{light_green_3}
			Yes & \textbf{R20} & MUST & Identify high-level control flow primitives in LLVM IR \\
			\rowcolor{light_green_3}
			Yes & \textbf{R21} & MUST & Support language-agnostic interaction with other components \\
			\hline
		\end{tabular}
	\end{center}
	\caption{A summary of the evaluation against requirements of the control flow recovery tool, which specifies what requirements (abbreviated as ``Req.'') that have been satisfied (abbreviated as ``Sat.'').}
	\label{tbl:eval_summary_of_control_flow_recovery_tool}
\end{table}

% --- [ Subsections ] ----------------------------------------------------------

% ~~~ [ Essential Requirements ] ~~~~~~~~~~~~~~~~~~~~~~~~~~~~~~~~~~~~~~~~~~~~~~~

\subsubsection{Essential Requirements}
\label{sec:eval_llvm_ir_library_essential_requirements}

% * R1 - Read the assembly language representation of LLVM IR
% * R3 - Interact with an in-memory representation of LLVM IR
% * R4 - Generate CFGs from LLVM IR basic blocks

The modified Go bindings for LLVM (see section \ref{sec:impl_go_bindings_for_llvm}) includes read (\textbf{R1}) and write (\textbf{R2}) support for the assembly language representation of LLVM IR, and enables interaction with an in-memory representation of LLVM IR (\textbf{R3}). The \texttt{ll2dot} tool depends on \textbf{R1} and \textbf{R3} for parsing LLVM IR assembly files and inspecting their in-memory representation, which is required to gain access to information about the basic blocks of each function and their terminating instructions. This information determines the node names and the directed edges, when generating CFGs from LLVM IR; as further described in section \ref{sec:design_control_flow_graph_generation}. Appendix \ref{app:control_flow_graph_generation_example} demonstrates that the \texttt{ll2dot} tool is capable of generating CFGs from LLVM IR (\textbf{R4}), thus proving that \textbf{R1}, \textbf{R3} and \textbf{R4} have been satisfied.

% * R2 - Write the assembly language representation of LLVM IR

To support generating CFGs for LLVM IR assembly which contains unnamed basic blocks, the \texttt{ll2dot} tool requires access to the names of unnamed basic blocks. These names are not available from the API of the original Go bindings for LLVM however, as they are generated on the fly by the assembly printer. To work around this issue, the assembly printer of LLVM 3.6 was patched to always print the generated names of unnamed basic blocks (see appendix \ref{app:unnamed_patch}). With this patch in place, the debug facilitites of the modified Go bindings for LLVM could be utilized to write (\textbf{R2}) the assembly to temporary files, which were parsed to gain access to the names of unnamed basic blocks; as further described in section \ref{sec:impl_go_bindings_for_llvm}. The generated CFG presented in appendix \ref{app:control_flow_graph_generation_example} contains the names of unnamed basic blocks (e.g. basic blocks with numeric names), thus proving that \textbf{R2} has been satisfied.


