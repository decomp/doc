% ~~~ [ Reliability Improvements ] ~~~~~~~~~~~~~~~~~~~~~~~~~~~~~~~~~~~~~~~~~~~~~

\subsubsection{Reliability Improvements}
\label{sec:con_reliability_improvements}

As described in section~\ref{sec:impl_go_bindings_for_llvm}, there are many reliability issues caused by the Go bindings for LLVM. To mitigate these issues a pure Go library is being developed for interacting with LLVM IR (see section~\ref{sec:impl_llvm_ir_library}). This library will be reusable by other projects, and the requirements of the third-party Go compiler \texttt{llgo}\footnote{LLVM-based compiler for Go: \url{https://llvm.org/svn/llvm-project/llgo/trunk/README.TXT}} are actively being tracked\footnote{Requirements · Issue \#3: \url{https://github.com/llir/llvm/issues/3}}.

To ensure reliable interoperability between components written in different programming languages, the intermediate representation (i.e. LLVM IR) of the decompilation pipeline must be well-defined. Previous efforts to produce a formal grammar for LLVM IR have only focused on subsets of the language (as mentioned in section~\ref{sec:req_llvm_ir_library}), and no such grammar has been officially endorsed. Producing an official formal specification for LLVM IR would require huge efforts, but it would enable interesting opportunities. For instance, with a formal grammar it would be possible to create a tool which automatically generates grammatically correct LLVM IR assembly which may be used to verify the various implementations of LLVM IR. This approach has been used by the GoSmith tool to generate random, but legal, Go programs which have uncovered 31 bugs in the official Go compiler, 18 bugs in the Gccgo compiler, 5 bugs in the \texttt{llgo} compiler, and 3 bugs in the Go language specification~\cite{gosmith}.
