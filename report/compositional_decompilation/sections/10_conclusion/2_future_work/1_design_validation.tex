% ~~~ [ Design Validation ] ~~~~~~~~~~~~~~~~~~~~~~~~~~~~~~~~~~~~~~~~~~~~~~~~~~~~

\subsubsection{Design Validation}
\label{sec:con_design_validation}

The principle of separation of concern has influenced every aspect of the design of the decompilation pipeline and its individual components. Conceptually, the components of the decompilation pipeline are grouped into three modules which separate concerns regarding the source language (front-end module), the general decompilation tasks (middle-end module), and the target language (back-end module). This conceptual separation is a vital aspect of the decompilation pipeline design, and it will therefore be thoroughly examined. Should a component violate the principle of separation of concern, either in isolation or within the system as a whole, it must be redesigned or reimplemented. To identify such issues, key areas of the decompilation pipeline will be extended to put pressure on the design.

Firstly, an additional back-end (e.g. support for Python as a target language) will be implemented to put pressure on the design of the middle-end module. The second back-end would only be able to leverage the target-independent information of the general decompilation tasks (e.g. control flow analysis) if the middle-end module was implemented correctly.

Secondly, a key component (e.g. data flow analysis) will be implemented in a separate programming language (e.g. Haskell, Rust, Prolog, …) to validate the language-agnostic aspects of the design. This component would only be able to interact with the rest of the decompilation pipeline, through well-defined input and output (e.g. LLVM IR, JSON, DOT, …), if the other components were implemented correctly.

The separation of the front-end and the middle-end has already been validated. These modules are only interacting through an intermediate representation (i.e. LLVM IR), and a variety of source languages are already supported by the front-end module which consists of components from several independent open source project (e.g. Dagger, Fracture, MC-Semantic, Clang, …).

The design of the control flow analysis component has both advantages and limitations, as discussed in section~\ref{sec:design_control_flow_analysis}. The most significant limitation is the lack of support for control flow primitives with a variable number of nodes in their graph representations (e.g. n-way conditionals). To gain a better understanding of this issue, an analysis of control flow primitives from different high-level languages will be conducted. Should the n-node control flow primitives prove to be rare, hard-coded support for n-way conditionals (e.g. \texttt{switch}-statements) and similar control flow primitives would suffice. Otherwise, a general solution to the problem will be required (such as the introduction of a domain specific language which describes dynamic properties of the nodes and edges in DOT files). The OpenCL decompiler presented by S. Moll solved this problem by converting n-way conditionals into sets of 2-way conditionals~\cite{decomp_of_llvm}.

As described in section~\ref{sec:eval_control_flow_analysis_library}, the current implementation of the control flow analysis component enforces a single-entry/single-exit invariant on the graph representations of high-level control flow primitives. This invariant prevents the recovery of infinite loops, as their graph representation has no exit node. At this stage it is unclear whether a refined implementation may relax the invariant to support single-entry/no-exit graphs, or if the limitation is inherent to the design. This issue requires further investigation before a potential solution may be proposed.
