% --- [ Subgraph Isomorphism Search Library ] ----------------------------------

\subsection{Subgraph Isomorphism Search Library}
\label{sec:impl_subgraph_isomorphism_search_library}

Implementing the subgraph isomorphism search algorithm was without doubt the most difficult endeavour of the entire project. It took roughly five iterations of implementing, evaluating and rethinking the algorithm to find an approach which felt right and another two iterations to develop a working implementation which passed all the test cases.

As mentioned in section~\ref{sec:ver_performance}, an early throwaway prototype provided a partial implementation of the subgraph isomorphism algorithm proposed by Ullman. The prototype was intended to provide insight into the subgraph isomorphism problem domain, and was eventually discarded.

The second throwaway prototype was specifically designed to exploit known properties of CFG (e.g. connected graphs with a single entry node) to limit the search space. Focusing on connected graphs drastically simplified the general problem of subgraph isomorphism search, and enabled algorithms which traverse the graph from a given start node to identify subgraph isomorphisms. The second prototype had many issues (e.g. non-deterministic, unable to handle graph cycles), but provided valuable insight into how a subgraph isomorphism search algorithm may be designed when focusing on connected graphs.

In contrast to its predecessor, the third prototype separated subgraph isomorphism candidate discovery from candidate validation logic. A subgraph isomorphism candidate is a potential isomorphism of a subgraph in a graph, which provides a mapping from subgraph node names to graph node names. Should the source and the destination nodes of each directed edge in the subgraph translate through the candidate node mapping to corresponding nodes (with a directed edge from the source to the destination node) in the graph, and should furthermore each node in the subgraph have the same number of directed edges as the nodes of the candidate (with a few caveats regarding entry and exit nodes), then the candidate is considered a valid isomorphism of the subgraph in the graph. The third prototype was still incomplete (mainly with regards to candidate discovery) when discarded, but the separation of candidate discovery and candidate validation logic has had a large influence on its succeeding prototypes.

As described in section~\ref{sec:design_control_flow_analysis} and further evaluated in section~\ref{sec:eval_control_flow_analysis_library_essential_requirements}, the current implementation of the subgraph isomorphism search algorithm enforces a single-entry/single-exit invariant on the subgraphs to simplify control flow analysis. This allows identified subgraphs to be replaced with single nodes, which inherit the predecessors of the subgraph entry node and successors of the subgraph exit node. For this reason, the candidate validation logic disregards the directed edges from predecessors of subgraph entry nodes and the directed edges to successors of subgraph exit nodes, when validating subgraph isomorphism candidates; which should clarify the aforementioned caveats of the preceding paragraph.

Similar to the third prototype, the fourth throwaway prototype separated candidate discovery from candidate validation. In addition, it introduced the concept of treating candidate node mappings as equations which may be solved, or at least partially solved. The candidate node mappings were extended from one-to-one node mappings (one subgraph node name maps to exactly one graph node name) to node pair mappings, which represent one-to-many node mappings (one subgraph node name maps to zero or more graph node names). The candidate discovery logic was extended to record all potential candidate nodes for a given subgraph node, when traversing the graph in search of candidates. A simple equation solver was implemented which was capable of identifying unique node pair mappings and propagate this information to successively simplify equations until they are either solved or require other methods for solving; an example of which is presented in figure~\ref{fig:equation_unique}. The equation solver would however fail to find a solution if two node pair mappings had the same candidate nodes, as illustrated in figure~\ref{fig:equation_fail}.

\begin{figure}[htbp]
	\centering
	\begin{subfigure}[t]{0.30\textwidth}
		\lstinputlisting[language=go, style=go, breaklines=false, numbers=none]{inc/7_impl/equation_unique_1.json}
		\caption{Step 1.}
	\end{subfigure}
	\enskip
	\begin{subfigure}[t]{0.23\textwidth}
		\lstinputlisting[language=go, style=go, breaklines=false, numbers=none]{inc/7_impl/equation_unique_2.json}
		\caption{Step 2.}
	\end{subfigure}
	\enskip
	\begin{subfigure}[t]{0.17\textwidth}
		\lstinputlisting[language=go, style=go, breaklines=false, numbers=none]{inc/7_impl/equation_unique_3.json}
		\caption{Step 3.}
	\end{subfigure}
	\caption{In step 1, the unique node pair mapping between the subgraph node name \textit{C} and the graph node name \textit{Z} is identified, and the remaining node pair mappings are simplified by removing \textit{Z} from their candidate nodes. Similarly, in step 2, the unique node pair mapping between \textit{B} and \textit{Y} is identified; thus simplifying the equation further. Lastly, in step 3, the unique node pair mapping between \textit{A} and \textit{X} is identified, and the equation is thereby solved.}
	\label{fig:equation_unique}
\end{figure}

\begin{figure}[htbp]
	\centering
	\begin{subfigure}[ht]{0.23\textwidth}
		\lstinputlisting[language=go, style=go, breaklines=false, numbers=none]{inc/7_impl/equation_fail.json}
	\end{subfigure}
	\caption{An equation which the simple equation solver of the forth prototype would fail to solve, as it cannot be simplified by identifying unique node pair mappings.}
	\label{fig:equation_fail}
\end{figure}

The fifth throwaway prototype extended the capabilities of the simple equation solver by trying different candidate node mappings recursively until a valid subgraph isomorphism was found or known not to exist. These equations were solved concurrently using Go routines (independently executing functions, which are multiplexed onto system threads) which relayed the answers back using channels (typed and synchronised communication channels).

At this stage, the algorithm design had started to feel mature and the focus shifted from implementing throwaway prototypes to building a solid foundation. Starting with the parts of the system that were best understood, one part or concept at the time were removed from the throwaway prototype and carefully reimplemented in a new library through a series of steps. Firstly, the API of the new library was taken into careful consideration, and a set of stub functions and core data structures were added and thoroughly documented. Secondly, test cases were written for each stub function of the library. Lastly, the stub functions were implemented one at the time and verified against the test cases.

While the new implementation passed most test cases, there were a few corner cases for which the library produced incorrect results. The concurrent nature of the library made it difficult to debug, and a decision was made to reimplement the equation solver without concurrency. This resulted in a cleaner implementation which was easy to debug and successfully passed all test cases.

The final implementation of the subgraph isomorphism search algorithm is a cleaned up and thoroughly tested version of the non-concurrent library, which has a 94.8\% code coverage; as further described in section~\ref{sec:ver_code_coverage}. The final implementation was developed in the \textit{``isobug''} branch on GitHub, and merged\footnote{Fix subgraph isomorphism search: \url{https://github.com/decomp/decomp/issues/183}} once stable into the \textit{``master''} branch.
