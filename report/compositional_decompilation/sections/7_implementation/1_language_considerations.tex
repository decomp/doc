% --- [ Language Considerations ] ----------------------------------------------

% <howto>
% * choice of programming language(s) (implementation)

\subsection{Language Considerations}

As stated by H. Mayer in 1989, \textit{``No programming language is perfect. There is not even a single best language; there are only languages well suited or perhaps poorly suited for particular purposes. Understanding the problem and associated programming requirements is necessary for choosing the language best suited for the solution.''}~\cite{no_perfect_lang_quote} This project seeks to explore the potential of a compositional approach to decompilation, and the components of the decompilation pipeline will require support for analysing and manipulating source code, and interacting with LLVM IR. The Go programming language emphasises composition at its core and provides extensive support for source code analysis, as indicated by the vast number of tools developed for analysing and manipulating Go source code (examples of such tools are given in section~\ref{sec:ver_continuous_integration}). As the LLVM compiler framework is written in C\texttt{++}, several projects (e.g. Dagger, Fracture, MC-Semantics) have chosen this language for interacting with LLVM IR. Meanwhile, a mature LLVM IR library is yet to be written for Go.

In 2012 Rob Pike (one of the Go language inventors) gave a talk titled \textit{``Less is exponentially more''} which included a personal description of the historic events leading up to the inception of Go. The starting point of the language was C, not C\texttt{++}, which Go aimed to simplify further by removing cruft. This is in direct contrast to the direction of C\texttt{++} which gains more features with each passing release. The \textit{less is more} mindset is deeply rooted in the mentality of Go developers, and there is a strong emphasis on the use of composition to solve problems, as indicated by the following extract from Rob Pike's talk.

\begin{quote}
	\itshape
	``If C\texttt{++} and Java are about type hierarchies and the taxonomy of types, Go is about composition.

	Doug McIlroy, the eventual inventor of Unix pipes, wrote in 1964~(!):

	\begin{quote}
		``We should have some ways of coupling programs like garden hose--screw in another segment when it becomes necessary to massage data in another way. This is the way of IO also.''
	\end{quote}

	That is the way of Go also. Go takes that idea and pushes it very far. It is a language of composition and coupling.'' \\

	\normalfont
	--- Rob Pike, 2012~\cite{less_is_more}
\end{quote}

Every aspect of Go development embodies the Unix philosophy (see figure~\ref{fig:unix_philosophy}), which is no surprise as Ken Thompson (one of the original inventors of Unix) is part of the core Go team.

\begin{figure}[htbp]
	\begin{center}
		\begin{quote}
			\textit{``Write programs that do one thing and do it well. Write programs to work together.''}
		\end{quote}
		\caption{The Unix philosophy~\cite{art_of_unix}.}
		\label{fig:unix_philosophy}
	\end{center}
\end{figure}

To conclude the language considerations, Go has been chosen as the primary language for the decompilation pipeline based on its simplicity and emphasis on composition. Furthermore, the Go standard library includes production quality support for lexing and parsing of Go source code. The language may therefore be a good candidate for developing lexers and parsers for LLVM IR, as will be further discussed in section~\ref{sec:impl_llvm_ir_library}.
