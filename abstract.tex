\begin{abstract}

% TODO: Merge the four parts into a single paragraph?

% --- [ Background and purpose ] -----------------------------------------------

% 66 words.

Decompilation or reverse compilation is the process of translating low-level machine-readable code into high-level human-readable code. The problem is non-trivial due to the amount of information lost during compilation, but it can be divided into several smaller problems which may be solved independently. This report explores the feasibility of composing a decompilation pipeline from independent components, and the potential of exposing those components to the end-user.

% --- [ Methodology ] ----------------------------------------------------------

% 62 words.

The components of the decompilation pipeline are conceptually grouped into three modules. The front-end translates a source language (e.g. x86 assembly) into LLVM IR; a platform-independent low-level intermediate representation. The middle-end structures the LLVM IR by identifying high-level control flow primitives (e.g. pre-test loops, 2-way conditionals). Finally, the back-end translates the structured LLVM IR into a high-level target programming language (e.g. Go).

% --- [ Results ] --------------------------------------------------------------

% 72 words.

The decompilation pipeline has been proven capable of recovering nested pre-test and post-test loops (e.g. \texttt{while}, \texttt{do-while}), and 1-way and 2-way conditionals (e.g. \texttt{if}, \texttt{if-else}) from LLVM IR. The control flow analysis stage of the middle-end can easily be extended to identify new control flow primitives. It uses subgraph isomorphism search algorithms to locate control flow primitives in control flow graphs, both of which are described using the Graphviz DOT file format.

% --- [ Conclusion ] -----------------------------------------------------------

% 96 words.

\textbf{QUESTION}: \textit{Does the future development part belong in the abstract? Is there any other part that an abstract would normally include that is missing?}

There exists huge potential for future development. Investigate adding further post-processing stages to make the Go output more idiomatic; e.g. the \texttt{grind} tool by Russ Cox moves variable declarations closer to their usage. Finish developing the LLVM IR library to replace the C++ bindings with a pure Go implementation; consider \texttt{llgo} a potential future user. Validate the language-agnostic aspects of the design by implementing components in other languages; e.g. data flow analysis in Haskell. Verify that the middle-end handles heavy decompilation tasks (i.e. control flow analysis, data flow analysis) by implementing additional back-ends (e.g. Python output).

\end{abstract}
